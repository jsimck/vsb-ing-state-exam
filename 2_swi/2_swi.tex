\documentclass[11pt]{article}

% Packages
\usepackage[czech]{babel}
\usepackage[utf8]{inputenc}
\usepackage[useregional]{datetime2}
\usepackage[T1]{fontenc}
\usepackage[a4paper, total={15.24cm, 23.32cm}]{geometry}
\usepackage[thinlines]{easytable}
\usepackage{graphicx}
\usepackage[ampersand]{easylist}
\usepackage{changepage}
\usepackage{float}
\usepackage{color}
\usepackage{enumitem}

% Config
\setlength\parindent{0pt}
\renewcommand{\baselinestretch}{1.2} 
\setitemize{itemsep=0pt}

\title{\textbf{II. Softwarové inženýrství}}
\date{\small\vspace{-9ex}Update: \today}

\begin{document}
\maketitle

\section{Softwarový proces. Jeho definice, modely a úrovně vyspělosti.}

\pagebreak
\section{Vymezení fáze „sběr a analýza požadavků“. Diagramy UML využité v dané fázi.}

\pagebreak
\section{Vymezení fáze „Návrh“. Diagramy UML využité v dané fázi. Návrhové vzory – členění, popis a příklady.}

\pagebreak
\section{Objektově orientované paradigma. Pojmy třída, objekt, rozhraní. Základní vlastnosti objektu a vztah ke třídě. Základní vztahy mezi třídami a rozhraními. Třídní vs. instanční vlastnosti.}

\pagebreak
\section{Mapování UML diagramů na zdrojový kód.}

\pagebreak
\section{Správa paměti (v jazycích C/C++, JAVA, C\#, Python), virtuální stroj, podpora paralelního zpracování a vlákna.}

\pagebreak
\section{Zpracování chyb v moderních programovacích jazycích, princip datových proudů – pro vstup a výstup. Rozdíl mezi znakově a bytově orientovanými datovými proudy.}

\pagebreak
\section{Jazyk UML – typy diagramů a jejich využití v rámci vývoje.}

\end{document}
