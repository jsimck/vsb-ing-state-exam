\documentclass[11pt]{article}

% Packages
\usepackage[czech]{babel}
\usepackage[utf8]{inputenc}
\usepackage[useregional]{datetime2}
\usepackage[T1]{fontenc}
\usepackage[a4paper, total={15.24cm, 23.32cm}]{geometry}
\usepackage[thinlines]{easytable}
\usepackage{graphicx}
\usepackage[ampersand]{easylist}
\usepackage{changepage}
\usepackage{float}
\usepackage{color}
\usepackage{enumitem}
\usepackage{latexsym}
\usepackage{hyperref}
\usepackage{amssymb}
\usepackage{amsmath}
\usepackage{amsfonts}
%\usepackage{minted}
\usepackage{caption}
\usepackage{amsfonts}
\usepackage{makecell}
\usepackage{tikz}
\usepackage{enumitem}
\usepackage{multirow}

% Config
\renewcommand{\baselinestretch}{1.2} 
\setitemize{itemsep=0pt}
\setenumerate{itemsep=0pt}
\hypersetup{
	colorlinks,
	citecolor=black,
	filecolor=black,
	linkcolor=black,
	urlcolor=black
}
\title{\vspace{-10ex}\textbf{I. Matematické základy informatiky}}
\date{\small\vspace{-9ex}Update: \today}
%\setminted{fontsize=\small,baselinestretch=1}
\newcommand{\hhline}{\Xhline{2\arrayrulewidth}}
\newcommand{\resline}{\vspace{1.5mm}\hrule\vspace{1.5mm}}
\DeclareUnicodeCharacter{2212}{-}
\begin{document}
\maketitle
\setcounter{tocdepth}{1}
\tableofcontents{\small\vspace{-15ex}}
\pagebreak
\section{Konečné automaty, regulární výrazy, uzávěrové vlastnosti třídy regulárních jazyků.}
\textbf{Softwarové inženýrství} je inženýrská disciplína zabývající se praktickými problémy vývoje
rozsáhlých softwarových systémů.

\section*{Softwarový proces}
\textbf{Softwarový proces} je po částech uspořádaná množina kroků směřujících k vytvoření nebo úpravě softwarového díla.
\begin{itemize}
\item Krokem může být \textbf{aktivita} nebo opět \textbf{podproces} (hierarchická dekompozice procesu). 
\item Aktivity a podprocesy mohou \textbf{probíhat v čase souběžně}, tudíž je vyžadována jejich koordinace. 
\item Je \textbf{nutné zajistit opakovatelnost použití procesu ve vztahu k jednotlivým softwarovým projektům}, tedy zajistit jeho \textbf{znovupoužitelnost}.  Cílem je dosáhnout stabilních výsledků vysoké úrovně kvality.
\item Řada činností je zajišťována lidmi vybavenými určitými schopnostmi a znalostmi a majícím k dispozici technické prostředky nutné pro realizaci těchto činností.
\item \textbf{Softwarový produkt} je realizován v kontextu organizace s danými ekonomickými možnostmi a organizační strukturou.
\end{itemize}

\subsection{Vyspělost úrovně}
Úroveň definice a využití softwarového procesu je hodnocena dle stupnice \textbf{SEI (Software Engineering Institute)} \textbf{1 - 5} vyjadřující vyspělost firmy či organizace z daného hlediska. Tento model hodnocení vyspělosti a schopností dodavatele softwarového produktu se nazývá \textbf{CMM (Capability Maturity Model)} a jeho jednotlivé úrovně lze stručně charakterizovat asi takto:

\begin{enumerate}
\item \textbf{Počáteční (Initial)} - firma \textbf{nemá definován softwarový proces} a každý projekt je řešen \textbf{případ od případu} (ad hoc).
\item \textbf{Opakovatelná (Repeatable)} - firma identifikovala v jednotlivých projektech \textbf{opakovatelné postupy} a tyto je schopna reprodukovat v každém novém projektu.
\item \textbf{Definovaná (Defined)} - softwarový proces je \textbf{definován (a dokumentován)} na základě integrace dříve identifikovaných opakovatelných kroků.
\item \textbf{Řízená (Managed)} - na základě definovaného softwarového procesu je firma schopna jeho \textbf{řízení} a \textbf{monitorování}.
\item \textbf{Optimalizovaná (Optimized)} - zpětnovazební informace získaná \textbf{dlouhodobým procesem monitorování} softwarového procesu je využita ve prospěch jeho optimalizace.
\end{enumerate}

\subsection*{Modely softwarového procesu}
\subsection{Vodopádový model}
Základem téměř všech modelů softwarového procesu se stal vodopádový model. Tento vodopádový model vychází z \textbf{rozdělení životního cyklu softwarového díla} na čtyři základní fáze:
\begin{enumerate}
\item Analýza požadavků a jejich specifikace.
\item Návrh softwarového systému.
\item Implementace.
\item Testování a udržování vytvořeného produktu.
\end{enumerate}
Princip vodopádu spočívá v tom, že \textbf{následující množina činností spjatá s danou fází} nemůže započít dříve, než skončí předchozí. Jinými slovy řečeno, výsledky předchozí fáze „vtékají“ jako vstupy do fáze následující.
\\\\
\noindent\makebox[\textwidth]{\includegraphics[width=12cm]{assets/swi1}}
Model je možno v \textbf{různých modifikacích} a \textbf{rozšířeních} nalézt ve většině současných přístupů. Tyto modifikace vznikly především kvůli odstranění některých jeho \textbf{nedostatků}, mezi které patří:
\begin{itemize}
\item \textbf{Prodleva} mezi zadáním projektu a vytvořením spustitelného systému je příliš dlouhá.
\item \textbf{Výsledek závisí} na \textbf{úplném a korektním zadaní požadavků} kladených na výsledný produkt.
\item \textbf{Nelze odhalit výslednou kvalitu produktu} danou splněním všech požadavků, dokud není výsledný softwarový systém hotov.
\end{itemize}

\subsubsection{Modifikace Vodopádového modelu}
\begin{itemize}
\item \textbf{Inkrementální}: \textbf{postupné vytváření verzí} softwaru zahrnujících postupně širší spektrum funkcí definovaných postupně v průběhu jeho vytváření. V podstatě se jedná o více menších vodopádů provedených za sebou tak, aby každý z nich odpovídal nové sadě doplněných požadavků.
\item \textbf{Spirálový}: zahrnuje do svého \textbf{životního cyklu další fáze} jako je vytvoření a hodnocení \textbf{prototypu} ověřující funkcionalitu cílového systému, přičemž \textbf{každý cyklus nabaluje další požadavky} specifikované zadavatelem.
\end{itemize}


\subsection{RUP (Rational Unified Process)}
Rational Unified Process (RUP) je \textbf{metodika vývoje softwaru} vytvořená společností Rational Software Corporation. Je použitelná pro \textbf{jakýkoliv rozsah projektu}, ale díky vysoké rozsáhlosti RUPu je vhodné přizpůsobit metodiku specifickým potřebám. RUP je vhodnější spíš pro \textbf{rozsáhlejší projekty} a \textbf{větší vývojové týmy}, neboť klade důraz na \textbf{analýzu} a \textbf{design}, \textbf{plánování}, \textbf{řízení zdrojů} a \textbf{dokumentaci}.

\subsubsection{Hlavní znaky}
\begin{itemize}
\item Softwarový produkt je vyvíjen \textbf{iteračním způsobem}.
\item Jsou \textbf{spravovány požadavky} na něj kladené.
\item Využívá se \textbf{již existujících softwarových komponent}.
\item Model softwarového systému je \textbf{vizualizován} (\textbf{UML}).
\item Průběžně je \textbf{ověřována} \textbf{kvalita} produktu.
\item \textbf{Změny} systému\textbf{ jsou řízeny} (každá změna je přijatelná a změny jsou sledovatelné).
\end{itemize}
V současné době, kdy se předmětem vývoje staly softwarové systémy vysoké úrovně sofistikace, je \textbf{nemožné nejprve specifikovat celé zadání}, následně navrhnout jeho řešení, vytvořit softwarový produkt implementující toto zadání, vše otestovat a předat zadavateli k užívání. Jediné možné řešení takovému problému je přístup postavený na \textbf{iteracích}, umožňující \textbf{postupně upřesňovat cílový produkt} cestou jeho \textbf{inkrementálního rozšiřovaní} z původní hrubé formy do výsledné podoby.  Softwarový systém je tak \textbf{vyvíjen ve verzích}, které lze průběžně ověřovat se zadavatelem a případně jej pozměnit pro následující iteraci.


\subsubsection{Cykly, fáze, iterace (Stále se váže k RUP)}
Každý \textbf{cyklus vede k vytvoření takové verze systému}, kterou lze \textbf{předat uživatelům a implementuje jimi specifikované požadavky}. Jak již bylo uvedeno v předchozí kapitole, každý takový vývojový cyklus lze rozdělit do \textbf{čtyř} po sobě jdoucích \textbf{fází}: 
\begin{enumerate}
\item \textbf{Zahájení}, kde je původní myšlenka rozpracována do \textbf{vize koncového produktu} a je definován rámec toho, jak celý systém bude vyvíjen a implementován. 
\item \textbf{Rozpracování} je fáze věnovaná \textbf{podrobné specifikaci požadavků} a \textbf{rozpracování architektury} výsledného produktu. 
\item \textbf{Tvorba} je zaměřena na \textbf{kompletní vyhotovení požadovaného díla}.  Výsledné  programové vybavení je vytvořeno kolem navržené kostry (architektury) softwarového systému. 
\item \textbf{Předání} je závěrečnou fází, kdy \textbf{vytvořený produkt je předán do užívání}. Tato fáze zahrnuje i další aktivity jako je beta \textbf{testování}, \textbf{zaškolení} apod. 
\end{enumerate}
Každá fáze může být dále \textbf{rozložena do několika iterací}.

\subsubsection{Iterace}
Iterace je \textbf{úplná vývojová smyčka vedoucí k vytvoření spustitelné verze systému} reprezentující \textbf{podmnožinu} vyvíjeného cílového produktu, a která je \textbf{postupně rozšiřována každou iterací} až do výsledné podoby. 
\\\\
\noindent\makebox[\textwidth]{\includegraphics[width=9cm]{assets/swi2}}


\pagebreak
\section{Bezkontextové gramatiky a jazyky. Zásobníkové automaty, jejich vztah k bezkontextovým gramatikám.}
\subsection{Bezkontextové gramatiky (BG)}
Bezkontextová gramatika definuje \textbf{bezkontextový jazyk}. Je tvořena \textbf{neterminály} (proměnné), \textbf{terminály} (konstanty) a \textbf{pravidly}, které každému neterminálu definují přepisovací pravidla. Jeden neterminál označíme jako \textbf{startovní}, kde začínáme a podle pravidel je dál přepisujeme na výrazy složené z terminálu a neterminálu. Jakmile už není co přepisovat, výraz obsahuje už jen neterminály, získali jsme \textbf{slovo}.

\begin{itemize}
\item Je \textbf{uzavřená} vůči operacím \textbf{sjednocení}, \textbf{zřetězení}, \textbf{iteraci} a \textbf{zrcadlový obraz}.
\item Ke každé bezkontextové gramatice existuje \textbf{ekvivalentní zásobníkový automat}.
\end{itemize}

\subsubsection{Formální definice BG}
Bezkontextová gramatika je definována jako uspořádaná čtveřice $G = (\Pi, \Sigma, S, P)$, kde:
\begin{itemize}
	\item $\Pi$ (\textit{velké pí}) je konečná množina \textbf{neterminálních} symbolů (neterminálů).
	\item $\Sigma$ je konečná množina \textbf{terminálních} symbolů (terminálů), $\Pi \cap \Sigma = \emptyset$.
	\item $S$ je \textbf{počáteční neterminál}, $S \in \Sigma$.
	\item $P$ je konečná množina \textbf{přepisovacích pravidel}, $P \subseteq \Pi \times (\Pi \cup \Sigma)^*$.
\end{itemize}

\subsubsection{Základní pojmy}
\begin{itemize}
\item \textbf{Bezkontextový jazyk} -- formální jazyk, který je akceptovaný nějakým zásobníkovým automatem.
\item \textbf{Derivace slova} -- jedno konkrétní odvození slova pomocí gramatiky, tedy záznam postupných přepisů od startovního neterminálu po konečné slovo. Derivace se podle postupu při přepisování dělí na:
\begin{itemize}
\item \textbf{levou} -- přepisujeme nejprve levé neterminály,
\item \textbf{pravou} -- přepisujeme nejprve pravé neterminály.
\end{itemize}
\item \textbf{Derivačni strom} -- grafické znázornění derivace slova stromem. Pro všechny možné derivace (levou, pravou, moji) by měl derivační strom být \textbf{stejný}. Není-li tomu tak jedná se o \textbf{nejednoznačnou gramatiku}, což je nežádoucí jev. 
\begin{itemize}
	\item \textbf{Špatně} = A $\rightarrow$ A | $\epsilon$ (lze generovat až N způsoby), \textbf{Správně} = A $\rightarrow \epsilon$ 
\end{itemize}
\item \textbf{Chomského normální forma} -- gramatika může obsahovat pouze pravidla typu: \textbf{A $\rightarrow$ BC} nebo \textbf{A $\rightarrow$ a} nebo \textbf{S $\rightarrow \epsilon$} (pokud gramatika generuje pouze prázdný řetězec).
\item \textbf{Nevypouštějící gramatika} -- neobsahuje $\epsilon$ (\textit{epsilon}) přechody.
\end{itemize}

\begin{figure}[H]
	\centering
	\includegraphics[width=0.6\textwidth]{assets/bg}
\end{figure}

\subsection{Zásobníkové automaty (ZA)}
Slouží k \textbf{rozpoznání bezkontextových jazyků}. S využitím zásobníků si může pamatovat kolik a jaké znaky přečetl, což je potřeba právě k rozpoznání bezkontextového jazyka. Zásobníkový automat je v podstatě konečný automat rozšířený o zásobník. 

\begin{figure}[H]
	\centering
	\includegraphics[width=0.30\textwidth]{assets/za}
	\includegraphics[width=0.30\textwidth]{assets/cfl}
\end{figure}

\begin{itemize}
\item ZA na základě \textbf{aktuálního znaku} na pásce, \textbf{prvního znaku v zásobníku} a \textbf{aktuálního stavu} změní svůj stav a \textbf{přepíše} znak v zásobníku podle daných pravidel.
\item ZA \textbf{přijímá} dané slovo, jestliže skončí v konfiguraci $(q, \epsilon, \epsilon)$, tedy když se přečte celé vstupní slovo a zásobník je \textbf{prázdný}.
\item \textbf{Konfigurace} je dána: aktuálním stavem, obsahem pásky a obsahem zásobníku.
\item \textbf{Deterministický} -- nesmí se objevit dvě pravidla se stejnou levou stranou, a pokud existuje
pravidlo $(q, \epsilon, X)$, tak nesmí zároveň existovat pravidlo $(q, a, Y)$. Pokud se deterministický ZNKA ocitne
ve stejné konfiguraci více než jednou, tak obsahuje nekonečný cyklus.
\end{itemize}

\subsubsection{Formální definice zásobníkového automatu}
Zásobníkový automat $M$ je definován jako šestice $M = (Q, \Sigma, \Gamma, \delta, q_0, Z_0)$, kde:
\begin{itemize}
\item $Q$ je konečná neprázdná množina \textbf{stavů}.
\item $\Sigma$ je konečná neprázdná množina \textbf{vstupních symbolů} (vstupní abeceda).
\item $\Gamma$ (\textit{velká gamma}) je konečná neprázdná množina \textbf{zásobníkových symbolů}.
\item $\delta$ je \textbf{přechodová funkce} (konečná množina instrukcí), $\delta: Q \times (\Sigma \cup \{\epsilon\}) \times \Gamma \rightarrow P_{\rm fin}(Q \times \Gamma^*)$.
\item $q_0$ je \textbf{počáteční stav}, $q_0 \in Q$.
\item $Z_0$ je \textbf{počáteční zásobníkový symbol}, $Z_0 \in \Gamma$.
\end{itemize}

\subsubsection{Definice instrukcí (pravidel) v ZA}
Instrukce (sady instrukcí reprezentují přechodovou funkci $\delta$) definují \textbf{chování automatu}:
\begin{equation}
(q, a, X) \rightarrow (q', \alpha)\textrm{, kde } a\in \Sigma.
\end{equation}
Tato instrukce je aplikovatelná jen v situaci (neboli konfiguraci), kdy \textbf{řídicí jednotka} je ve stavu $q$, \textbf{čtecí hlava} na vstupní pásce čte symbol $a$ a na vrcholu zásobníku je symbol $X$. Pokud je \textbf{instrukce aplikována}, vykoná se následující:
\begin{enumerate}
\item řídicí jednotka \textbf{přejde do stavu} $q'$,
\item čtecí hlava na vstupní pásce se \textbf{posune o jedno políčko doprava},
\item vrchní symbol v zásobníku se \textbf{odebere} (vymaže),
\item \textbf{na vrchol zásobníku se přidá} řetězec $\alpha$ tak, že jeho nejlevější symbol je aktuálním vrcholem zásobníku.
\end{enumerate}

\begin{table}[H]
	\vspace{-2mm}
	\centering
	\begin{tabular}{l|l|p{6.5cm}}
		\textbf{Pravidlo} & \textbf{Akce (Z = zásobník)} & \textbf{Význam} \\\hhline
		$ \delta(q_1, a, X) \rightarrow (q_1, YX) $ 	&                 \textbf{přidání} prvku do Z & na začátek zásobníku se vloží $Y$ \\ 
		$ \delta(q_1, a, X) \rightarrow (q_1, Y) $	&                  \textbf{přepsání} prvku v Z & první prvek zásobníku se přepíše na $Y$ \\ 
		$ \delta(q_1, a, X) \rightarrow (q_1, \epsilon) $	&                  \textbf{smazání} prvku ze Z & první prvek zásobníku se smaže neboli nahradí prázdným slovem $\epsilon$ \\ 
		$ \delta(q_1, a, X)\rightarrow(q_2, X) $	&                  \textbf{změna stavu}& stav $ q_1 $ se změní na stav $ q_2 $ \\ 
		$\delta(q_1, a, X)\rightarrow\emptyset$	&                  \textbf{pád} automatu & ukončení výpočtu, slovo nebylo přijato \\ 
	\end{tabular}
\end{table}

\subsection{Převod BG na zásobníkový automat}
Využívá se tzv. metody shora-dolů, která obsahuje pouze \textbf{1 stav}:
\begin{enumerate}
\item pro všechny \textbf{neterminály} vypíšu pravidla typu: $(q, \epsilon, A) \rightarrow \{(q, B), (q, C)\}$,
\item všechny \textbf{terminály} přepíšu na pravidla typu: $(q, a, a) \rightarrow (q, \epsilon)$.
\end{enumerate}

\noindent\\\begin{minipage}[t]{0.35\textwidth}
\textbf{Vstupní gramatika:}\\
$S \rightarrow A | B$\\
$A \rightarrow a$\\
$B \rightarrow (c)$\\\smallskip\\
$\Sigma = \{A, B, S\}$\\
$\Gamma = \{a, c, (, )\}$
\end{minipage}
\begin{minipage}[t]{0.65\textwidth}
\textbf{Instrukce, převedené dle výše uvedených pravidel:}\\
$(Q, \epsilon, S) \rightarrow \{(q, A), (q, B)\}$\\
$(Q, \epsilon, A) \rightarrow (q, a)$\\
$(Q, \epsilon, B) \rightarrow (q, (c))$\\\smallskip\\
$(Q, a, a) \rightarrow (q, \epsilon)$\\
$(Q, (, () \rightarrow (q, \epsilon)$\\
$(Q, c, c) \rightarrow (q, \epsilon)$\\
$(Q, ), )) \rightarrow (q, \epsilon)$\\
\end{minipage}


\pagebreak
\section{Matematické modely algoritmů - Turingovy stroje a stroje RAM. Složitost algoritmu, asymptotické odhady. Algoritmicky nerozhodnutelné problémy.}
Ve snaze \textbf{popsat jakýkoliv algoritmus} si vymysleli matematici Turingovy a RAM stroje. Jde o dva různé přístupy (modely) univerzálních počítačů/programovacích jazyků. Jinými slovy těmito stroji lze \textbf{definovat} a \textbf{provést} \textbf{libovolný algoritmus}.

Historicky prvním ,,univerzálním programovacím jazykem'' byl Turingův stroj. Byl popsán dříve, ještě před rozmachem počítačů, proto se od reálného počítače (programování) podstatně liší, na rozdíl od RAM stroje. Turingův stroj například pracuje s \textbf{celou abecedou} zatímco RAM (podobně jako počítač) s \textbf{čísly}.

\subsection{Turingův stroj (TS)}
\begin{figure}[H]
	\centering
	\includegraphics[width=0.8\textwidth]{assets/turing}
\end{figure}
Turingův stroj je podobný konečnému automatu, ale má \textbf{oboustranně nekonečnou pásku} (je na ni zapsáno vstupní slovo), místo symbolu $\epsilon$ pro prázdné znaky se používá $\Box$, \textbf{hlava} je \textbf{čtecí} i \textbf{zapisovací} a pohybuje se po pásce v \textbf{obou směrech}.

\subsubsection{Formální definice TS}
Turingův stroj, je definován jako šestice $M = (Q, \Sigma, \Gamma, \delta, q_0, F)$, kde:
\begin{itemize}
\item $Q$ je konečná neprázdná množina \textbf{stavů}.
\item $\Sigma$ je konečná neprázdná množina \textbf{vstupních symbolů} (vstupní abeceda).
\item $\Gamma$ je konečná neprázdná množina \textbf{páskových symbolů}, kde $\Sigma \subseteq \Gamma$ a $\Gamma - \Sigma$ je (přinejmenším) speciální znak $\Box$ (prázdný znak [Blank]).
\item $\delta$ je přechodová funkce, $\delta: (Q - F) \times \Gamma \rightarrow Q \times \Gamma \times \{-1, 0, +1\}$.
\item $q_0$ je \textbf{počáteční stav}, $q_0 \in Q$.
\item $F$ je množina \textbf{koncových stavů}, $F \subseteq Q$.
\end{itemize}

\subsubsection{Definice instrukcí (pravidel) v TS}
Podobně jako ZA lze konkrétní Turingův stroj zadat seznamem instrukcí. Tyto instrukce jsou opět dány přechodovou
funkcí, význam instrukce: $(q, a) \rightarrow (q', a', m)$ je tento:
\begin{center}
(akt. stav [$q$], znak na pásce [$a$]) $\rightarrow$ (\textbf{nový stav} [$q'$], \textbf{nový znak} [$a'$], \textbf{posun} [$\{-1;0;+1\}$])
\end{center}
\subsubsection{Příklad}
Tento příklad invertuje slovo, které je uvedené na úvodním obrázku u TS:
\begin{center}
\begin{minipage}[t]{0.3\textwidth}
$Q = \{q_1, q_2\}$\\
$\Sigma = \{a, b, c, \Box\}$\\
$q_0 = q_1$\\
$F = \{q_2\}$\\
\end{minipage}
\begin{minipage}[t]{0.3\textwidth}
$(q_1, a) \rightarrow (q_1, b, +)$\\
$(q_1, b) \rightarrow (q_1, a, +)$\\
$(q_1, \Box) \rightarrow (q_2, \Box, 0)$
\end{minipage}
\end{center}

\subsubsection{Modifikace TS}
\begin{itemize}
\item \textbf{N-páskový TS} -- \textbf{čte} a \textbf{zapisuje} do \textbf{více pásek} najednou, jediná změna je v přechodové funkci: $\delta :Q\times \Gamma ^{n}\rightarrow Q\times (\Gamma \times \{L,R,N\})^{n}$.
\item \textbf{N-hlavový TS} -- má více čtecích hlav než klasický TS, každá hlava zapisuje/čte a pohybuje se \textbf{nezávisle na ostatních}.
\item \textbf{Nedeterministický TS} -- umožňuje výběr z více možností, pro jednu konfiguraci můžeme definovat \textbf{více pravidel}.
\end{itemize}

\subsubsection{Základní pojmy}
\begin{itemize}
\item \textbf{Turingovsky úplný} -- stroj (počítač, programovací jazyk, úloha, ...), která má stejnou výpočetní sílu jako TS. Lze v něm \textbf{odsimulovat} libovolný jiný TS zadaný na vstupu.
\item \textbf{Church-Turingova teze} -- říká, že jakýkoliv výpočet lze úspěšně uskutečnit algoritmem běžícím na počítači, tedy ,,ke každému algoritmu existuje ekvivalentní TS''.
\end{itemize}

\subsection{Model RAM (Random Access Machine)}
RAM stroje již vycházejí ze skutečných počítačů, dá se tedy říct, že se jedná o jednoduchou abstrakci reálného procesoru s jeho strojovým kódem pracujícím s lineárně uspořádanou pamětí. Tento model slouží zejména k analýzy algoritmů z hlediska (\textbf{paměťové}, \textbf{časové}) \textbf{složitosti}. Skládá se z těchto částí:
\begin{enumerate}
\item \textbf{Programová jednotka} -- uchovává program, tvořený konečnou posloupností instrukcí.
\item \textbf{Neomezená pracovní paměť} -- neomezená lineárně uspořádaná paměť, tvořená buňkami, do který lze zapisovat/číst celá čísla ($\mathbb{Z}$), adresovaná přirozenými čísly ($\mathbb{N}$) (0 = \textbf{pracovní} registr, 1 = \textbf{indexový} registr).
\item \textbf{Vstupní a výstupní páska} -- lze na ně sekvenčně zapisovat/číst celá čísla ($\mathbb{Z}$).
\item \textbf{Centrální jednotka} -- obsahuje programový register ukazující, která instrukce má být provedena. Ta se provede a programý registr se příslužně změní (zvýší o 1, o více v případě skoku).
\end{enumerate}

\begin{figure}[!ht]
\centering
\includegraphics[width=0.63\textwidth]{assets/ram}
\end{figure}

Výše uvedený program vypočítá \textbf{aritmetický průměr}, který následně uloží do buňky paměti pod indexem č. \textbf{2}. Výsledek po dělení je roven $4,666$ a po zaokrouhlení $5$.

\subsubsection{Instrukce a typy operandů RAM}
\begin{table}[H]
\centering
\begin{tabular}{l|l}
\textbf{Typ} & \textbf{Hodnota operandu} \\ \hhline	
$=i$ & přímo číslo udané zápisem $i$  \\
$i$ & číslo obsažené v buňce s adresou $i$   \\
$*i$ & číslo v buňce s adresou $i + j$, kde $j$ je aktuální obsah indexového registru
\end{tabular}
\end{table}

\begin{table}[H]
\centering
\begin{tabular}{l|l}
\textbf{Zápis} & \textbf{Význam} \\ \hhline	
\texttt{READ} & do pracovního registru (PR) se \textbf{načte} vstup a hlava se posune
doprava \\
\texttt{WRITE} & na výstup se \textbf{zapíše} hodnota PR \\
\texttt{LOAD} $op$ & do PR se \textbf{načte} hodnota dána operátorem $op$  \\
\texttt{STORE} $op$ & hodnota PR se \textbf{uloží} na do registru daného operátorem $op$ \\
\texttt{ADD} $op$ & k hodnotě PR se \textbf{přičte} hodnota daná operátorem $op$  \\
\texttt{SUB} $op$ &  od hodnoty v PR se \textbf{odečte} hodnota daná operátorem $op$ \\
\texttt{MUL} $op$ &  PR se \textbf{vynásobí} hodnotou danou operátorem $op$ \\
\texttt{DIV} $op$ &  PR se \textbf{vydělí} hodnotou danou operátorem $op$ \\
\texttt{JUMP} \textit{návěští} & provede se \textbf{skok} na instrukci danou \textit{návěštím} \\
\texttt{JZERO} \textit{návěští} & pokud je hodnota v \textbf{PR rovna 0}, provede se skok na \textit{návěští} \\
\texttt{JGTZ} \textit{návěští} & pokud je hodnota v \textbf{PR větší než 0}, provede se skok na \textit{návěští} \\
\texttt{HALT} & \textbf{korektní ukončení} programu \\

\end{tabular}
\end{table}

\subsection{Složitost algoritmů}
Abychom mohli \textbf{porovnávat} různé algoritmy řešící stejný problém, zavádí se pojem složitost algoritmu. Složitost je jinak řečeno \textbf{náročnost algoritmu} -- čím menší složitost tím je algoritmus lepší.
Přičemž nás může zajímat složitost z pohledu \textbf{času}, či \textbf{paměti}:
\begin{itemize}
\item \textbf{Časová složitost} -- sleduje jak závisí \textbf{doba} výpočtu alg. na množství vstupních dat.
\item \textbf{Prostorová složitost} -- sleduje jak závisí \textbf{množství použité paměti} výpočtu alg. na množství vstupních dat.
\end{itemize}
Jelikož konkrétní čísla (čas, bity) se liší v \textbf{závislosti vstupních datech}, množství zpracovávaných dat a použitém programovacím jazyku, neudává se složitost čísly, nýbrž \textbf{funkcí závislou na velikosti vstupních dat}. Tato funkce se získá počítáním proběhlých instrukcí algoritmu sestaveném v univerzálním RAM stroji. A počítá se s nejhorším možným případem vstupu. To je důležité například u třídících algoritmů, kde hraje velkou roli to, jak moc už je vstupní pole setříděné (vstupuje-li do algoritmu už setříděná posloupnost čísel, algoritmus skončí okamžitě, zatímco s opačně seřazenými čísly se bude trápit dlouho.

\subsection{Asymptotická notace}
Je \textbf{způsob klasifikace počítačových algoritmů}. Ve většině případů nemusíme znát přesný počet provedených instrukcí a spokojíme se pouze s odhadem toho, jak rychle tento počet narůstá se zvyšujícím se vstupem. Asymptotická notace nám umožní \textbf{zanedbat méně důležité detaily} a \textbf{odhadnout} přibližně, \textbf{jak rychle daná funkce roste}.
 V souvislosti s asymptotickými odhady složitosti se používjí tyto zapisy:
\begin{itemize}
	\item $f \in O(f)$ -- $f$ roste \textbf{nejvýše tak rychle} jako $g$ ($f$ je ohraničena $g$ \textbf{shora}) $[\leq]$.
	\item $f \in o(g)$ -- $f$ roste \textbf{(striktně) pomaleji} než $g$ ($f$ je ohraničena $g$ \textbf{shora ostře}) $[<]$.
	\item $f \in \Theta (g)$ -- $f$ roste \textbf{stejně rychle} jako $g$ $[=]$.
	\item $f \in \omega(g)$ -- $f$ roste \textbf{(striktně) rychleji} než $g$ ($f$ je ohraničena $g$ \textbf{zdola ostře}) $[>]$.
	\item $f \in \Omega(g)$ -- $f$ roste \textbf{rychleji} než $g$ ($f$ je ohraničena $g$ \textbf{zdola}) $[\geq]$.
\end{itemize}

\begin{minipage}{0.4\textwidth}
\begin{figure}[H]
	\centering
	\includegraphics[width=\textwidth]{assets/asympt}
\end{figure}
\end{minipage}
\begin{minipage}{0.6\textwidth}
\textbf{Seřazeno podle složitosti}:
\small
\begin{itemize}
\item $f(n) \in \Omega(log n)$ -- logaritmická funkce (složitost),
\item $f(n) \in \Omega(n)$ -- lineární funkce (složitost),
\item $f(n) \in \Omega(n^2)$-- kvadratická funkce (složitost),
\item $f(n) \in O(n^k)$ pro nějaké $k > 0$ -- polynomiální,
\item $f(n) \in \Omega(k^n)$ pro nějaké $k > 1$ -- exponenciální.
\end{itemize}
\end{minipage}

\subsubsection{Úskalí asymptotické notace}
Při používání asymptotických odhadů časové složitosti je třeba si uvědomit některá úskalí:
\begin{itemize}
	\item Asymptotické odhady se týkají pouze toho,\textbf{ jak roste čas s rostoucí velikostí vstupu} $\rightarrow$ neříkají nic o \textbf{konkrétní době výpočtu}. V asymptotické notaci mohou být \textbf{skryty velké konstanty}.
	\item Algoritmus, který má lepší asymptotickou časovou složitost než nějaký jiný algoritmus, \textbf{může být ve skutečnosti rychlejší} až pro nějaké hodně velké vstupy.
	\item Většinou analyzujeme složitost v \textbf{nejhorším případě}. Pro některé algoritmy může být doba výpočtu v nejhorším případě mnohem větší než doba výpočtu na „typických“ instancích (typicky Quicksort $\rightarrow$ nejhorší: $O(n^2)$, průměrná: $O(n \log n)$).
\end{itemize}

\subsection{Algoritmicky nerozhodnutelné problémy}
Rozhodovací problém je rozhodnutelný (řešitelný) pokud pro libovolný vstup z množiny vstupů, skončí algoritmus svůj výpočet a vydá správný výstup (tedy jestliže \textbf{existuje turingův stroj, který jej řeší}).

Pokud nalezneme takový vstup, pro který všechny dosavadní algoritmy nejsou schopny nalézt výstup, můžeme tento problém označit za \textbf{nerozhodnutelný}. Speciální případ jsou \textbf{doplňkové problémy}, které vracejí přesně opačné výsledky než původní problém.

\subsubsection{Definice problému}
Problém je určen \textbf{trojicí} $(IN, OUT, p)$, kde:
\begin{itemize}
\item $IN$ je množina (přípustných) \textbf{vstupů},
\item $OUT$ je množina \textbf{výstupů},
\item $p: IN \rightarrow OUT$ je \textbf{funkce} přiřazující každému vstupu odpovídající výstup. 
\end{itemize}

\subsubsection{Ano/Ne problémy}
Jsou to problémy, jejichž \textbf{výstupní množina obsahuje dva prvky} $OUT=\{\rm ano, ne\}$.
 Na ano/ne problémy se dají převést ostatní problémy nepotřebujeme-li znát přesný výsledek:
\begin{itemize}
\item Nepotřebuji najít v poli nejmenší číslo, stačí mi vědět \textbf{zda pole obsahuje číslo menší než nula}.
\item Nepotřebuji znát nejkratší cestu grafem, stačí mi najít cestu, \textbf{která je kratší než 8}.
\end{itemize}

\subsubsection{Riceova věta}
Tato věta ukazuje nerozhodnutelnost celé třídy problémů, její znění je následující ,,\textit{Každá netriviální vstupně/výstupní (I/O) vlastnost programů je \textbf{nerozhodnutelná}}''.

\begin{itemize}
\item Vlastnost X \textbf{je vstupně/výstupní} právě tehdy, když každé dva programy se stejnou I/O tabulkou buď oba vlastnost X mají nebo ji oba nemají.
\item Připomeňme tedy ještě, že vlastnost V je \textbf{triviální}, když ji mají buď všechny programy nebo ji nemá žádný program; taková vlastnost je podle definice také \textbf{vstupně/výstupní}.
\end{itemize}

\begin{table}[H]
	\centering
	\begin{tabular}{l|c|c|c}
		\textbf{Problém}            & \textbf{1} & \textbf{2} & \textbf{3} \\\hhline
		\textbf{Je triviální?}      &     A       &      N      &      N      \\ 
		\textbf{Je I/O?}            &   A         &        A    &        N    \\
		\textbf{Je nerozhodnutelný} &         N   &       A     &           N
	\end{tabular}
\end{table}

\subsubsection{Částečná rozhodnutelnost}
Částečně rozhodnutelný problém, je takový problém, pro který jsme v případě vstupů, u nichž očekáváme odpověď ANO, \textbf{schopni vrátit odpověď ANO}, a v případě NE vrátit buď NE nebo $\perp$ (program se nezastaví a nejsme schopni zjistit, zda by odpověď byla opravdu NE).

\subsubsection{Převeditelnost mezi nerozhodnutelnými problémy}
Důkaz neřešitelnosti lze provést skrze jiné, \textbf{už dokázané}, problémy. Řekneme, že problém $P_1$ je převeditelný na problém $P_2$ (značíme $P_1 \rightsquigarrow P_2$), jestliže alg., který k instanci $I_1$ problému $P_1$ sestrojí instanci $I_2$ problému $P_2$ tak, \textbf{že odpověď $P_1, I_1$ je stejná jako $P_2, I_2$}. Např.: DHP je převeditelný na HP.  Z toho vyplývá, že pokud $P_1$ je nerozhodnutelný tak i $P_2$ je \textbf{nerozhodnutelný}.

\subsubsection{Optimalizační problémy}
Optimalizační problémy \textbf{hledají nejlepší řešení} v množině různých řešení. Příkladem je například: hledání nejkratší cesty, nejmenší kostry, apod.

\subsubsection{Příklady optimalizačních problémů}
\begin{enumerate}
\item \textbf{Hledání nejkratší cesty v grafu}
\begin{itemize}
	\item \textbf{\textsc{Vstup}}: Orientovaný graf $G = (V, E)$ a dvojice vrcholů $u, v \in V$.
	\item \textbf{\textsc{Výstup}}: Nejkratší cesta z $u$ do $v$.
\end{itemize}
\item \textbf{Hledání minimální kostry v grafu}
\begin{itemize}
\item \textbf{\textsc{Vstup}}: Neorientovaný souvislý graf $G = (V_G,E_G)$ s ohodnocenými hranami. 
\item \textbf{\textsc{Výstup}}: Souvislý graf $H = (V_H,E_H)$, kde $V_H = V_G$ a $E_H \subseteq E_G$, který má součet hodnot všech hran minimální.
\end{itemize}
\item \textbf{Eq-CFG (Ekvivalence bezkontextových gramatik)}
\begin{itemize}
\item \textbf{\textsc{Vstup}}: Dvě bezkontextové gramatiky $G1, G2$.
\item \textbf{\textsc{Otázka}}: Platí $L(G1) = L(G2)$? Generují obě gramatiky stejný jazyk?
\end{itemize}
\item \textbf{HP (Problém zastavení [Halting Problem])}
\begin{itemize}
\item \textbf{\textsc{Vstup}}: Turingův stroj $M$ a jeho vstup $w$.
\item \textbf{\textsc{Otázka}}: Zastaví se $M$ na $w$ (tzn. je výpočet stroje M pro vstupní slovo $w$ konečný)?
\end{itemize}
\end{enumerate}


\begin{figure}[H]
	\centering
	\includegraphics[width=0.5\textwidth]{assets/halting}
\end{figure}

\pagebreak
\section{Třídy složitosti problémů. Třída PTIME a NPTIME, NP-úplné problémy.}
\subsection*{Model ISO/OSI}
Počítačové sítě vyvíjelo více firem, zpočátku to byly uzavřené a nekompatibilní systémy. Hlavním účelem sítí je však vzájemné propojování, a tak vyvstala potřeba stanovit pravidla pro přenos dat v sítích a mezi nimi. \textbf{Mezinárodní ústav pro normalizaci ISO} (International Standards Organization) vypracoval tzv. referenční model \textbf{OSI (Open Systems Interconnection)}, který rozdělil práci v síti do \textbf{7 vzájemně spolupracujících vrstev}.
\\\\
Jak již bylo řečeno, model ISO/OSI rozděluje síťovou práci na vrstvy. Princip spočívá v tom, že vyšší vrstva převezme úkol od podřízené vrstvy, zpracuje jej a předá vrstvě nadřízené. Vertikální spolupráce mezi vrstvami (nadřízená s podřízenou) je \textbf{věcí výrobce} sítě. Model \textbf{ISO/OSI doporučuje}, jak mají vrstvy \textbf{spolupracovat horizontálně} – dvě stejné vrstvy modelu mezi různými sítěmi (či síťové prvky různých výrobců) musejí spolupracovat. Model je důležitý především pro výrobce síťových komponent. V praktické práci se sítí jej moc nevyužijeme. Umožňuje však pochopit principy práce síťových prvků a zároveň patří k základní terminologii sítí.

\begin{itemize}
\item\textbf{Aplikační vrstva }- Je \textbf{určitou aplikací} (např. oknem v programu) zpřístupňující uživatelům síťové služby. Nabízí a zajišťuje přístup k souborům (na jiných počítačích), vzdálený přístup k tiskárnám, správu sítě, elektronické zprávy (včetně e-mailu)…
\item\textbf{Prezentační vrstva} - Má na starosti \textbf{konverzi dat}, přenášená data mohou totiž být v různých sítích různě kódována. Tato vrstva zajišťuje sjednocení formy vzájemně přenášených údajů. Dále data komprimuje, případně šifruje… V praxi často splývá s relační vrstvou.
\item \textbf{Relační vrstva} - \textbf{Navazuje} a po skončení přenosu \textbf{ukončuje} \textbf{spojení}. Může provádět \textbf{ověřování} uživatelů, \textbf{zabezpečení} přístupu k zařízením…
\item\textbf{Transportní vrstva} - Tato vrstva\textbf{ zajišťuje přenos dat mezi koncovými uzly}. Jejím účelem je poskytnout takovou kvalitu přenosu, jakou požadují vyšší vrstvy. Vrstva nabízí spojově (TCP) a nespojově orientované (UDP) protokoly. (Platí pouze pro \textbf{TCP/IP})
\item\textbf{Síťová vrstva }- Je zodpovědná za spojení a \textbf{směrování mezi dvěma počítači nebo celými sítěmi} (tj. uzly), mezi nimiž neexistuje přímé spojení. Stará se o síťové adresování.
\item \textbf{Linková (spojová) vrstva }- Poskytuje \textbf{spojení mezi dvěma sousedními systémy}. \textbf{Uspořádává data} z fyzické vrstvy do logických celků známých jako \textbf{rámce} (frames). Seřazuje přenášené rámce, stará se o nastavení parametrů přenosu linky, oznamuje neopravitelné chyby. Formátuje fyzické rámce, opatřuje je fyzickou adresou a poskytuje synchronizaci pro fyzickou vrstvu.
\item\textbf{Fyzická vrstva }- Fyzická vrstva definuje všechny elektrické a fyzikální vlastnosti zařízení. Jakým signálem je reprezentována logická jednička, jak přijímací stanice rozezná začátek bitu, jaký je tvar konektoru, k čemu je který vodič v kabelu použit.
\end{itemize}


Rodina protokolů TCP/IP (Transmission Control Protocol/Internet Protocol) obsahuje \textbf{sadu protokolů} pro komunikaci v počítačové síti a je hlavním protokolem celosvětové sítě \textbf{Internet}. Komunikační protokol je množina pravidel, která určují syntaxi a význam jednotlivých zpráv při komunikaci. Architektura TCP/IP je členěna do čtyř vrstev (na rozdíl od referenčního modelu OSI se sedmi vrstvami):

\begin{enumerate}
	\item \textbf{Vrstva síťového rozhraní} (Network interface)
	\item \textbf{Síťová (IP) vrstva} (Internet layer)
	\item \textbf{Transportní vrstva} (Transport layer)
	\item \textbf{Aplikační vrstva }(Application layer)
\end{enumerate}

\noindent\makebox[\textwidth]{\includegraphics[width=8cm]{assets/4_tcpip}}

Komunikace mezi \textbf{stejnými vrstvami dvou různých systémů} je řízena \textbf{komunikačním protokolem} za použití spojení vytvořeného sousední nižší vrstvou. Architektura umožňuje výměnu protokolů jedné vrstvy bez dopadu na ostatní. Příkladem může být možnost komunikace po různých médiích fyzické vrstvy modelu OSI - ethernet (optické vlákno, kroucená dvojlinka, Wi-Fi), sériová linka.


\subsubsection*{1. Vrstva síťového rozhraní}
Nejnižší vrstva umožňuje \textbf{přístup k fyzickému přenosovému médiu}. Je specifická pro každou síť v závislosti na její implementaci. Příklady sítí: Ethernet, Token ring, FDDI, 100BaseVG, X.25, SMDS.

\subsubsection*{2. Síťová vrstva}
Vrstva zajišťuje p\textbf{ředevším síťovou adresaci}, \textbf{směrování} a předávání datagramů (\textbf{packety}). Protokoly: \textbf{IP}, \textbf{ARP}, \textbf{RARP}, \textbf{ICMP}, \textbf{IGMP}, \textbf{IGRP}, \textbf{IPSEC}. Je implementována ve všech prvcích sítě - směrovačích i koncových zařízeních.

\textbf{Protokol IP (Internet Protocol)}. Od nadřazených protokolů transportní vrstvy obdrží datové segmenty s požadavkem na odeslání. K segmentům připojí vlastní hlavičku a vytvoří IP datagram. V IP hlavičce je především IP adresa příjemce a odesílatele. IP protokol je \textbf{nespojový} (před zahájením výměny dat nevytváří relaci) a \textbf{nespolehlivý} (předání paketů na místo určení není kontrolováno). Paket IP se tedy může ztratit, být doručen mimo pořadí, zdvojen nebo zpožděn. Protokol IP neobsahuje prostředky pro zotavení z chyb tohoto typu. To vše má zajistit nadřízená transportní vrstva – protokol \textbf{TCP}.

\subsubsection*{3. Transportní vrstva}
Transportní vrstva je implementována až v \textbf{koncových zařízeních} (počítačích) a umožňuje proto přizpůsobit chování sítě potřebám aplikace. Poskytuje transportní služby kontrolovaným spojením spolehlivým protokolem \textbf{TCP} (transmission control protocol) nebo nekontrolovaným spojením nespolehlivým protokolem \textbf{UDP} (user datagram protocol).
\\\\
\textbf{Protokol TCP (Transmission Control Protocol)} vytváří \textbf{virtuální okruh} mezi koncovými aplikacemi, zajišťuje tedy spolehlivý přenos dat.
\begin{itemize}
	\item Spolehlivá transportní služba, doručí adresátovi všechna data \textbf{bez ztráty} a ve \textbf{správném pořadí}.
	\item Služba se spojením, má fáze navázání spojení, přenos dat a ukončení spojení.
	\item Transparentní přenos libovolných dat.
	\item \textbf{Plně duplexní spojení}, současný obousměrný přenos dat.
	\item Rozlišování aplikací pomocí portů.
	\item 3-way handshake
	\item Komunikace je řízená pomocí příznakových bitů (ACK, SYN, atd.)
\end{itemize}
\noindent\makebox[\textwidth]{\includegraphics[width=14cm]{assets/4_tcp}}

\textbf{Protokol UDP (User Datagram Protocol)} poskytuje \textbf{nespolehlivou} transportní službu pro takové aplikace, které nepotřebují spolehlivost, jakou má protokol TCP. Nemá fázi navazování a ukončení spojení a už první segment UDP obsahuje aplikační data. UDP je používán aplikacemi jako je \textbf{DHCP}, \textbf{TFTP}, \textbf{SNMP}, \textbf{DNS} a \textbf{BOOTP}.
\begin{itemize}
	\item Nespolehlivá transportní služba, neověřuje zda data došla v pořádku nebo ve správném pořadí.
	\item Nižší režie než u TCP (rychlejší).
	\item Zajištění spolehlivosti je na aplikacích vyšší vrstvy.
	\item Nemá fázi navázání a ukončení spojení, rovnou zasílá data.
	\item Hlavička UDP má pouze 4 části (délku, zdrojový/cílový port, checksum)
\end{itemize}

\noindent\makebox[\textwidth]{\includegraphics[width=14cm]{assets/4_udp}}

\subsubsection*{4. Aplikační vrstva}
Jedná se přímo o programy (procesy), které využívají přenosu dat po síti ke konkrétním službám pro uživatele. Příklady: \textbf{Telnet} (TCP 23), \textbf{FTP} (TCP 20, 21), \textbf{HTTP} (TCP 80), \textbf{DHCP} (UDP 67, 68), \textbf{DNS} (TCP/UDP 53), \textbf{SSH} (TCP 22).
\\\\
Aplikační protokoly používají vždy jednu ze dvou základních služeb transportní vrstvy: \textbf{TCP} nebo \textbf{UDP}, případně obě dvě (např. DNS). Pro rozlišení aplikačních protokolů se používají tzv. \textbf{porty}, což jsou domluvená číselná označení aplikací. Každé síťové spojení aplikace je jednoznačně určeno \textbf{číslem portu} a \textbf{transportním protokolem} (a samozřejmě adresou počítače).

\subsection*{Překlad síťových adres NAT}
\textbf{NAT (Network address translation)} se většinou používá pro přístup více počítačů z \textbf{lokální sítě} na \textbf{Internet} pod \textbf{jedinou veřejnou adresou}. Překládá zdrojovou a cílovou IP adresu, je realizován na \textbf{routerech}, \textbf{firewallech} většinou zařízeních 3 vrstvy. Umožňuje připojit více počítačů na jednu veřejnou IP adresu - řeší se tak nedostatek přidělených veřejných IP adres. Využívá se překladové tabulky. 
\\\\
\textbf{Výhody:} 
\begin{itemize}
	\item Zvyšuje bezpečnost počítačů připojených za NATem (potenciální útočník nezná opravdovou IP adresu). 
	\item Umožňuje připojit více počítačů na jednu veřejnou IP adresu, čímž se obchází nedostatek IPv4 adres
\end{itemize}
\textbf{Nevýhody:}
\begin{itemize}
	\item Zařízení za NATem nemají skutečné připojení k Internetu (není možné se snadno připojit na zařízení za NATem.)
	\item NAT znemožňuje správnou funkcionalitu některých software
\end{itemize}

\subsubsection*{Typy NATu}
Typicky se využívá kombinace obou níže zmíněných řešení.
\begin{itemize}
	\item \textbf{Statický NAT} - Překladová tabulka je konfigurována manuálně administrátorem.
	\item \textbf{Dynamický NAT} - Obsah překladové tabulky je vytvářen dynamicky v závislosti na síťovém provozu. Veřejné adresy sou alokovány jednotlivým spojením jejich vypůjčením z \textbf{NAT Poolu}.
	\item \textbf{Network adress and por translation NAPT} - Několik uzlů využívá pouze jednu veřejnou IP adresu. Jednotlivé uzly jsou identifikovány pomocí různých čísel portů.
\end{itemize}

\subsection*{IPv6}
IPv6 \textbf{nahrazuje} dosluhující protokol IPv4. Přináší zejména \textbf{masivní rozšíření adresního prostoru} a zdokonalení schopnosti přenášet \textbf{vysokorychlostně data}. Starší protokol IPv4 poskytuje omezený adresní prostor – teoreticky $2^32$ adres. IPv6. Obsahuje celkem $2^128$ adres. Většina přenosových a aplikačních vrstev protokolů vyžaduje malé nebo žádné změny pro funkčnost s IPv6. Výjimkami jsou protokoly aplikací zahrnující adresy síťové vrstvy např.: FTP. \textbf{Multicast} je součástí základní specifikace IPv6 na rozdíl od IPv4, kde byl zaveden později. IPv6 nepoužívá \textbf{broadcast} na místní linku. Každá adresa má \textbf{128b}. \textbf{Odstraněna potřeba NAT}. Protokol pro IP vrstvu šifrování a autentizaci \textbf{IPsec} je integrální součástí souboru protokolů IPv6, na rozdíl od IPv4, kde je přítomen volitelně (obvykle ale implementován).
\\\\
IPv6 adresy se obvykle zapisují jako osm skupin čtyř hexadecimálních číslic: \textbf{2001:0db8::1428:57ab}. Vzhledem k zdlouhavému zápisu se může nejdelší sekvence nul nahradit \textbf{::}. 4 nuly můžeme nahradit jednou.

\subsubsection*{IPv6 Packet}
Paket IPv6 se skládá ze dvou hlavních částí: hlavičky a těla.
\begin{itemize}
	\item \textbf{Verzi} - 4 bity, verze 6
	\item \textbf{Dopravní třídu} - 8 bitů na prioritu paketu. Úroveň priority se dělí na rozsahy: kde zdroj podporuje kontrolu přetížení a bez podpory kontroly přetížení.
	\item \textbf{Pojmenování toku} - 20 bitů pro správu QoS. Původně určeno pro speciální obsluhu aplikací reálného času, nyní se nepoužívá.
	\item \textbf{Délka těla} - 16 bitů pro délku těla paketu. Při vynulování se nastaví „jumbo“ tělo (skok za skokem)
	\item \textbf{Následující hlavička} - 8 bitů, určuje další vnořený protokol. Hodnoty se shodují s hodnotami definovanými pro IPv4.
	\item \textbf{Zdrojová a cílová adresa} - 128 bitů na každou adresu.
	\item\textbf{Hop limit }- 8 bitů, číselně definuje počet povolených přechodů síťovými prvky. Každý přechod znamená snížení čísla o 1.
\end{itemize}

\noindent\makebox[\textwidth]{\includegraphics[width=16cm]{assets/4_ipv6}}


\pagebreak
\section{Jazyk predikátové logiky prvního řádu. Práce s kvantifikátory a ekvivalentní transformace formulí.}
%Standardní zobrazovací řetězec a realizace jeho jednotlivých kroků. Gouraudovo a Phongovo stínování. Řešení viditelnosti. Grafický standard OpenGL: stručná charakteristika.
\subsection{Standardní zobrazovací řetěz}
\begin{itemize}
	\item Klade důraz na rychlost nikoli na kvalitu.
	\item Realizuje ho OpenGL.
\end{itemize}
\begin{figure}[H]
\centering
\includegraphics[width=0.8\textwidth]{assets/5_retezec}
\end{figure}

\begin{itemize}
	\item  \textbf{Pokrytí povrchu objektů sítí rovinných plošek:}
	\begin{itemize}
		\item Ploškami bývají nejčastěji trojúhelníky nebo čtyřúhelníky.
		\item Pro objekty ve tvaru mnohostěnu je takové dělní vcelku samozřejmé.
		\item K přesnějšímu výpočtu barev bývá, ale někdy dělení na plošky jemnější.
		\item Někdy síť rovinných plošek žádaný povrch pouze aproximuje.
	\end{itemize}
	\item \textbf{Výpočet osvětlení ve vrcholech sítě}
	\item k tomu známe:
	\begin{itemize}
		\item Polohu, intenzitu a barvu světelných zdrojů.
		\item Souřadnice vrcholů ($P$), normál ($n$) a konstanty popisující optické vlastnosti materiálu ($O_a, O_d, O_s$)
		\begin{figure}[H]
		\centering
		\includegraphics[width=0.5\textwidth]{assets/5_vypocet_barevneho_vjemu}
		\end{figure}
	\end{itemize}
	\item \textbf{Aplikace zobrazovací transformace na vrcholy}
	\begin{itemize}
		\item Oblíbenou technikou je středové promítání. To je zadáno:
		\begin{itemize}
			\item Polohou průmětny.
			\item Polohou středu promítání.
		\end{itemize}
		\begin{figure}[H]
		\centering
		\includegraphics[width=0.5\textwidth]{assets/5_stred_promitani}
		\end{figure}
	\end{itemize}
	\item \textbf{Ořezání zorným objemem}
	\begin{itemize}
		\item Objekty nebo jejich části, nacházející se mimo zorný objem (obvykle jehlan) jsou odstraněny.
	\end{itemize}
		\begin{figure}[H]
		\centering
		\includegraphics[width=0.5\textwidth]{assets/5_orezani_zornym_pole}
		\end{figure}
	%\begin{itemize}
	\item \textbf{Rasterizace plošek}
	\begin{itemize}
		\item Postupně zpracovávány všechny plošky.
		\item Pro každou plošku rozsvěceny všechny její pixely.
		\item Barva každého pixelu se stanoví interpolací mezi hodnotami ve vrcholech.
		\begin{figure}[H]
		\centering
		\includegraphics[width=0.5\textwidth]{assets/5_rasterizace_plosek}
		\end{figure}
	\end{itemize}
	\item \textbf{Řešení viditelnosti (z--buffer)}
	\begin{itemize}
		\item Pro rozhodnutí viditelnosti se použijí hodnoty souřadnice $z$ (zde je $z_1 > z_2$).
		\item Před řešením viditelnosti bývá centrálním promítání převedeno na rovnoběžné.
		\begin{figure}[H]
		\centering
		\includegraphics[width=0.5\textwidth]{assets/5_pip_zbuffer}
		\end{figure}
	\end{itemize}
	\item \textbf{Nanášení textury}
	\begin{itemize}
		\item Vzhled obrázků lze vylepšit nánášením textury.
	\end{itemize}
\end{itemize}

\pagebreak
\section{Pojem relace, operace s relacemi, vlastnosti relací. Typy binárních relací. Relace ekvivalence a relace uspořádání.}
\subsection{Relace}
\begin{itemize}
\item \textbf{N-ární relace} nad množinami $A_1, \ldots, A_n$ je \textbf{libovolná podmnožina kartézského} součinu $A_1 \times \ldots \times A_n$ (tyto množiny jsou \textbf{nosičemi} relace).
\item \textbf{Kartézský součin} množin$A$ a $ B $, označovaný $ A \times B $, je množina všech uspořádaných dvojic, kde první prvek z dvojice patří do množiny $ A $ a druhý do množiny$  B $. Příklad: $\{a, b\} \times \{a, b, c\} = \{(a, a), (a, b), (a, c), (b, a), (b, b), (b, c)\}$.
\end{itemize}
\subsubsection{Typy relací}
\begin{itemize}
\item \textbf{homogenní} -- jediný druh nosiče ($A \times A$),
\item \textbf{heterogenní} -- alespoň dva různé druhy nosiče ($A \times B$),
\item \textbf{unární} (n = 1), \textbf{binární} (n = 2), \textbf{ternární} (n = 3), \textbf{n-ární} -- podle arity,
\item \textbf{triviální} -- \textbf{úplná} ($\rho = A_1 \times A_n$), \textbf{prázdná} ($\rho = \emptyset$),
\item \textbf{netriviální} -- $\emptyset \subset \rho \subset A_1 \times \ldots \times A_n$.
\end{itemize}

\subsubsection{Vlastnosti relací}
Binární relace $R \subseteq A \times A$ je:

\begin{itemize}
\item \textbf{Reflexivní} -- $\forall x \in{} A: (x, x) \in{} R $.
\item \textbf{Ireflexivní} -- $\forall x \in{} A: (x, x) \notin{}R $.
\item \textbf{Symetrická} -- $\forall x \in{} A: (x, y) \in{}R \Rightarrow (y, x) \in{}R $.
\item \textbf{Asymetrická} -- $\forall x \in{} A: (x, y) \in{}R \Rightarrow (y, x) \notin{}R $.
\item \textbf{Antisymetrická} -- $\forall x \in{} A: (x, y) \in{}R \land (y, x) \in{}R \Rightarrow x = y$.
\item \textbf{Tranzitivní} -- $\forall x \in{} A: (x, y) \in{}R \land (y, z) \in{}R \Rightarrow (x,z) \in R$.
\end{itemize}

\subsubsection*{Příklad}
\begin{itemize}
\item Relace ,,$=$'' na $ \mathbb{N} $ je reflexivní, symetrická, antisymetrická a tranzitivní, ale není ireflexivní ani asymetrická.
\item Relace ,,$\leq$'' na $ \mathbb{N} $ je reflexivní, antisymetrická a tranzitivní, ale není ireflexivní, symetrická ani asymetrická.
\item Relace ,,$<$'' na $ \mathbb{N} $ je ireflexivní, asymetrická, antisymetrická a tranzitivní, ale není reflexivní ani symetrická.
\end{itemize}

\subsection{Operace s relacemi}
\begin{itemize}
\item \textbf{Průnik} -- Prvek $ x $ náleží do průniku relací $ R1 \cap R2 $, pokud patří do množiny $ R1 (x \in R1) $ \textbf{a zároveň} do $ R2 (x \in R1)$.
\item \textbf{Sjednocení} -- Prvek $ x $ náleží do sjednocení relací $ R1 \cup R2 $, pokud patří do množiny $ R1 (x \in R1) $ \textbf{nebo} $ R2 (x \in R1)$.
\item \textbf{Doplněk} -- Doplňkem $ R1' $ k relaci $ R1 $ rozumíme \textbf{všechny prvky které nepatří} do $ R1 $.
\item \textbf{Inverze} -- Relace $ R^{-1} \subseteq B \times A $ je inverzní k relaci $ R \subseteq A \times B $, pokud $ xR^{-1}y \Leftrightarrow yRx $.
\item \textbf{Skládání relací} -- Výsledkem je množina dvojic, kde pokud existují dvojice $(a, b)\in R$ a $(b, c)\in S$, pak jejich složení $(a, c) \in R \circ S$.
\end{itemize}

\begin{figure}[H]
	\centering
	\includegraphics[width=.6\textwidth]{assets/relace_skladani}
\end{figure}

\subsection{Typy binárních relací}
Mezi nejznámější typy binárních relací patří \textbf{ekvivalenece} ($=$) [Re, Sy, Tr], \textbf{uspořádání} ($ <, >, \leq, \geq $) [Re, An, Tr] a \textbf{tolerance} [Re, Sy].

\subsubsection{Ekvivalence [Re, Sy, Tr]}
Relace ekvivalence představuje jakési zjemnění relace rovnosti. Vždy můžeme rozhodnout, že jsou dva prvky množiny stejné, tj. že a = a. Ale někdy se nám hodí zjistit, zda jsou si dva prvky \textbf{pouze podobné}, ne nutně stejné. Neboli zda mají stejnou nějakou zásadní vlastnost. Například dvě knihy můžeme považovat za podobné, pokud mají stejný žánr nebo \textbf{pomocí ekvivalence}: dvě knihy jsou ekvivalentní pokud mají stejný žánr.

\begin{itemize}
\item Binární relace na množině $ X $ je ekvivalentní, pokud je $ R $ na $A$: \textbf{reflexivní}, \textbf{symetrická} a \textbf{tranzitivní}.
\item \textbf{Třída ekvivalence} prvku $ a $ je množina všech prvků ekvivalentních s daným prvkem $ a $.
\item \textbf{Průnik} dvou ekvivalencí je zase ekvivalence.
\item \textbf{Sjednocení} dvou ekvivalencí nemusí znamenat, že výsledek bude ekvivalentní.
\end{itemize}

\subsubsection{Uspořádání [Re, An, Tr]}
\begin{itemize}
\item Binární relace na množině $ X $ je \textbf{neostrým uspořádání}, pokud je $ R $ na $A$: \textbf{reflexivní}, \textbf{antisymetrická} a \textbf{tranzitivní}.
\item Binární relace na množině $ X $ je \textbf{ostrým uspořádání}, pokud je $ R $ na $A$: \textbf{ireflexivní}, \textbf{antisymetrická} a \textbf{tranzitivní}.
\item Uspořádání je \textbf{úplné} pokud neexistují neporovnatelné prvky.
\end{itemize}

\pagebreak
\section{Pojem operace a obecný pojem algebra. Algebry s jednou a dvěma binárními operacemi.}
\begin{itemize}
\item \textbf{Utajení} (confidentality) – posluchač na kanále datům nerozumí
\item \textbf{Autentizace} (authentication) – jistota, že odesílatel je tím, za koho se vydává 
\item \textbf{Integrita} (integrity) – jistota, že data nebyla na cestě zmodifikována 
\item \textbf{Nepopiratelnost} (non-repudiation) – zdroj dat nemůže popřít jejich odeslání
\end{itemize}

\subsection*{Útoky}
\begin{itemize}
	\item \textbf{ARP dotazy} - falšování ARP odpovědí (falešný překlad IP-to-MAC). \textbf{Oprava} užitím statických ARP záznamů.
	\item \textbf{Routing protocol} - falšování routovacích informací propagovaných routovacím protokolem (RIP atd). Oprava 	\textbf{filtrováním} zdrojů routovacích informací.
	\item \textbf{Switchované sítě} - proti přetížení přepínací tabulky. Oprava užitím limitování počtu MAC na portu, statickým 	listem povolených MAC.
	\item \textbf{Brute Force} - zadávání hesel pomocí hrubé síly.
	\item \textbf{Denialenial of service (DoS)} - cílem útočníka vyčerpání systémových prostředků (paměť, CPU, šířka pásma) síťového prvku nebo serveru a jeho zhroucení nebo změna požadovaného chování.
\begin{itemize}
	\item \textbf{Smurf} – zahlcení cíle ICMP pakety (ping), jejich zpracování mívá někdy přednost před běžným provozem; útočník pošle žádost o ping všem (broadcast) a jako odesilatele uvede cíl útoku. \textbf{Řešení:} packetové filtry.
	\item \textbf{SYN flood} – neustálé navazování TCP spojení (příznak SYN), server alokuje prostředky a pošle (SYN-ACK) a čeká na odpověď, které se nedočká. \textbf{Řešení:} zkrácení doby čekání na odpověď.
\end{itemize}
	\item \textbf{Distributed DoS (DDoS)} – DoS útok je vedený z mnoha stanic, které byli již před tím napadeny a nyní jsou využity k tomuto útoku. Je obtížně blokovatelný kvůli přístupu mnoha stanic.
\end{itemize}

\subsection*{Filtrování provozu}
\subsubsection*{Paketové filtry (nestavové)}
\textbf{Filtrování probíhá dle informací v hlavičce 3 a 4 vrstvy}. Pravidla udávají, ze které adresy a portu na kterou adresu a port může být paket procházející rozhraním routeru propuštěn. Na routerech CISCO je realizován jako \textbf{Access Control List (ACL)} prostřednictvím sekvence záznamu, které povolují/zakazují přenos paketu, které odpovídají daným kritériím. Samotný paketový filtr je rychlý, nenáročný na systémové zdroje, ale úroveň kontroly je relativně malá.

\subsubsection*{Stavový firewall}
(též stavový paketový filtr, anglicky stateful firewall) odděluje důvěryhodnou (interní) síť od nedůvěryhodné (externí) sítě. Funguje stejně jako jednoduchý paketový filtr, ukládá \textbf{navíc ale i informace o povolených spojeních}, podle kterých pak může rozhodovat, zda procházející pakety patří do již povoleného spojení a mohou být propuštěny nebo zda musí znovu projít kontrolou. Firewall je \textbf{velmi rychlý}, poskytuje slušnou úroveň zabezpečení a snazší konfiguraci.  Poskytuje urychlené zpracování paketu již povoleného spojení. Obdobou stavového firewallu je nestavový firewall, který se rozhoduje pouze na základě informací obsažených v konkrétním paketu (pracuje na nižší síťové vrstvě ISO/OSI modelu) a aplikační firewall, který pracuje naopak na vyšší síťové vrstvě. 
\\\\
\noindent\makebox[\textwidth]{\includegraphics[width=8cm]{assets/7_filter}}

\subsubsection*{Virtuální privátní síť }
(zkratka VPN, anglicky virtual private network) je v informatice prostředek k \textbf{propojení několika počítačů prostřednictvím (veřejné) nedůvěryhodné počítačové sítě}. Lze tak snadno dosáhnout stavu, kdy spojené počítače budou mezi sebou moci komunikovat, jako kdyby byly propojeny v rámci jediné uzavřené privátní (a tedy důvěryhodné) sítě. Při navazování spojení je totožnost obou stran ověřována pomocí \textbf{digitálních certifikátů}, dojde k autentizaci, veškerá komunikace je šifrována, a proto můžeme takové propojení považovat za bezpečné.

\subsection*{Šifrování}
\subsubsection*{Symetrické šifrování}
Pro šifrování i dešifrování se používá \textbf{pouze jeden klíč}, který musí mít všichni účastníci, kteří chtějí data šifrovat nebo dešifrovat. \textbf{Nebezpečí hrozí při distribuci }tohoto klíče, který pokud bude prozrazen tak všichni účastníci musí začít používat nový klíč.  Šifrování je \textbf{rychlé} a \textbf{jednoduché} pro implementaci (možná implementace přímo v HW). Nejznámější \textbf{DES}, \textbf{3DES}, \textbf{IDEA}. \textbf{DES} již dnes není bezpečný, používá se \textbf{3DES}, který \textbf{DES} zašifruje 3x po sobě.
\\\\
\noindent\makebox[\textwidth]{\includegraphics[width=10cm]{assets/7_symetric}}

\subsubsection*{Asymetrické šifrování}
Podstatou je \textbf{generace dvou šifrovacích klíčů}, které spolu spolupracují. Tyto klíče se vzájemně matematicky doplňují a je možné oba použít jak pro dešifrování tak šifrování. Veřejný klíč je volně šiřitelný pro všechny, kteří chtějí šifrovat posílaná data. Soukromý klíč je tajný a je pouze pro potřebu pro dešifrování toho, co bylo zašifrováno veřejným klíčem. Uživatel tedy pro šifrovanou komunikaci potřebuje oba klíče. Výhodou je, že\textbf{ není třeba veřejný klíč speciálně ukrývat} (je vyřešena metoda distribuce), pouze je třeba zajistit mechanizmus proti modifikaci veřejných klíčů při přenosu - \textbf{certifikovaná autorita (CA)} klíč digitálně podepisuje. Nevýhodou je, že v porovnání se symetrickým je \textbf{pomalejší} z důvodu použitých matematických funkcí. Asymetrické systémy jsou např. \textbf{RSA} nebo \textbf{DSA}. Stupeň \textbf{bezpečnosti se odvíjí od délky} použitého klíče.
\\\\
\noindent\makebox[\textwidth]{\includegraphics[width=10cm]{assets/7_asymetric}}


\subsubsection*{Certifikační autorita}
\begin{itemize}
	\item Entita, které je důvěřováno.
	\item Registruje (podepsané) veřejné klíče.
	\item První kontakt s certifikační autoritou \textbf{musí proběhnout osobně }(získání dvojice podepsaný veřejný-privátní klíč).
	\item Veřejný klíč certifikační autority musí být \textbf{důvěryhodným} způsobem zaveden do každého systému.
\end{itemize}

\subsection*{Zabezpečení na jednotlivých vrstvách OSI-RM }
\begin{itemize}
	\item \textbf{L7} – S/MIME 
	\item \textbf{L4} – SSL (jen TCP) 
	\item \textbf{L3} – IPSec (nezávislé na médiu i aplikaci)
	\item \textbf{L2} – hop-by-hop, neefektivní
\end{itemize}




\subsection*{Autentizace}
Je proces ověření proklamované identity subjektu. Probíhá nejčastěji jednou ze tří metod:
\begin{itemize}
	\item Řekneme něco, co známe (\textbf{heslo}, PIN)
	\item Ukážeme něco, co máme (ID karta, \textbf{hardwarový klíč})
	\item Necháme systémem změřit něco našeho (\textbf{biometrické údaje}, otisk prstů, sítnice)
\end{itemize}
Existuje \textbf{několika stupňové ověření}, například kombinací PIN a biometrických údajů. Po ověření identity následuje autorizace což je souhlas k provedení operace či umožnění přístupu. \textbf{Často se používá v souladu s šifrováním:}
\begin{itemize}
	\item U \textbf{symetrického} odesílatel šifruje uživatelské jméno sdíleným klíčem a příjemce kontroluje platnost tohoto uživatelského jména.
	\item U \textbf{asymetrického} se užívá digitálních podpisů Certifikační autority.
\end{itemize}



\pagebreak
\section{FCA – formální kontext, formální koncept, konceptuální svazy. Asociační pravidla, hledání často se opakujících množin položek.}
UML je jazyk umožňující \textbf{specifikaci} (struktura a model), \textbf{vizualizaci} (grafy), \textbf{konstrukci} (vygenerování kódu, např class diagram) a \textbf{dokumentaci artefaktů} softwarového systému. V průběhu let se UML stal \textbf{standardizovaným jazykem }určeným pro vytvoření výkresové dokumentace (softwarového) systému v \textbf{různých fázích vývoje}. K vytváření jednotlivých modelů systému jazyk UML poskytuje celou řadu diagramů umožňujících \textbf{postihnout různé aspekty systému}. Jedná se celkem o čtyři základní náhledy a k nim přiřazené diagramy: 

\begin{enumerate}
\item \textbf{Funkční náhled}
\begin{enumerate}
\item \textbf{Diagram případů užití} -- popisuje vztahy mezi aktéry a jednotlivými případy použití. Poskytuje funčkní náhled na systém (kdo se systémem pracuje a jak). Uplatňuje se pro realizaci \textbf{DFD} (Data flow diagram) ve fázi \textbf{specifikace požadavků} (VIS).
\end{enumerate}
\item \textbf{Logický náhled}
\begin{enumerate}
\item \textbf{Diagram tříd} -- specifikuje množinu tříd, rozhraní a jejich vzájemné vztahy. Tyto diagramy slouží k vyjádření statického pohledu na systém.
\begin{figure}[H]
	\centering
	\includegraphics[width=.9\textwidth]{assets/class.png}
\end{figure}
\item \textbf{Objektový diagram} -- zobrazuje \textbf{instance tříd} (objekty), někdy nazýván jako instanční diagram.
Je \textbf{snímkem objektů} a jejich vztahů v systému \textbf{v určitém časovém okamžiku}, který chceme z nějakého důvodu zdůraznit. Využívá se v datové analýze pro ERD.
\begin{figure}[H]
	\centering
	\includegraphics[width=.5\textwidth]{assets/obj_diag.jpg}
\end{figure}
\item \textbf{Stavový diagram} -- dokumentuje \textbf{životní cyklus} objektu dané třídy z hlediska jeho \textbf{stavů}, \textbf{přechodů} mezi těmito stavy a \textbf{událostmi}, které tyto přechody uskutečňují. 
\begin{figure}[H]
	\centering
	\includegraphics[width=.5\textwidth]{assets/stav_diag.png}
\end{figure}
\end{enumerate}
\item \textbf{Dynamický náhled popisující chování}
\begin{enumerate}
\item \textbf{Diagram aktivit} --  popisuje podnikový proces pomocí jeho stavů reprezentovaných vykonáváním aktivit a pomocí přechodů mezi těmito stavy způsobených ukončením těchto aktivit. Účelem diagramu aktivit je blíže popsat tok činností daný vnitřním mechanismem jejich provádění. 
\item \textbf{Sekvenční diagramy} -- popisuje interkace mezi objekty z hlediska jejich \textbf{časového uspořádaní}.
\item \textbf{Diagramy spolupráce} -- je obdobně jako předchozí sekvenční diagram zaměřen na interkace, ale z pohledu strukturální organizace objektů. Jinými slovy není primárním aspektem časová posloupnost posílaných zpráv, ale \textbf{topologie rozmístění objektů}.

Časová posloupnost zaslání zpráv je vyjádřena jejich \textbf{pořadovým číslem}. Návratová hodnota je vyjádřena \textbf{operátorem přířazení} \textbf{$:=$}. Opakované zaslání zprávy je dáno symbolem \textbf{*} a v hranatých závorkách uvedením podmínky opakování cyklu. Navíc tento diagram zavádí i následující \textbf{typy viditelnosti} vzájemně spojených objektů: 
\begin{enumerate}
\item \texttt{<<local>>} -- vyjadřuje situaci, kdy objekt je vytvořen v těle operace a po jejím vykonání je zrušen, 
\item \texttt{<<global>>} -- specifikuje globálně viditelný objekt,
\item \texttt{<<parameter>>} -- vyjadřuje fakt, že objekt je předán druhému jako argument na něj zaslané zprávy,
\item \texttt{<<association>>} -- specifikuje trvalou vazbu mezi objekty (někdy se také hovoří o tzv. známostním spojení).
\end{enumerate}

\begin{figure}[H]
	\centering
	\includegraphics[width=.8\textwidth]{assets/diag_spoluprace.png}
\end{figure}
\end{enumerate}
\item \textbf{Implementační náhled}
\begin{enumerate}
\item \textbf{Diagram komponent} -- ilustruje organizaci a \textbf{závislosti mezi softwarovými komponentami}. Zdrojové komponenty tvoří soubory vytvořené použitým programovacím jazykem. Diagram spustitelných komponent specifikuje všechny komponenty vytvořené námi i ty, které nám dává k dispozici implementační prostředí.
\item \textbf{Diagram nasazení} -- upřesňuje nejen ve smyslu konfigurace technických prostředků, ale především z hlediska rozmístění implementovaných softwarových komponent na těchto prostředcích.
\end{enumerate}
\end{enumerate}


\subsection{Diagramy a jejich použití v rámci fází vývoje}
\begin{itemize}
\item \textbf{Specifikace požadavků}: Diagram případů užití, Sekvenční diagramy, Diagram aktivit.
\item \textbf{Návrh}: Diagram tříd, Objektový diagram, Stavový diagram, Sekvenční diagramy.
\item \textbf{Implementace}: Diagramy spolupráce, Diagram komponent, Diagram nasazení.
\end{itemize}


\pagebreak
\section{Metrické a topologické prostory – metriky a podobnosti.}
\subsection{Metrický prostor}

\subsection{Topologický prostor}

\pagebreak
\section{Shlukování.}

\pagebreak
\section{Náhodná veličina. Základní typy náhodných veličin. Funkce určující rozdělení náhodných veličin.}
\textbf{Náhodný pokus} -- Děj, jehož výsledek není předem jednoznačně určen podmínkami, za nichž probíhá.\\
\textbf{Náhodný jev} -- Tvrzení o výsledku náhodného pokusu, přičemž o pravdivosti tohoto tvrzení lze po ukončení pokusu rozhodnout.
Náhodná veličina \uv{NV} -- \textbf{číselné} vyjádření výsledku náhodného pokusu.\\
\begin{itemize}
	\item NV je funkce, která každému elementárnímu jevu $\omega \in \Omega$ přiřadí reálné číslo.
	\item NV obvykle značíme velkými písmeny.
	\item NV přiřadí každému elementárnímu jevu $\omega$ reálné číslo (převádí elementární jevy (abstraktní objekty) na čísla).
	\item Hodnota $X(\omega)$ NV X závisí na tom, který elementární jev $\omega$ nastal.
	\item Víme--li, který elementární jev $\omega$ nastal, známe hodnotu $X(\omega)$ NV X.
	\item Příklady:
	\begin{itemize}
		\item[]	\begin{center}
			$X$... číselný výsledek hodu kostkou\\
			\end{center}
			\textbf{Náhodný pokus}: Hod kostkou. \\
			\textbf{Náhodný jev}: Padne sudé číslo. ($X \in \{2;4;6\}$) \\\\
			
		\item[]	\begin{center}
			$X$... rychlost připojení k internetu (Mb/s) \\
			\end{center}
			\textbf{Náhodný pokus}: Měření rychlosti připojení k internetu (download). \\
			\textbf{Náhodný jev}: Rychlost připojení k internetu je vyšší než 20 Mb/s.  ($X > 20$)\\\\
			
		\item[]	\begin{center}
			$X$... počet dívek mezi 1 000 náhodně vybranými dětmi \\
			\end{center}
			\textbf{Náhodný pokus}: Náhodný výběr 1 000 dětí a zjištění počtu dívek mezi nimi. \\
			\textbf{Náhodný jev}: Mezi 1 000 náhodně vybranými dětmi bude více než 500 dívek. ($X > 500$)
	\end{itemize}

\end{itemize}





\subsection{Distribuční funkce F(x)}
\begin{itemize}
	\item Distribuční funkce F(t) udává pravděpodobnost, že náhodná veličina X bude \textbf{menší než} dané reálné číslo $t$. 
	$$F(t) = P(X <t)$$
	\item Distribuční funkce jednoznačně určuje rozdělení NV, tj. známe--li distribuční funkci, umíme určit pravděpodobnost $P(X \in M)$ pro libovolnou $M\subset \mathbb{R}$.
	\begin{figure}[H]
	\centering
	\includegraphics[width=0.75\textwidth]{assets/11_dist_fce}
	\end{figure}
\end{itemize}

\subsection{Základní typy náhodných veličin}
\begin{itemize}
	\item \textbf{Diskrétní NV} 
	\item \textbf{Spojité NV} 
\end{itemize}

\subsubsection{Diskrétní náhodná veličina (\uv{DNV})}
\begin{itemize}
		\item mohou nabývat spočetně mnoha hodnot (počet dní hospitalizace, počet dní nemocenské, počet zákazníků v lékarně během jednoho dne..) 
		\item DNV X s distribuční funkcí $F_x(t)$ je charakterizována \textbf{Pravděpodobnostní funkcí} $P(X = x_i)$, tj. funkcí pro níž platí:
				\begin{equation*}
					F_x(t) = \sum_{x_i<t}P(X = x_i)= \sum_{x_i<t}P(x_i).
				\end{equation*}

		\item \textbf{Pravděpodobnostní funkce} 
			\begin{itemize}
				\item $P(x_i) \geq 0$
				\item $\sum_i P(X = x_i) = 1$
				\item Lze zadat předpisem, tabulkou, grafem.
				\begin{figure}[H]
					\centering
					\includegraphics[width=0.6\textwidth]{assets/11_dnv_prav}
				\end{figure}
				\item[$\Rightarrow$] \textit{Příklad} -- Při ověřování kvality výroby jsou náhodně vybrány dva výrobky a je testována jejich kvalita. Počet vadných výrobků mezi vybranými modelujeme náhodnou veličino X. Z dlouhodobého pozorování jsou známy údaje uvedené v následující tabulce.
				\begin{table}[H]
					\centering
					\begin{tabular}{c|c}
						\textbf{Počet vadných výrobků}            & \textbf{Pravděpodobnost} \\\hhline
						0      &   0,25      \\ 
						1      &   0,50    \\
						2	   &   0,25
					\end{tabular}
				\end{table}
				\textit{Určete pravděpodobnostní a distribuční funkci počtu vadných výrobků v testovaném vzorku.}
				\begin{figure}[H]
					\centering
					\includegraphics[width=0.75\textwidth]{assets/11_vztah_prav_dist_dnv}
				\end{figure}
				\item Body nespojitosti distribuční f--ce jsou body, v nichž je pravděpodobnostní f--ce nenulová.
				\item $P (X= a) = \lim_{x\to a+} F(x) - F(a)$ , tj. velikost \uv{skoku} distribuční funkce v bodech nespojitosti je rovna příslušným hodnotám pravděpodobnostní funkce.
			\end{itemize}
		\item \textbf{Distribuční funkce}
				\begin{itemize}
					\item $F(t) = \sum_{x_i < t} P(x_i)$
					\item Lze zadat předpisem, tabulkou, grafem.
					\begin{figure}[H]
					\centering
					\includegraphics[width=0.6\textwidth]{assets/11_dnv_dist}
				\end{figure}
				\item Distribuční funkce $F(t)$ udává pravděpodobnost, že náhodná veličina $X$ bude \textbf{menší než} dané reálné číslo $t$. $$F(t) = P(X < t)$$
				\item Vlastnosti distribuční funkce:
				\begin{itemize}
					\item[$\circ$] $0 \leq F(t) \leq 1$,
					\item[$\circ$] je neklesající,
					\item[$\circ$] je zleva spojitá,
					\item[$\circ$] má nejvýše spočetně mnoho bodů nespojitosti,
					\item[$\circ$] $F(t) \rightarrow 0 $ pro $ t \rightarrow -\infty $ (\uv{začíná} v 0),
					\item[$\circ$] $F(t) \rightarrow 1 $ pro $ t \rightarrow \infty $ (\uv{končí} v 1).
				\end{itemize}
				\begin{figure}[H]
					\centering
					\includegraphics[width=0.75\textwidth]{assets/11_vztah_prav_dist_dnv2}
				\end{figure}
				$$P(X = a) \lim_{x \to a+} F(x) - F(a)$$
			\end{itemize}
	\end{itemize}

\subsubsection{Spojitá náhodná veličina (\uv{SNV})}

\begin{itemize}
		\item Náhodná veličina $X$ má spojité rozdělení pravděpodobnosti (zkráceně \uv{je spojitá}) právě když má spojitou distribuční funkci.
		\item Mohou nabývat všech hodnot na nějakém intervalu (mají spojitou distribuční funkci) (doba do remise onemocnění, výška, váha, BMO, IQ, vitální kapacita plic, chyba měření...).
		\item SNV X s distribuční funkcí $F_x(t)$ je charakterizována \textbf{hustotou pravděpodobností} $f(x)$, tj. funkcí pro níž platí:
		\begin{equation*}
				F_x(t) = \int_{-\infty}^{t} f(x) dx.
		\end{equation*}
		\item \textbf{Hustota pravděpodobnosti f(x)}
		\begin{itemize}
			\item $F(x) = \int_{-\infty}^{t} f(x) dx \Rightarrow f(x) = \frac{dF}{dx}$
			\item Vlastnosti hustoty pravděpodobnosti:
			\begin{itemize}
				\item[] $f(x)$ je reálná nezáporná funkce,
				\item[] $\int_{-\infty}{\infty} f(x) = 1$ (plocha pod křivkou hustoty je 1), 
				\item[] $\lim_{x \to -\infty} f(x) = 0$ (\uv{začíná v 0}),
				\item[] $\lim_{x \to \infty} f(x) = 0$ (\uv{končí v 0}).
				\begin{figure}[H]
					\centering
					\includegraphics[width=0.4\textwidth]{assets/11_snv_muze}
					\caption{$f(x)$ může nabývat hodnot vyšších než 1}
				\end{figure}
			\end{itemize}
		\end{itemize}
		\item \textbf{Distribuční funkce}
		$$F_x(t) = \int_{-\infty}^{t} f(x) dx$$
		\begin{figure}[H]
			\centering
			\includegraphics[width=0.5\textwidth]{assets/11_dist_fce_snv}
		\end{figure}
		\begin{figure}[H]
			\centering
			\includegraphics[width=0.6\textwidth]{assets/11_vztah_prav_dist_snv}
			\caption{Vztah mezi pravděpodobností a distribuční funkcí}
		\end{figure}
		$$P(X = a) = 0$$
	\end{itemize}

\subsection{Číselné charakteristiky NV}
\begin{itemize}
	\item Distribuční funkce (pravděpodobnostní funkce, hustota pravděpodobnosti) popisují rozdělení NV jednoznačně, do všech podrobností. 
	\item Někdy nás zajímá pouze některý aspekt NV, který se dá popsat jedním číslem:
	\begin{itemize}
		\item očekávaná hodnota NV,
		\item variabilita možných hodnot,
		\item hodnota, pod níž leží pouze malé množství hodnot NV,
		\item šikmost rozdělení,
		\item koncentrace hodnot NV kolem očekávané hodnoty (špičatost rozdělení).\\

		\item \textbf{Obecný moment r--tého řádu} (značí se $\mu_r$ nebo $E(X^r)$, pro r=1,2...) \\pro diskrétní NV: $\qquad \mu_r = \sum_{(i)}x_i^r \cdot P(x_i)$ \\pro spojitou NV: $\qquad \mu_r = \int_{-\infty}^{\infty} x^r \cdot f(x) dx$
		\item \textbf{Střední hodnota} (expected value, mean, značí se jako $\mathbf{E(X)}$ nebo $\mu$) \\pro diskrétní NV: $\qquad E(X) = \mu = \sum_{(i)}x_i \cdot P(x_i)$\\pro spojitou NV: $\qquad E(X) = \mu = \int_{-\infty}^{\infty} x \cdot f(x)dx$
		\item \textbf{Centrální moment r--tého řádu} $\mathbf{\mu_r}$' (značíme $\mu_r$' $= E[(X - EX)]^r$) \\pro diskrétní NV: $\qquad \mu_r' = \sum_{(i)}(x_i -EX)^r \cdot P(x_i)$\\pro spojitou NV: $\qquad \mu_r' = \int_{-\infty}^{\infty} (x - EX)^r \cdot f(x)dx$
		\item \textbf{Rozptyl} (dispersion, variance; značí se $\mu_2'$ nebo $DX$ nebo $\sigma^2$) \\pro diskrétní NV: $\qquad D(X) = \mu_2' = \sum_{(i)}(x_i -EX)^2 \cdot P(x_i)$\\pro spojitou NV: $\qquad D(X) = \mu_2' = \int_{-\infty}^{\infty} (x - EX)^2 \cdot f(x)dx$
	\end{itemize}
\end{itemize}

\subsubsection{Význam střední hodnoty}
Střední hodnotu $E(X)$ náhodné veličiny $X$ lze chápat jako:
\begin{itemize}
	\item průměrnou (očekávanou) hodnotu NV $X$, kolem níž hodnoty NV kolísají,
	\item míru polohy, populační průměr,
	\item vážený průměr všech možných hodnot ($E(X) = \sum_{(i)}x_i \cdot P(x_i)$),
	\item \uv{těžiště} možných hodnot.
	\begin{figure}[H]
			\centering
			\includegraphics[width=0.7\textwidth]{assets/11_stredni_hodnota}
	\end{figure}
\end{itemize}

\subsubsection{Kvantily}
\textbf{p--kvantil} $\mathbf{x_p}$ (také $100_p\%$--ní kvantil je číslo, pro které platí): $$P(X < x_p) = p.$$ 
\begin{center}
$\Rightarrow F(x_p) = p \Rightarrow x_p = F^{-1}(p)$ \\
(tj. kvantilová funkce $F^{-1}(p)$ je funkcí inverzní k distribuční funkci $F(x_p)$)
\begin{figure}[H]
			\centering
			\includegraphics[width=0.9\textwidth]{assets/11_kvantily}
	\end{figure}
\end{center}

\begin{itemize}
	\item Kvantily obvykle určujeme pouze pro SNV.
	\item Význačné kvantily
	\begin{itemize}
		\item \textbf{Kvartily}
		\begin{itemize}
			\item[] Dolní kvartil $x_{0,25}$
			\item[] Medián $x_{0,5}$
			\item[] Horní kvartil $x_{0,75}$
		\end{itemize}
		\item \textbf{Decily} -- $x_{0,1}; x_{0,2}; ... ;x_{0,9}$
		\item \textbf{Percentily} -- $x_{0,01}; x_{0,02}; ... ;x_{0,99}$
		\item \textbf{Minimum} $x_{min}$ a \textbf{Maximum} $x_{max}$
	\end{itemize}
\end{itemize}

\subsubsection{Modus}
\textbf{Modus} $\mathbf{\hat{x}}$ -- typická hodnota náhodné veličiny \\ 
\textbf{pro diskrétní NV}: $x_i: P(X = \hat{x}) \geq P(X = x_i)$ (tzn. modus je taková hodnota DNV, v níž $P(x_i)$ nabývá svého maxima) \\
\textbf{pro spojité NV}: $ x_i: f(\hat{x}) \geq f(x)$ (tzn. modus je taková hodnota SNV, v níž $f(x)$ nabývá svého maxima)
\begin{itemize}
	\item Modus není těmito podmínkami určen jednoznačně, tzn. náhodná veličina může mít několik modů (např. výsledek hodu kostkou). Má-li NV právě jeden modus, mluvíme o unimodálním rozdělení NV.
	\item Má-li NV unimodální symetrické rozdělení, pak $E(X) = x_{0,5} = \hat{x}$.
\end{itemize}

\subsubsection{Význam rozptylu}
\begin{itemize}
	\item Míra variability dat kolem střední hodnoty.
	\item Střední kvadratická odchylka od střední hodnoty ($D(X) = E(X - E(X))^2$).
	\item Malý rozptyl $\approx$ hodnoty NV se s vysokou pravděpodobností objevují blízko $E(X)$.
	\item Velký rozptyl $\approx$ hodnoty NV se často objevují ve velké vzdálenosti od $E(X)$.
	\item Jednotka rozptylu je kvadrátem jednotky náhodné veličiny.
	\begin{figure}[H]
			\centering
			\includegraphics[width=0.9\textwidth]{assets/11_rozptyl}
	\end{figure}
\end{itemize}

\subsubsection{Směrodatná odchylka}
\textbf{Směrodatná odychylka} $\mathbf{\sigma}$ $$\sigma(Y) = \sqrt{D(X)}$$
Směrodatná odchylka neumožňuje srovnávat variabilitu náhodných veličin měřených v různých jednotkách! 

\subsubsection{Variační koeficient}
\textbf{Variační koeficient} $\mathbf{\gamma}$ je definován pouze pro nezáporné náhodné veličiny. $$\gamma = \frac{\sigma}{\mu'}, resp. \frac{\sigma}{\mu'} \cdot 100 [\%]$$
\begin{itemize}
	\item Variační koeficient –- směrodatná odchylka v procentech střední hodnoty
	\item Čím nižší var. koeficient, tím homogennější soubor.
	\item $V_X > 50\%$ značí silně rozptýlený soubor.
\end{itemize}

\pagebreak
\section{Vybraná rozdělení diskrétní a spojité náhodné veličiny - binomické, hypergeometrické, negativně binomické, Poissonovo, exponenciální, Weibullovo, normální rozdělení.}
\newcommand{\pointing}{\includegraphics[width=1em]{assets/point.png}}
\begin{itemize}
	\item \textbf{Náhodná veličina} -- číselné vyjádření výsledku náhodného pokusu.
	\item Základní typy NV -- \textbf{diskrétní} NV, \textbf{spojitá} NV.
	\item \textbf{Rozdělení pravděpodobnosti} -- Předpis, který jednoznačně určuje všechny pravděpodobnosti typu $P(X \in M)$, kde $M \subset \mathbb{R}$
	
	(tj. $P (X = a), P (X < a), P(X > a), P(a < X < b), ...,$ kde $a, b \in \mathbb{R}$).
	\begin{itemize}
	\item Rozdělení pravděpodobnosti \textbf{DNV} -- distribuční funkcí $F(x) = P(X < x),$ resp. \textbf{pravěpodobnostní funkcí} $P(x_i)$.
	\item Rozdělení pravděpodobnosti \textbf{SNV} -- distribuční funkcí $F(x) = P(X < x),$ resp. \textbf{hustotou pravěpodobnostní} $f(x)$.
\end{itemize}
	\item \textbf{Číselné charakteristiky pro popis NV} -- střední hodnota $\mathbf{E(X)}$, rozptyl $\mathbf{D(X)}$, směrodatná odchylka $\mathbf{\sigma(X)}$, p--kvantily $\mathbf{x_p}$
\end{itemize}

\subsection{Diskrétní náhodná veličina}
\textit{Příklady}
\begin{itemize}
	\item[$\circ$] počet šroubů typu M10 mezi 10 vybranými (víme-li, že do dodávky 100 šroubů bylo omylem zařazeno 20 šroubů typu M50)
	\item[$\circ$] počet pacientů (z 10 očkovaných) u nichž byla použita prošlá očkovací látka (víme-li, že v balení bylo 20 dávek očkovací látky, přičemž 5 z nich bylo prošlých)

\begin{itemize}
	\item[$\rhd$] \textbf{Obecně:} -- počet úspěchů v $n$ (závislých) pokusech
	\end{itemize}
	\item[$\circ$] počet správně přenesených bitů předtím než dojde ke $4.$ chybě (víme-li, že pravděpodobnost chybného přenosu bitu je $0,12$)
	\item[$\circ$] počet dobrovolníků, které budeme muset testovat dříve než najdeme 5 dárců s krevní skupinou AB (předpokládejme, že dobrovolníci neznají svou krevní skupinu, pravděpodobnost výskytu krevní skupiny (populační frekvence) krevní skupiny AB je $0,05$)
\begin{itemize}
	\item[$\rhd$] \textbf{Obecně:} -- počet (nezávislých) pokusů do $k.$ úspěchu
	\end{itemize}
	\item[$\circ$] počet škrábanců na $1 m^2$ lakovaného povrchu (víme-li, že průměrně lze očekávat 3 škrábance na $10 m^2$)
	\item[$\circ$] počet červených krvinek v $10 ml$ krve ženy (víme-li, že u průměrně lze pozorovat $4,8 \cdot 10^{12}$ červených krvinek v $1l$ krve (u žen))
\begin{itemize}
	\item[$\rhd$] \textbf{Obecně:} -- počet událostí v časovém intervalu, na ploše, v objemu
	\end{itemize}
\end{itemize}
\subsubsection{Bernoulliho pokusy}
\begin{itemize}
	\item \textbf{Posloupnost nezávislých pokusů majících pouze dva možné výsledky} (takových pokusů, kdy úspěch v libovolné skupině pokusů neovlivňuje pravděpodobnost úspěchů v pokusu, který do této skupiny nepatří).
	\item \textbf{v nichž jev A} (úspěch) \textbf{nastává s pravděpodobností } $\mathbf{\pi}$ a neúspěch s pravděpodobností $1 - \pi$.
\end{itemize}

\textit{Příklad:} Předpokládejme, že pravděpodobnost narození dívky je $0,49$. Jaká je pravděpodobnost toho, že mezi čtyřmi dětmi v rodině je právě jedna dívka?
\begin{itemize}
\item X… počet dívek mezi 4 dětmi 
\item D ... narodí se dívka, $P(D) = \pi$ 
\item $\bar{D}$ ...  narodí se kluk, $P(D) = 1 - \pi$
\end{itemize}
\begin{equation*}
\begin{split}
&(X = 1) \,... \, \{D\bar{D}\bar{D}\bar{D},\bar{D}D\bar{D}\bar{D},\bar{D}\bar{D}D\bar{D},\bar{D}\bar{D}\bar{D}D\} \\
&P(D\bar{D}\bar{D}\bar{D}) = P(\bar{D}D\bar{D}\bar{D}) = P(\bar{D}\bar{D}D\bar{D}) = P(\bar{D}\bar{D}\bar{D}D) = \pi \cdot (1 - \pi)^3 \\
&P(X = 1) = P(D\bar{D}\bar{D}\bar{D} \cup \bar{D}D\bar{D}\bar{D} \cup \bar{D}\bar{D}D\bar{D} \cup \bar{D}\bar{D}\bar{D}D) = \binom{4}{1} \cdot \pi \cdot (1 - \pi)^3 \\
&P(X = 1) = 0,260
\end{split}
\end{equation*}

\subsubsection{Binomické rozdělení}
\begin{itemize}
	\item \textbf{X .. počet úspěchů v $n$ Bernoulliho pokusech} $$X \sim Bi(n;\pi)$$ \\ $n$ -- počet pokusů \\ $\pi$ -- pravděpodobnost úspěchu
 	\begin{figure}[H]
	\centering
	\includegraphics[width=0.8\textwidth]{assets/12_binom}
	\end{figure}
	\item \textbf{Střední hodnota}: $\qquad$ $E(X) = n\pi$
	\item \textbf{Rozptyl}: $\qquad\qquad\qquad\;\; D(X) = n\pi(1 - \pi)$
	\item Vlastnosti
	\begin{figure}[H]
	\centering
	\includegraphics[width=0.7\textwidth]{assets/12_binom_vlast}
	\caption{$\pi \not= 0,5$, s rostoucím $n$ se rozdělení stává více a více symetrickým}
	\end{figure}
	\begin{figure}[H]
	\centering
	\includegraphics[width=0.7\textwidth]{assets/12_binom_vlast2}
	\end{figure}
	\item[$\circ$] \textit{Příklad} Mezi 200 vajíčky určenými pro prodej v jisté maloobchodní prodejně je 50 vajíček prasklých. Jaká je pravděpodobnost, že vybereme--li si náhodně 20 vajec, bude 8 z nich prasklých?
	\begin{itemize}
		\item $X$ … počet prasklých vajec z 20 vybraných (výběr bez vracení)

		\begin{center}
		\begin{tikzpicture}[sibling distance=8em,
			  every node/.style = { align=center,
			    top color=white, bottom color=white}]]
			  \node {celkem vajec\\200}
			    child { node {prasklých vajec\\50} }
			    child { node {neprasklých vajec\\150} };
		\end{tikzpicture}
		\end{center}
		$$P(X = 8) = \frac{\binom{50}{8} \binom{150}{12}} {\binom{200}{20}}$$ 
		\item[] \textbf{počet příznivých možností} -- vybíráme 8 prasklých z 50 prasklých a zároveň 12 ze 150 neprasklých \\ \textbf{počet všech možností }-- vybíráme 20 vajec z 200
	\end{itemize}
\end{itemize}

\subsubsection{Hypergeometrické rozdělení}
\begin{itemize}
	\item \textbf{ $\mathbf{X}$ ... počet prvků se sledovanou vlastností ve výběru $n$ prvků}
	\item V souboru \textbf{N} prvků je \textbf{M} s danou vlastností a zbylých (\textbf{N -- M}) prvků tuto vlastnost nemá. Postupně vybereme ze souboru \textbf{n} prvků, z nichž žádný \textbf{nevracíme zpět}. 
	$$X \sim H(N;M;n)$$ \\ $N$ -- rozsah základního souboru \\ $M$ -- počet prvků s danou vlastností \\ $n$ -- rozsah výběru
	\item \textbf{Pravděpodobnostní funkce}: $P(X = k) = \frac {\mathbin{\color{blue} \binom{M}{k} \binom{N-M}{n-k}}} {\mathbin{\color{magenta}\binom{N}{n}}}$
	\begin{itemize}
		\item \textcolor{blue}{Počet příznívých možností, tj. počet možností jak vybrat $k$ prvků s danou vlastností z $M$ a zároveň $(n-k)$ prvků, které uvedenou vlastnost nemají z $(N - M)$ prvků.}
		\item \textcolor{magenta}{Počet všech možností, jak vybrat $n$ prvků z $N$ (nezáleží na pořadí).}
	\end{itemize}
	\item \textbf{Střední hodnota}: $\qquad$ $E(X) = n \cdot \frac{M}{N}$
	\item \textbf{Rozptyl}: $\qquad\qquad\qquad\;\; D(X) = n \cdot \frac{M}{N}(1 - \frac{M}{N})(\frac{N-n}{N-1})$
	\item \textbf{Možnost aproximace} -- je--li $n/N$, tzv. \textbf{výběrový poměr}, menší než $0,05$, lze hypergeometrické rozdělení nahradit binomickým s parametry $n$ a $M/N$. $$(\frac{n}{N} < 0,05) \Rightarrow [H(N;M;n) \approx Bi(n;\frac{M}{N})]$$
\end{itemize} 

\subsubsection{Negativně binomické (Pascalovo) rozdělení}
\begin{itemize}
	\item \textbf{$X$ ... počet Bernoulliho pokusů do $k.$ výskytu události (úspěchu), včetně $k.$ výskytu} 
	$$X \sim NB(k;\pi)$$ \\ $k$ -- požadovaný počet úspěchů (výskytu události) \\ $\pi$ -- pravděpodobnost úspěchu
	\item \textbf{Pravděpodobnostní funkce:}
	\begin{figure}[H]
	\centering
	\includegraphics[width=0.7\textwidth]{assets/12_neg_binom}
	\end{figure}
	\item \textbf{Střední hodnota}: $\qquad$ $E(X) = \frac{k}{\pi}$
	\item \textbf{Rozptyl}: $\qquad\qquad\qquad\;\; D(X) = \frac{k(1- \pi)}{\pi^2}$
	\item V případě negativně binomické náhodné veličiny není definice ustálená. Někteří statistici (popř. statistický software) ji definují jako \textbf{počet neúspěchů před $k.$ úspěchem}.
\end{itemize}

\subsubsection{Poissonův proces}
\textbf{Bodový proces}
\begin{itemize}
	\item Sledujeme chod procesu, v němž čas od času dochází k nějaké význačné události \\
	\begin{figure}[H]
	\centering
	\includegraphics[width=0.5\textwidth]{assets/12_poisson_proces}
	\end{figure}
	\item Registrujeme počet událostí $N(s;t)$ v časovém intervalu $<s;s + t>$
	\item \textbf{Rychlost výskytu události $\lambda$} je parametrem Poissonova procesu
	\item Poissonův proces lze obdobně jako v časovém intervalu definovat na libovolné uzavřené prostorové oblasti (na ploše, v objemu).
\end{itemize}
Jako \textbf{Poissonův proces} označujeme proces, který je:
\begin{itemize}
	\item \textbf{ordinální} -- pravděpodobnost výskytu více než jedné události v limitně krátkém časovém intervalu $(t \to 0)$ je nulová. (tzv. \textbf{řídké jevy}),
	\item \textbf{stacionální} -- rychlost výskytu událostí $\lambda$ je konstantní v průběhu sledovaného intervalu,
	\item \textbf{beznásledný} -- pravděpodobnost výskytu událostí není závislá na čase, který uplynul od minulé události.
\end{itemize}

\subsubsection{Poissonovo rozdělení}
\begin{itemize}
	\item \textbf{$X$ ... počet výskytu události v uzavřené oblasti (v čase, na ploše, v objemu)}
	\item náhodný pokus jako Poissonův proces (nezávislé události probíhající v čase $t$, s rychlostí výskytu $\lambda$; popř. nezávislé události objevující se na ploše $t$, resp. v objemu $t$ s hustotou výskytu $\lambda$).
	$$X \sim Po(\lambda t)$$
	\item \textbf{Pravděpodobnostní funkce}: $P(X = k) = \frac{(\lambda t)^k}{k!}e^{-\lambda t}$
	\item \textbf{Střední hodnota}: $\qquad$ $E(X) = \frac{k}{\pi}$
	\item \textbf{Rozptyl}: $\qquad\qquad\qquad\;\; D(X) = \frac{k(1- \pi)}{\pi^2}$
\end{itemize}


\subsection{Spojitá náhodná veličina}
\begin{itemize}
	\item Hustota pravděpodobnosti f(x) -- funkce $f(x)$ je \textbf{hustotou pravděpodobnosti} (na intervalu $a \leq x \leq b$), jestliže splňuje následující podmínky:
	\begin{itemize}
		\item $f(x) \geq 0; x \in \mathbb{R}$
		\item plocha pod křivkou hustoty je rovna 1 ($\int_{-\infty}^{\infty} f(x)dx = 1$).
	\end{itemize}
\end{itemize}

\subsubsection{Exponenciální rozdělení}
\begin{itemize}
	\item \textbf{$X$ ... délka časových intervalů mezi událostmi v Poissonově procesu}
	$$X \sim Exp(\lambda)$$
	\item Předpokládá konstantní intenzitu události $\lambda(t)$- rozdělení \uv{bez paměti}.
	\item \textbf{Hustota pravděpodobnosti}:
	 $f(t) =	\begin{cases}
 			   \lambda \cdot e^{-\lambda t}       & \quad t > 0\\
 			   0,  & \quad t \leq 0.
 			 \end{cases}$
	\item \textbf{Distribuční funkce}:
	$\qquad\qquad F(t) =	\begin{cases}
 			   1 -  e^{-\lambda t}       & \quad t > 0\\
 			   0,  & \quad t \leq 0.
 			 \end{cases}$
	\item \textbf{Střední hodnota}: $\qquad\qquad\quad E(X) = \frac{1}{\lambda}$
	\item \textbf{Rozptyl}: $\qquad\qquad\qquad\qquad\quad D(X) = (E(X))^2 = \frac{1}{\lambda^2}$ \\
	\item \textbf{Riziková funkce} (intenzita poruch) $\lambda (t)$
	\begin{itemize}
		\item Pro nezápornou náhodnou veličinu $X$ se spojitým rozdělením popsaným distribuční funkcí $F(t)$ definujeme pro $F(t) \not= 1$ rizikovou funkci jako  $$\lambda(t) = \frac{f(t)}{1 - F(t)}.$$
		\item $\lambda(t)$ -- Představuje-li náhodná veličina X dobu do poruchy nějakého zařízení, pak pravděpodobnost, že pokud do času t nedošlo k žádné poruše, tak k ní dojde v následujícím krátkém úseku délky $\Delta t$, je přibližně 
		$$P(t \geq X <t + \Delta t|X > t) = \lambda(t) \cdot \Delta t$$
			\begin{figure}[H]
	\centering
	\includegraphics[width=0.7\textwidth]{assets/12_exp_rizk_fce}
	\end{figure}
	\item[$\circ$] \textit{Příklad} Střední doba čekání zákazníka na obsluhu v prodejně je 50 sekund. Doba čekání se řídí exponenciálním rozdělením (pravděpodobnost, že zákazník nebude obsloužen klesá s rostoucím časem exponenciálně). Jaká je pravděpodobnost, že náhodný zákazník bude obsloužen dříve než za 30 sekund?
	\begin{itemize}
		\item[] X -- doba čekání na obsluhu
		\item[] $X \sim Exp (\lambda = \frac{1}{50}s^-1), E(X) = \frac{1}{\lambda} = 50 s$
	\end{itemize}
\end{itemize}
\end{itemize}

\subsubsection{Weibullovo rozdělení}
\begin{itemize}
	\item \textbf{X ... délka časových intervalů mezi událostmi v Poissonově procesu}
	\item Zobecnění exponenciálního rozdělení
	\item Tímto rozdělením lze modelovat i dobu do výskytu události u systémů (jedinců), které jsou v období dětských nemocí, resp. období stárnutí.
	$$X \sim W(\theta;\beta)$$
	\\ $\theta$ -- parametr měřítka (scale) $\theta = \frac{1}{\lambda}; \theta > 0$ \\ $\beta$ -- parametr tvaru (shape); $\beta > 0$
	\item \textbf{Hustota pravděpodobnosti}:
	 $f(t) =	\begin{cases}
 			   \beta\lambda^{\beta} t^{\beta-1}e^{-(\lambda t)^{\beta}},       & \quad t > 0\\
 			   0,  & \quad t \leq 0.
 			 \end{cases}$
	\item \textbf{Distribuční funkce}:
	$\qquad\qquad F(t) =	\begin{cases}
 			   1 -  e^{-(\lambda t)^{\beta} }      & \quad t > 0\\
 			   0,  & \quad t \leq 0.
 			 \end{cases}$
 	\item \textbf{Riziková funkce}:
	$\qquad\qquad\quad\; \lambda(t) =	\begin{cases}
 			   \beta\lambda^\beta t^{\beta-1}      & \quad t > 0\\
 			   0,  & \quad t \leq 0.
 			 \end{cases}$
 	\begin{figure}[H]
	\centering
	\includegraphics[width=0.7\textwidth]{assets/12_weib_riz_fce}
	\caption{Riziková funkce}
	\end{figure}
\end{itemize}
\subsubsection{Normální rozdělení}
\begin{itemize}
	\item Bývá vhodné k popisu náhodných veličin, které lze interpretovat jako aditivní výsledek mnoha nepatrných a vzájemně nezávislých faktorů (výška člověka, IQ, délka končetiny).
	\item Popisuje náhodné veličiny, jejichž hodnoty se symetricky shlukují kolem střední hodnoty a vytvářejí tak charakteristický tvar hustoty pravděpodobnosti známý pod názvem \textbf{Gaussova křivka}. $$X \sim N (\mu;\sigma^2)$$  \\$\mu$ -- střední hodnota \\ $\sigma^2$ -- rozptyl
	\item \textbf{Hustota pravděpodobnosti}: $f(x) = \frac{1}{\sigma \sqrt{2\pi}} e^{-\frac{1}{2}({\frac{x - \mu}{\sigma}})^2}$
	\begin{figure}[H]
	\centering
	\includegraphics[width=0.7\textwidth]{assets/12_norm_roz}
	\caption{Riziková funkce}
	\end{figure}
	\item \textbf{Distribuční funkce}: $F(x) = \frac{1}{\sigma \sqrt{2\pi}} \int_{-\infty}^{x} e^{-\frac{1}{2}(\frac{t - \mu}{\sigma})^2} dt$\\ (integrál nelze řešit analyticky)
\end{itemize}

%BTICH END

\pagebreak
\section{Popisná statistika. Číselné charakteristiky a vizualizace kategoriálních a kvantitativních proměnných.}
 \subsection{Kvantitativní -- Numerická proměnná}
\begin{itemize}
	\item \textbf{Míry polohy} --  určující typické rozložení hodnot proměnné (jejich rozmístění na číselné ose).

 		\begin{itemize}
 			\item Průměr -- $\bar{x} = \frac{\sum\limits_{i=1}^n x_i}{n}$
 			\item Modus -- střed shorthu
 			\item Kvantily -- dolní kvartil, médián, horní kvartil,..
 		\end{itemize}
 	\item \textbf{Míry variability} -- určující variabilitu (rozptyl) hodnot kolem své typické polohy.
 		\begin{itemize}
 			\item Variační rozpětí $x_{max} - x_{min}$
 			\item Inverkvartilové rozpětí -- $IQR = x_{0,75} - x_{0,25}$
 			\item Výběrový rozptyl
 			\item Výběrová směrodatná odchylka
 			\item Variační koeficient
 		\end{itemize}
 	\item \textbf{Míra šikmosti a špičatosti}	
 		\begin{itemize}
 			\item Výběrová šikmost
 			\item Výběrová špičatost
 		\end{itemize}
 	\item \textbf{Identifikace odlehlých pozorování}	
 		\begin{itemize}
 			\item Vnitřní hradby dolní mez: $h_D = x_{0,25} - 1,5IQR$, horní mez: $h_H = x_{0,75} + 1,5IQR$
 			\item Vnější hradby dolní mez: $h_D = x_{0,25} - 3IQR$, horní mez: $h_H = x_{0,75} + 3IQR$
 			\item Z--souřadnice
 			\item Mediánová souřadnice
 		\end{itemize}
\end{itemize}

\textbf{Grafické zobrazení numerické proměnné:}
\begin{itemize}
	\item Empirická distribuční funkce
	\item Krabicový graf (box plot)
	\item Číslicový histogram (lodyha s listy, steam and leaf)
\end{itemize}

 \subsection{Kvalitativní --Kategoriální proměnná}
\begin{enumerate}[label=\alph*)]
	\item Nominální proměnná -- nemá smysl uspořádaní
	\begin{itemize}
		\item \textbf{Základní statistiky pro popis nominální proměnné:}
	 		\begin{itemize}
	 			\item četnost
	 			\item relativní četnost
	 			\item modus
	 		\end{itemize}
	\item \textbf{Grafické zobrazení nominální proměnné:}
			\begin{itemize}
				\item histogram
				\item výsečový graf
			\end{itemize}
	\end{itemize}
	\item Ordinální proměnná -- má smysl uspořádání
	\begin{itemize}
		\item \textbf{Základní statistiky pro popis ordinální proměnné:}
	 		\begin{itemize}
	 			\item četnost
	 			\item relativní četnost
	 			\item kumulativní četnost
	 			\item relativní kumulativní četnost
	 			\item modus
	 		\end{itemize}
		\item \textbf{Grafické zobrazení ordinální proměnné:}
				\begin{itemize}
						\item histogram
						\item výsečový graf
						\item Lorenzova křivka
						\item Paretův graf
				\end{itemize}
	\end{itemize}
\end{enumerate}
\textbf{Paretův princip} – 80\% následků pramení z 20\% příčin.\\
\textbf{Paretova analýza} – postup vedoucí k nalezení \uv{životně důležité menšiny} (spektra příčin ovlivňujících rozhodujícím způsobem následky).


\subsection{Míry polohy a variability}
Snad nejpoužívanějšími mírami polohy jsou průměry proměnných. Průměry představují průměrnou nebo typickou hodnotu výběrového souboru. Zřejmě nejznámějším průměrem pro kvantitativní proměnnou je \textbf{aritmetický průměr}
\begin{itemize}
	\item \textbf{Aritmetický průměr} $\mathbf{\bar{x}}$ (mean)
	\item \textbf{Aritmetický vážený průměr}
	\item \textbf{Harmonický průměr}  -- Pro výpočet průměru v případech, kdy proměnná má charakter části z celku (úlohy o společné práci, ...).
	\item \textbf{Harmonický vážený průměr} -- Pokud máme údaje setříděné do tabulky četností.
	\item \textbf{Geometrický průměr} -- Pracujeme-li s kladnou proměnnou představující relativní změny (růstové indexy,cenové indexy...). $\bar{x}_G = \sqrt[n]{x_1 \cdot x_2 \cdot ... \cdot x_n}$
	\item \textbf{Geometrický vážený průměr} -- Pracujeme-li s kladnou proměnnou představující relativní změny (růstové indexy,cenové indexy...). $\bar{x}_G = \sqrt[n]{x_1^{n_1} \cdot x_2^{n_2} \cdot ... \cdot x_n^{n_k}}$ kde $n = \sum\limits_{i=1}^k n_i$. \\

	\item \textbf{Modus} -- Pro \textbf{diskrétní proměnnou} definujeme modus jako hodnotu nejčetnější varianty proměnné (podobně jako u kvalitativní proměnné), u spojité proměnné považujeme za modus $\mathbf{\hat{x}}$ hodnotu kolem níž je největší koncentrace hodnot proměnné. Mnohdy mluvíme o typické hodnotě proměnné. Pro určení této hodnoty využijeme tzv. \textbf{short}, což je nejkratší interval, v němž leží alespoň 50\% hodnot proměnné (v případě výběru o rozsahu $n = 2k(k \in N)$ (sudý počet hodnot), leží v shorthu $k$ hodnotě - což je 50\% $(n/2)$ hodnot proměnné, v případě výběru o rozsahu $n = 2k + 1(k \in N)$ (lichý počet hodnot), leží v shorthu $k + 1$ hodnot - což je o 1 více než je 50\% hodnot proměnné). \textbf{Modus pak definujeme jako střed shorthu}. \\

	\item \textbf{Výběrové kvantily} (quantile, resp. percentile) -- jsou to statistiky, které charakterizují polohu jednotlivých hodnot v rámci proměnné. Podobně jako modus, jsou i výběrové kvantily \textbf{rezistentní} (odolné) vůči odlehlým pozorováním. Obecně je výběrový kvantil chápán jako hodnota, která rozděluje výběrový soubor na dvě části -- hodnoty, které jsou menší než daný kvantil, druhá část obsahuje hodnoty, které jsou větší nebo rovno danému kvantilu. Pro určení kvantilu je nutné \textbf{výběr uspořádat} od nejmenší hodnoty k největší. \\

	\item \textbf{Kvartily}
		\begin{itemize}
			\item \textbf{Dolní kvartil} $x_{0,25}$ -- $25\%$--ní kvartil  (rozděluje datový soubor tak, že 25\% hodnot je menších než tento kvartil a zbytek, tj. 75\% větších (nebo rovných))
			\item \textbf{Medián} $x_{0,5}$ -- $50\%$--ní kvartil
			\item \textbf{Horní kvartil} $x_{0,75}$ -- $75\%$--ní kvartil\\
			Kvartily dělí výběrový soubor na 4 přibližně stejně velké části.
			\item \textbf{Decily} -- $x_{0,1};x_{0,2};...;x_{0,9}$ -- Decily dělí výběrový soubor na 10 přibližně stejně četných částí.
			\item \textbf{Percentily} -- $x_{0,01};x_{0,02};...;x_{0,99}$ -- Percentily dělí výběrový soubor na 100 přibližně stejně četných částí.
		\end{itemize}

	\item \textbf{Empirická distribuční funkce F(x) pro kvantitativní proměnnou}
	\begin{itemize}
		\item Empirická distribuční funkce je monotónně rostoucí, zleva spojitou funkcí, která \uv{skáče} podle relativních četností příslušných jednotlivým hodnotám proměnné. 
		\item Označme si $p(x_i)$ relativní četnost hodnoty xi seřazeného výběrového souboru $x_1 < x_2 < ... < x_n$. Pro empirickou distribuční funkci $F(x)$ pak platí:
				\begin{figure}[H]
				\centering
				\includegraphics[width=0.6\textwidth]{assets/13_empiricka}
				\end{figure}
	\end{itemize}

	\item \textbf{Interkvartilové rozpětí IQR} -- Tato statistika je mírou variability souboru a je definována jako vzdálenost mezi horním a dolním kvartilem.
	\item \textbf{MAD} (median absolute deviation from the median) -- medián absolutních odchylek od mediánu 
			\begin{enumerate}
				\item Výběrový soubor uspořádáme podle velikosti.
				\item Určíme medián souboru.
				\item Pro každou hodnotu souboru určíme absolutní hodnotu její odchylky od mediánu.
				\item Absolutní odchylky od mediánu uspořádáme podle velikosti.
				\item Určíme medián absolutních odchylek od mediánu, tj. MAD.
			\end{enumerate}
	\item \textbf{Výběrový rozptyl} $\mathbf{s^2}$ (\uv{s kvadrát}, sample variance) --
		\begin{itemize}
			\item  je nejrozšířenější mírou variability výběrového souboru
			\begin{equation*}
				s^2 = \frac{\sum\limits_{i=1}^n (x_1 - \bar{x})^2} {n - 1}
			\end{equation*}
		\item Výběrový rozptyl je dán podílem součtu kvadrátu odchylek jednotlivých hodnot od průměru a rozsahu souboru sníženého o jedničku.
		\item Nevýhodou použití výběrového rozptylu jakožto míry variability je to, že jednotka této charakteristiky je druhou mocninou jednotky proměnné. Např. je--li proměnnou denní tržba uvedena v Kč, bude výběrový rozptyl této proměnné vyjádřen v Kč$^2$.
		\item Následující míra variability tuto vlastnost nemá.
	\end{itemize}
	\item \textbf{Výběrová směrodatná odchylka s} -- je definována jako kladná odmocnina výběrového rozptylu
	\begin{equation*}
				s = \sqrt{s^2} = \sqrt{\frac{\sum\limits_{i=1}^n (x_1 - \bar{x})^2} {n - 1}}.
	\end{equation*}
	Nevýhodou výběrového rozptylu i výběrové směrodatné odchylky je skutečnost, že neumožňují porovnávat varibilitu proměnných vyjádřených v různých jednotkách. Která proměnná má větší variabilitu – výška nebo hmotnost dospělého člověka? Na tuto otázku nám dá odpověď tzv. variační koeficient.
	\item \textbf{Variační koeficient} $\mathbf{V_x}$ -- vyjadřuje relativní míru variability proměnné x. Podle níže uvedeného vztahu jej lze stanovit pouze pro proměnné, které nabývají výhradně kladných hodnot. Variační koeficient je bezrozměrný. Uvádíme-li jej v [\%], hodnotu získanou z definičního vzorce vynásobíme 100\%.
	\begin{equation*}
				V_x = \frac{V}{\bar{x}}, popr. V_x = \frac{V}{\bar{x}} \cdot 100 [\%].
	\end{equation*}
\end{itemize}

\subsection{Identifikace odlehlých pozorování}
\begin{itemize}
	\item \textbf{Vnitřní hradby} -- Za odlehlé pozorování lze považovat takovou hodnotu $x_i$ , která je od dolního, resp. horního kvartilu vzdálená více než 1,5 násobek interkvartilového rozpětí. Tedy: $[(x_i < x_{0,25} - 1,5 \cdot IQR) \vee (x_i > x_{0,75} + 1,5 \cdot IQR)] \Rightarrow x_i$ je odlehlým pozorováním. 
	\item \textbf{z--souřadnice (z--skóre)} Za odlehlé pozorování lze považovat takovou hodnotu $x_i$, jejíž absolutní hodnota z-souřadnice je větší než 3, tj. hodnota, která je od průměru vzdálenější než 3s. Tedy: $z-skore_i = \frac{x_i - \bar{x}}{s}$ \\	
		$|z-skore_i| > 3 \Rightarrow |\frac{x_i - \bar{x}}{s}| >3s \Rightarrow x_i$ je odlehlým pozorováním.

	\item $\mathbf{x_{0,5}}$\textbf{--souřadnice ($\mathbf{x_{0,5}}$--skóre)} -- : Za odlehlé pozorování lze považovat takovou hodnotu $x_i$, jejíž absolutní hodnota mediánové souřadnice je větší než 3, tj. hodnota, která je od mediánu vzdálenější než $3 \cdot 1,483\cdot MAD$. Tedy: $x_{0,5}-skore_i = \frac{x_i - x_{0,5}}{1,483MAD}$ \\	
		$|x_{0,5}-skore_i| > 3 \Rightarrow |\frac{x_i - x_{0,5}}{1,483MAD}| >3 \Rightarrow |x_i - x_{0,5}| >3 \cdot 1,483MAD \Rightarrow x_i$ je odlehlým pozorováním.
\end{itemize}

\subsection{Míra šikmosti a špičatosti}	
 		\begin{itemize}
 			\item \textbf{Výběrová šikmost a} (skewness) -- vyjadřuje asymetrii rozložení hodnot proměnné kolem jejího průměru.
 				\begin{itemize}
 					\item $a = 0$ -- hodnoty proměnné jsou kolem jejího průměru rozloženy symetricky
					\item $a > 0$ -- u proměnné převažují hodnoty menší než průměr
					\item $a < 0$ -- u proměnné převažují hodnoty větší než průměr
				\begin{figure}[H]
				\centering
				\includegraphics[width=0.6\textwidth]{assets/13_sikmost}
				\end{figure}
				\end{itemize}
				\item \textbf{Souvislost mezi šikmostí a charakteristikami polohy}
				\begin{itemize}
							\item Symetrické rozdělení: $\bar{x} = x_{0,5}$
							\item Pozitivně zešikmené rozdělení: $\bar{x} > x_{0,5}$
							\item Negativně zešikmené rozdělení: $\bar{x} < x_{0,5}$
				\end{itemize}
 			\item \textbf{Výběrová špičatost b} (kurtosis) -- vyjadřuje koncentraci hodnot proměnné kolem jejího průměru.
 				\begin{itemize}
 					\item $b = 0$ -- špičatost odpovídá normálnímu rozdělení (bude definováno později)
					\item $b > 0$ -- špičaté rozdělení proměnné
					\item $b < 0$ -- ploché rozdělení proměnné
				\begin{figure}[H]
				\centering
				\includegraphics[width=0.6\textwidth]{assets/13_spicatost}
				\end{figure}
				\end{itemize}
 		\end{itemize}

\subsection{Grafické znázornění kvalitativní proměnné}
\begin{itemize}
	\item \textbf{Krabicový graf} (Box plot)
	\begin{itemize}
		\item Odlehlá pozorování jsou znázorněna jako izolované body, konec horního (popř. konec dolního) vousu představují maximum (popř. minimum) proměnné po vyloučení odlehlých pozorování,\uv{víko} krabice udává horní kvartil, \uv{dno} dolní kvartil, vodorovná úsečka uvnitř krabice označuje medián.
		\item Z polohy mediánu vzhledem ke \uv{krabici} lze dobře usuzovat na symetrii vnitřních 50\% dat a my tak získáváme dobrý přehled o středu a rozptýlenosti proměnné.
	\end{itemize}
				\begin{figure}[H]
				\centering
				\includegraphics[width=0.5\textwidth]{assets/13_box_plot}
				\end{figure}
	\item \textbf{Číslicový histogram} (Lodyha s listy, angl. Stem and leaf plot)
	\begin{itemize}
		\item Jak jsme si ukázali, výhodou krabicového grafu je jeho jednoduchost, někdy nám však chybí informace o konkrétních hodnotách proměnné. 
		\item Chtěli bychom proto nějak přehledně zapsat číselné hodnoty výběru a k tomu nám slouží právě číslicový histogram.
		\item Navíc nám tento graf dává dobrou představu o šikmosti proměnné.
		\item Příklad: \textit{Představme si proměnnou představující průměrné měsíční platy zaměstnanců ve státní správě}.
		\begin{table}[H]
	\centering
	\begin{tabular}{|l|}
		\hline
		\textbf{Průměrný měsíční plat [Kč]}                                \\ \hline
		10 654, 9 765, 8 675, 12 435, 9 675, 10 343, 18 786, 15 420, 8 675,	7 132, \\
		6 732,6 878, 15 657, 9 754, 9 543, 9 435, 10 647, 12 453, 9 987, 10 342.                                                                 \\ \hline

	\end{tabular}
\end{table}
				\begin{figure}[H]
				\centering
				\includegraphics[width=0.6\textwidth]{assets/13_cislicovy_hist}
				\end{figure}
		\begin{itemize}
		 \item Pro naší informaci nejsou tak důležité koruny ani desetikoruny rozdílu. V tomto případě se nám jedná přinejmenším o stokoruny.
		 \item Co kdybychom tedy informaci o \uv{nedůležitých} řádech zanedbali a znázornili setříděná data pouze na základě vyšších řádů? My jsme se rozhodli, že důležitý řád jsou pro nás stokoruny.
		 \item Hodnoty stojící o řád výš (v našem případě tisíce) zapíšeme setříděné pod sebe, tak, že tvoří jakýsi stonek (\textbf{lodyhu}), přičemž pod graf uvedeme tzv. \textbf{šířku lodyhy}, která udává koeficient, jímž se hodnoty uvedené v grafu násobí.
		 \item Druhý sloupec grafu, \textbf{listy}, budou tvořit číslice, reprezentujíci zvolený \uv{důležitý} řád, zapisované do příslušných řádků (opět seřazené podle velikosti).
		 \item Třetí sloupec udává absolutní četnosti příslušné daným řádkům.
		\end{itemize}
	\end{itemize}
\end{itemize}

\subsection{Statistické charakteristiky kvalitativních proměnných}
\subsubsection{Nominální proměnná}
Nominální proměnná nabývá v rámci souboru různých, avšak rovnocenných kategorií. Počet těchto kategorií nebývá příliš vysoký, a proto první statistickou charakteristikou, kterou k popisu proměnné použijeme je četnost.
\begin{itemize}
	\item \textbf{Četnost} $\mathbf{n_i}$ (absolutní četnost, \uv{frequency}) --  je definována jako počet výskytu dané varianty kvalitativní proměnné. V případě, že kvalitativní proměnná ve statistickém souboru o rozsahu $n$ hodnot  nabývá $k$ různých variant, jejichž četnosti označíme $n_1, n_2, ... , n_k$, musí zřejmě platit $n_1 + n_2 + ... + n_k = \sum\limits_{i=1}^k n_i = n$. \\
	Chceme-li vyjádřit, jakou část souboru tvoří proměnné s některou variantou, použijeme pro popis proměnné relativní četnost.

	\item \textbf{Relativní četnost} $\mathbf{p_i}$ (\uv{relative frequency}) je definována jako $p_i = \frac{n_i}{n}$, popř. $p_i = \frac{n_i}{n} \cdot 100 [\%]$. Pro relativní četnosti musí platit $p_1 + p_2 + ... + p_k = \sum\limits_{i=1}^k p_i = p$. \\
	Při zpracování kvalitativní proměnné je vhodné četnosti i relativní četnosti uspořádat do tzv. \textbf{tabulky rozdělení četnosti} (\uv{frequency table})
				\begin{figure}[H]
				\centering
				\includegraphics[width=0.6\textwidth]{assets/13_tabl_cetnost}
				\end{figure}
	\item \textbf{Modus} -- s definujeme jako název varianty proměnné vykazující nejvyšší četnost.\\
	\item \textbf{Grafické znázornění nominální proměnné} 
	\begin{itemize}
		\item \textbf{Histogram} -- je klasickým grafem, v němž na jednu osu vynášíme varianty proměnné a na druhou osu jejich četnosti. Jednotlivé hodnoty četností jsou pak zobrazeny jako výšky sloupců (obdélníků, popř. hranolů, kuželů...)
		\item \textbf{Výsečový graf} -- prezentuje relativní četnosti jednotlivých variant proměnné, přičemž jednotlivé relativní četnosti jsou úměrně reprezentovány plochami příslušných kruhových výsečí. (Změnou kruhu na elipsu dojde k trojrozměrnému efektu.)
	\end{itemize}
\end{itemize}
\subsubsection{Ordinální proměnná}
Ordinální proměnná, stejně jako proměnná nominální, nabývá v rámci souboru různých slovních variant, avšak tyto varianty mají přirozené uspořádání, tj. můžeme určit, která je \uv{menší} a která \uv{větší}. 
\begin{itemize}
	\item \textbf{Kumulativní četnost} $\mathbf{m_i}$ i (\uv{cumulative frequency}) -- definujeme jako počet hodnot proměnné, které nabývají varianty nižší nebo rovné i-té variantě. \\ 
	\textit{Uvažte např. proměnnou \uv{známka ze statistiky}, která nabývá variant: \uv{výborně}, \uv{velmi dobře}, \uv{prospěl}, \uv{neprospěl}, pak např. kumulativní četnost pro variantu \uv{prospěl} bude rovna počtu studentů, kteří ze statistiky získali známku \uv{prospěl} nebo lepší.} \\
	Jsou-li jednotlivé varianty uspořádány podle své \uv{velikosti}(\uv{$x_1 < x_2 < ... < x_k$}), platí $m_i = \sum\limits_{i=1}^i n_j$, Je tedy zřejmé, že kumulativní četnost $k$--té (\uv{nejvyšší}) varianty je rovna rozsahu proměnné $– mk = n$.

	\item \textbf{Kumulativní relativní četnost} $\mathbf{F_i}$ (\uv{cumulative relative frequency}) -- vyjadřuje jakou část souboru tvoří hodnoty nabývající i-té a nižší varianty. $F_i = \sum\limits_{j=1}^i p_j,$ což není nic jiného než relativní vyjádření kumulativní četnosti: $F_i = \frac{m_i}{n}$. \\ 
	Obdobně jako pro nominální proměnné, můžeme i pro proměnné ordinální prezentovat statistické charakteristiky pomocí tabulky rozdělení četnosti. Ta obsahuje ve srovnání s tabulkou rozdělení četností pro nominální proměnnou navíc hodnoty kumulativních a kumulativních relativních četností.\\
				\begin{figure}[H]
				\centering
				\includegraphics[width=0.8\textwidth]{assets/13_tabl_cetnost_ord}
				\end{figure}
	\item \textbf{Grafické znázornění ordinální proměnné}
	\begin{itemize}
		\item \textbf{Lorenzova křivka} ((polygon kumulativních četností, Galtonova ogiva, S křivka) -- S křivka) je spojnicovým grafem, který získáme tak, že na vodorovnou osu vynášíme jednotlivé varianty proměnné v pořadí od \uv{nejmenší} do \uv{největší} a na svislou osu příslušné hodnoty kumulativních četností. Znázorněné body spojíme úsečkami.
				\begin{figure}[H]
				\centering
				\includegraphics[width=0.5\textwidth]{assets/13_lorenzo}
				\end{figure}
		\item \textbf{Paretova analýza} -- lze formulovat tak, že 80\% následků pramení z 20\% příčin (20\% lidí vlastní 80\% celkového bohatství). V praxi pak bývá snahou nalézt toto malé spektrum příčin (životně důležitá menšina), které tak významně ovlivňuje výsledek.
	\end{itemize}
\end{itemize}


\pagebreak
\section{Metody statistické indukce. Intervalové odhady. Princip testování hypotéz.}
%Metody statistické indukce. Intervalové odhady. Princip testování hypotéz.

\subsection{Statistická indukce}
Statistická indukce je metoda, která dovoluje stanovit vlastnost celku (\textbf{základního souboru}) na základě pozorování jeho částí (\textbf{náhodného výběru}).
\begin{figure}[H]
\centering
\includegraphics[width=0.6\textwidth]{assets/14_stat_ind}
\end{figure}

\subsubsection{Základní soubor (populace)}
\begin{itemize}
	\item Je množina všech teoreticky možných objektů (např. jedinců) v uvažované situaci = statistický soubor, který je vymezen cílem výzkumu a pro který vyvozujeme závěry výzkumného šetření.
	\item Charakterizuje se \textbf{parametrem}, což je např. výška, váha, IQ, atp.
	\item Má konečný nebo nekonečný (hypotetický) \textbf{rozsah}, který je dán N (např.: N = 150 lidí, opic, rostlin,...).
\end{itemize}
\subsubsection{Výběrový soubor (výběr)}
\begin{itemize}
	\item Je část populace vybrané na základě předem stanovených kritérii resp. pravidel (podmnožina základního souboru).
	\item \textbf{O náhodném výběru} uvažujeme, když splňuje dvě základní vlastnosti:
\begin{itemize}
\item \textbf{pravděpodobnost} zařazení do vzorku je pro všechny statistické jednotky populace \textbf{nenulová},
\item statistické jednotky jsou do vzorku vybrané \textbf{nezávisle} jedna od druhé.
\end{itemize}
	\item \textbf{O reprezentativním výběru} uvažujeme, když výběrový soubor dobře odráží strukturu celého zkoumaného souboru.
\end{itemize}
\subsubsection{Principy statistického usuzování}
\begin{enumerate}
	\item Statistické usuzování znamená zobecňování z výběrových statistik na parametry rozdělení.
	\item Abychom mohli provést statistické usuzování, musíme mít nějakou teorii, jež popisuje náhodné chování sledovaných proměnných.
	\item Existují dva typy výběrových chyb: \textbf{náhodné výběrové chyby} a \textbf{systematické chyby}. Získáním náhodného výběru zmenšujeme systematickou chybu a získáváme podklad pro odhad náhodné výběrové chyby.
	\item Výběrová rozdělení statistik jsou teoretická \textbf{pravděpodobnostní rozdělení}, která popisují vztah mezi výběrovou statistikou a populací.
	\item Směrodatná odchylka výběrového rozdělení statistiky (odhad parametru) se nazývá směrodatná chyba. Odhaduje náhodnou výběrovou chybu vypočítané statistiky (odhadu parametru).
	\item Jak roste velikost výběru, výběrová chyba a směrodatná chyba se zmenšují.
	\item Směrodatná chyba se používá k získání intervalového odhadu parametrů i k testování hypotéz o parametrech rozdělení.
\end{enumerate}
\subsection{Základní metody statistické indukce}
\begin{itemize}
	\item \textbf{Intervalové odhady} (confidence intervals) -- umožnují odhadnout \textbf{nejistotu} v odhadu parametru náhodné veličiny.
	\item \textbf{Testování hypotéz}(hypothesis testing) -- umožnuje posoudit, zda experimentálně získaná data nepopírají předpoklad, který jsem \textbf{před} provedením testování učinili.
\end{itemize}
\begin{figure}[H]
\centering
\includegraphics[width=0.6\textwidth]{assets/14_metody_stat_ind}
\end{figure}
\subsubsection{Intervalové odhady}
\begin{itemize}
\item V praktických aplikacích často určujeme \textbf{odhad příslušného parametru} pomocí intervalového odhadu.
\item Tento odhad je reprezentován intervalem $<t_D, t_H>$, v němž hledaný parametr leží s předem určenou pravděpodobností (spolehlivostí), kterou označujeme $(1 − \alpha)$.
\item neboli parametr populace aproximujeme intervalem, v němž s velkou pravděpodobností příslušný populační parametr leží.
\end{itemize}

\subsubsection{Interval spolehlivosti (konfidenční interval)}
Interval spolehlivosti (konfidenční interval) pro parametr $\theta$ se spolehlivostí $1−\alpha$, kde $\alpha \in <0; 1>$, je \textbf{taková dvojice statistik} $(T_D, T_H)$, že $P(T_D \leq \theta \leq T_H) = 1 − \alpha$.
\begin{figure}[H]
\centering
\includegraphics[width=0.6\textwidth]{assets/14_inter_odhad_terminologie}
\end{figure}
\begin{itemize}
	\item \textbf{Intervalový odhad} $t_D,t_H$ je jednou \textbf{z realizací intervalu spolehlivosti}.
	\item Požadavky na interval spolehlivosti:
	\begin{itemize}
		\item Co \textbf{největší spolehlivost} odhadu.
		\item Co \textbf{nejmenší šírka} intervalu spolehlivost. (s rostoucí spolehlivostí se zvětšuje šířka intervalového odhadu a tím \textbf{klesá významnost} takto získané informace. S rostoucím rozsahem výběru se šířka intervalového odhadu snižuje.)
	\end{itemize}
\end{itemize}
\subsubsection*{Typy intervalů spolehlivosti}
	\begin{itemize}
		\item \textbf{oboustranné} 
		\begin{equation*}
				P(\theta < T_D) = P(\theta > T_H) = \frac{\alpha}{2}
		\end{equation*}
		Tyto dvě podmínky zaručují, že $P(T_D \leq \theta \leq T_H) = 1 - \alpha$
		\item \textbf{jednostranné} (odhadujeme--li například délku života nějakého zařízení, je pro nás důležitá pouze dolní mez)
		\begin{itemize}
			\item \textbf{levostranné} $P(\theta \geq T_D^*) = 1 - \alpha$
			\item \textbf{pravostranné} $P(\theta \leq T_H^*) = 1 - \alpha$
		\end{itemize}
	\end{itemize}
\begin{figure}[H]
\centering
\includegraphics[width=0.6\textwidth]{assets/14_spolehlivost_odhadu}
\end{figure}
\textbf{Co to znamená}, že spolehlivost odhadu je $1- \alpha$? -- Simulace 100 intervalových odhadů  (obrázek výše) střední hodnoty (spolehlivost $0,95$) získaných na základě opakovaných výběrů o rozsahu 30 z populace se střední hodnotou 100. 6 intervalů ze 100 \textbf{neobsahuje skutečnou střední hodnou}.
\subsection{Jak najít intervalový odhad parametru $\theta$?}
\textbf{Obecně:}
\begin{enumerate}
	\item Zvolíme \textbf{vhodnou výběrovou charakteristiku} $T(\mathbf{X})$, jejíž rozdělení známe.
	\item \begin{equation*}
			\begin{split}
				P(\frac{x_\alpha}{2} \leq T(\mathbf{X}) \leq x_{1 - \frac{alpha}{2}}) &= 1 - \alpha, \\
				P(T(\mathbf{X}) \leq x_{1-\alpha}) &= 1 - \alpha, \\
				P(T(\mathbf{X}) \geq x_\alpha) &= 1 - \alpha.	
			\end{split}
		\end{equation*}
\end{enumerate}

\subsection{Testování hypotéz}
\textbf{Statistická hypotéza} -- předpoklad (tvrzení) o rozdělení náhodné veličiny.
\begin{itemize}
	\item Zdrojem statistických hypotéz jsou například předchozí zkušenosti, teorie, kterou je třeba doložit, požadavky na kvalitu produktu, dohady založené na náhodném pozorování.
	\item \textbf{Příklady} statistických hypotéz:
	\begin{itemize}
		\item Střední životnost žárovek Ed je nižší než výrobcem udávaných 5 let.
		\item Mortalita je u laparoskopických operací nižší než u operací konvenčních.
		\item Průměrné výsledky srovnávacích testů závisí na typu absolvované střední školy.
		\item Pořízený datový soubor je výběrem z populace mající normální rozdělení.
	\end{itemize}
	\item \textbf{Parametrická statistická hypotéza} -- tvrzení ohledně efektu:
	\begin{itemize}
		\item Hypotézy \textbf{o parametru jedné populace} (o střední hodnotě, rozptylu, mediánu, parametru binomického rozdělení,...).
		\item Hypotézy \textbf{o parametrech dvou populací} (srovnávací testy).
		\item Hypotézy \textbf{o parametrech více než dvou populací} (ANOVA, Kruskalův--Wallisův test,...).
	\end{itemize}
	\item \textbf{Neparametrická statistická hypotéza} -- tvrzení o \textbf{jiné vlastnosti} (rozdělení náhodné veličiny) než o jejím parametru (např. hypotézy o typu rozdělení NV, hypotézy o závislosti NV,...)
\end{itemize}
\textbf{Příklad, ověření, zda statistická hypotéza je pravdivá}: Domníváme se, že střední hodnota obsahu cholesterolu v krvu je u české populace 4,7 mmol/l.
	\begin{equation*}
		\begin{split}
			H_0 : \mu = 4.7	 \\
			H_A : \mu \not = 4.7 
		\end{split}
	\end{equation*}
\subsubsection*{Jak tento předpoklad ověřit?}
\begin{itemize}
	\item Zjistíme údaje o obsahu cholesterolu v krvi u 100 náhodně vybraných Čechů.
	\item Průměrný obsah cholesterolu v krvi probandů (tj. jedinců, kteří jsou předmětem zkoumání) byl 5,4 mmol/l.
\end{itemize}
\subsubsection*{Jsou tyto výsledky v souladu s naší hypotézou?}
\begin{itemize}
	\item[$\circ$] I kdyby byla testovaná hypotéza pravdivá, nelze očekávat, že průměrná hodnota pozorovaná ve výběru bude přesně 4,7 mmol/l.
	\item[$\circ$] \textbf{Nulovou hypotézu zamítneme, pokud získané uspořádání výberu bude za předpokladu platnosti nulové hypotézy velmi nepravděpodobné}.
\end{itemize}
Rozhodovací proces, v němž proti sobě stojí nulová a alternativní hypotéza:
\begin{itemize}
	\item \textbf{Nulová hypotéza $\mathbf{H_0}$} -- tvrzení, že efekt je nulový, resp. že neexistuje závislost, že data mají určitý typ rozdělení, ...
	\item \textbf{Alternativní hypotéza $\mathbf{H_A}$ ($\mathbf{H_1}$)} -- tvrzení, popírající hypotézu nulovou (obvykle to, co chceme dokázat).
\end{itemize}

\subsubsection{Klasický přístup při testování hypotéz}
\begin{enumerate}
	\item Formulujeme \textbf{nulovou a alternativní hypotézu}.
	\item Zvolíme tzv. \textbf{testovací statistiku}, tj. výběrovou charakteristiku, jejíž rozdělení závisí na testovaném parametru $\theta$. (Rozdělení testované statistiky za předpokladu platnosti nulové analýzy nazýváme \textbf{nulové rozdělení}.)
	\item Ověříme \textbf{předpoklady testu}! %Hanzi, to musíš zařvat na Edu!
	\item Určíme \textbf{kritický obor} $W^*$, tj. množinu, v níž se, za předpokladu platnosti $H_0$, hodnoty testované statistiky vyskytují s velmi malou pravděpodností.
	\begin{itemize}
		\item Doplňkem k $W^*$ je tzv. \textbf{obor přijetí} $V^*$.
		\item Hranici mezi kritickým oborem a oborem přijetí označujeme jako \textbf{kritická hodnota testu} $t_{krit}$.
	\end{itemize}
	\item Na základě konkrétní realizace výběru určímě \textbf{pozorovanou hodnotu} $X_{OBS}$ testované statistiky. 
	\item Na základě vztahu mezi $X_{OBS}$ a $t_{krit}$ rozhodneme o výsledku testu (\uv{Zamítáme $H_0$.} nebo \uv{Nezamítáme $H_0$.})
\end{enumerate}

\subsubsection{Chyba I. a II. druhu}
\begin{figure}[H]
\centering
\includegraphics[width=0.6\textwidth]{assets/14_chyba_tab}
\end{figure}
Jestliže nulová hypotéza je ve skutečnosti platná a my ji přesto zamítneme, dopouštíme se chyby, označované jako chyba \textbf{I. druhu}. Pravděpodobnost, že k takovémuto pochybení dojde, nazýváme \textbf{hladina významnosti} a označujeme ji $\alpha$. Platí-li nulová hypotéza a my jsme ji nezamítli, rozhodli jsme správně. Pravděpodobnost tohoto rozhodnutí označujeme $1 − \alpha$ a nazýváme ji \textbf{spolehlivost testu}. Správným rozhodnutím je rovněž \textbf{zamítnutí nulové hypotézy v případě, že je platná hypotéza alternativní}. Tohoto rozhodnutí se dopouštíme s pravděpodobností $1 − \beta$, což bývá označováno jako síla testu. \textbf{Chybou II. druhu} je nezamítnutí nulové hypotézy v případě, že je platná hypotéza alternativní. Pravděpodobnost této chyby označujeme $\beta$.
\begin{figure}[H]
\centering
\includegraphics[width=0.6\textwidth]{assets/14_chyba_graf}
\caption{Demonstrace pravděpodobností chyb I. a II. druhu}
\end{figure}
Při testování hypotéz se samozřejmě snažíme postupovat tak, abychom minimalizovali obě chyby, tj. dosáhnout vysoké síly testu (nízkého $\beta$) při co nejnižší hladině významnosti $\alpha$. To však není možné, neboť snížením $\beta$ se zvýší hladina významnosti $\alpha$ a naopak. Proto je třeba najít kompromis mezi požadavky na $\alpha$ a $\beta$.


\subsubsection{Parametrická statistická hypotéza}
\textbf{Jednovýběrové testy}
\begin{itemize}
	\item Test o \textbf{střední hodnotě} (z--test, t--test).
	\item Test o \textbf{rozptylu}.
	\item Test o \textbf{parametru binomického rozdělení}.
	\item Test o \textbf{mediánu} (Wilcoxonův test, Mediánový test).
\end{itemize}
\textbf{Dvouvýběrové testy}
\begin{itemize}
	\item Test o shodě \textbf{dvou středních hodnot} (t--test, Aspinové--Welchův test).
	\item Test o shodě \textbf{rozptylů} (F--test).
	\item Test o shodě \textbf{parametrů dvou binomických rozdělení} (test homogenity dvou binomických rozdělení).
	\item Test o shodě \textbf{mediánů} (Mannův--Whitneyův test).
	\item \textbf{Párové testy} (párový t--test, párový znaménkový test).
\end{itemize}
\textbf{Vícevýběrové testy}
\begin{itemize}
	\item Testy \textbf{shody rozptylů} (Bartletův test, Hartleyův test, Cochranův test, Leveneův test).
	\item \textbf{Analýza rozptylu} (tzv. ANOVA, tj. shody středních hodnot) -- post hoc analýza pro analýzu rozptylu.
	\item Kruskalův--Wallisův test (test \textbf{shody mediánů}) -- post hoc analýza Kruskal--Wallisův test.
\end{itemize}
\textbf{ANOVA}
\begin{itemize}
	\item Test umožnující \textbf{srovnání průměrů více než dvou výběrových souborů}.
	\item Můžeme například zkoumat, zda:
	\begin{itemize}
		\item typ absolvované střední školy ovlivňuje počet bodů dosažených studenty u přijímací zkoušky z matematiky,
		\item použitá medikace ovlivňuje krevní tlak pacientů,
		\item typ použitého hnojiva ovlivňuje výnosy určité plodiny,
		\item pracovní výkon dělníka závisí na umístění stroje, apod.
	\end{itemize}
\end{itemize}

\end{document}
