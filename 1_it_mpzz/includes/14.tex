%Metody statistické indukce. Intervalové odhady. Princip testování hypotéz.

\subsection{Statistická indukce}
Statistická indukce je metoda, která dovoluje stanovit vlastnost celku (\textbf{základního souboru}) na základě pozorování jeho částí (\textbf{náhodného výběru}).
\begin{figure}[H]
\centering
\includegraphics[width=0.6\textwidth]{assets/14_stat_ind}
\end{figure}

\textbf{Základní soubor (populace)}
\begin{itemize}
	\item Je množina všech teoreticky možných objektů (např. jedinců) v uvažované situaci = statistický soubor, který je vymezen cílem výzkumu a pro který vyvozujeme závěry výzkumného šetření.
	\item Charakterizuje se \textbf{parametrem}, což je např. výška, váha, IQ, atp.
	\item Má konečný nebo nekonečný (hypotetický) \textbf{rozsah}, který je dán N (např.: N = 150 lidí, opic, rostlin,...).
\end{itemize}
\textbf{Výběrový soubor (výběr)}
\begin{itemize}
	\item Je část populace vybrané na základě předem stanovených kritérii resp. pravidel (podmnožina základního souboru).
	\item \textbf{O náhodném výběru} uvažujeme, když splňuje dvě základní vlastnosti: pravděpodobnost zařazení do vzorku je pro všechny statistické jednotky populace nenulová a statistické jednotky jsou do vzorku vybrané nezávisle jedna od druhé.
	\item \textbf{O reprezentativním výběru} uvažujeme, když výběrový soubor dobře odráží strukturu celého zkoumaného souboru.
\end{itemize}
\textbf{Principy statistického usuzování}
\begin{enumerate}
	\item Statistické usuzování znamená zobecňování z výběrových statistik na parametry rozdělení.
	\item Abychom mohli provést statistické usuzování, musíme mít nějakou teorii, jež popisuje náhodné chování sledovaných proměnných.
	\item Existují dva typy výběrových chyb: náhodné výběrové chyby a systematické chyby. Získáním náhodného výběru zmenšujeme systematickou chybu a získáváme podklad pro odhad náhodné výběrové chyby.
	\item Výběrová rozdělení statistik jsou teoretická pravděpodobnostní rozdělení, která popisují vztah mezi výběrovou statistikou a populací.
	\item Směrodatná odchylka výběrového rozdělení statistiky (odhad parametru) se nazývá směrodatná chyba. Odhaduje náhodnou výběrovou chybu vypočítané statistiky (odhadu parametru).
	\item Jak roste velikost výběru, výběrová chyba a směrodatná chyba se zmenšují.
	\item Směrodatná chyba se používá k získání intervalového odhadu parametrů i k testování hypotéz o parametrech rozdělení.
\end{enumerate}
\subsection{Základy metody statistické indukce}
\begin{itemize}
	\item \textbf{Intervalové odhady} (confidence intervals) -- umožnují odhadnout nejistotu v odhadu parametru náhodné veličiny 
	\item \textbf{Testování hypotéz}(hypothesis testing) -- umožnuje posoudit, zda experimentálně získaná data nepopírají předpoklad, který jsem \textbf{před} provedením testování učinili.
\end{itemize}
\begin{figure}[H]
\centering
\includegraphics[width=0.6\textwidth]{assets/14_metody_stat_ind}
\end{figure}
\subsubsection{Intervalové odhady}
\begin{itemize}
\item V praktických aplikacích často určujeme \textbf{odhad příslušného parametru} pomocí intervalového odhadu.
\item Tento odhad je reprezentován intervalem $<t_D, t_H>$, v němž hledaný parametr leží s předem určenou pravděpodobností (spolehlivostí), kterou označujeme $(1 − \alpha)$.
\item neboli parametr populace aproximujeme intervalem, v němž s velkou pravděpodobností příslušný populační parametr leží.
\end{itemize}

\begin{itemize}
	\item \textbf{Interval spolehlivosti}(konfidenční interval) pro parametr $\theta$ se spolehlivostí $1−\alpha$,kde $\alpha \in <0; 1>$ , je taková dvojice statistik $(T_D, T_H)$, že $P(T_D \leq \theta \leq T_H) = 1 − \alpha$.
\end{itemize}
\begin{figure}[H]
\centering
\includegraphics[width=0.6\textwidth]{assets/14_inter_odhad_terminologie}
\end{figure}
\begin{itemize}
	\item \textbf{Intervalový odhad} $t_D,t_H$ je jednou z realizací intervalu spolehlivosti.
	\item Požadavky na interval spolehlivosti:
	\begin{itemize}
		\item Co \textbf{největší spolehlivost} odhadu.
		\item Co \textbf{nejmenší šírka} intervalu spolehlivost. (S rostoucí šířkou intervalového odhadu klesá významnost získané informace.)
	\end{itemize}
	\item S rostoucí spolehlivostí se zvětšuje šířka intervalového odhadu a tím klesá významnost takto získané informace.
	\item S rostoucím rozsahem výběru se šíčka intervalového odhadu snižuje.
	\item Typy intervalů spolehlivosti:
	\begin{itemize}
		\item \textbf{oboustranné} 
		\begin{equation*}
				P(\theta < T_D) = P(\theta > T_H) = \frac{\alpha}{2}
		\end{equation*}
		Tyto dvě podmínky zaručují, že $P(T_D \leq \theta \leq T_H) = 1 - \alpha$
		\item \textbf{jednostranné} (odhadujeme--li například délku života nějakého zařízení, je pro nás důležitá pouze dolní mez)
		\begin{itemize}
			\item \textbf{levostranné} $P(\theta \geq T_D^*) = 1 - \alpha$
			\item \textbf{pravostranné} $P(\theta \leq T_H^*) = 1 - \alpha$
		\end{itemize}
	\end{itemize}
\end{itemize}
\begin{figure}[H]
\centering
\includegraphics[width=0.6\textwidth]{assets/14_spolehlivost_odhadu}
\caption{Co to znamená, že spolehlivost odhadu je $1- \alpha$? \\Simulace 100 intervalových odhadů střední hodnoty (spolehlivost $0,95$) získaných na základě opakovaných výběrů o rozsahu 30 z populace se střední hodnotou 100. 6 intervalů ze 100 neobsahuje skutečnou střední hodnou}
\end{figure}
\subsection{Jak najít intervalový odhad parametru $\theta$?}
\textbf{Obecně:}
\begin{enumerate}
	\item Zvolíme vhodnou výběrovou charakteristiku $T(\mathbf{X})$, jejíž rozdělení známe.
	\item \begin{equation*}
			\begin{split}
				P(\frac{x_\alpha}{2} \leq T(\mathbf{X}) \leq x_{1 - \frac{alpha}{2}}) = 1 - \alpha, \\
				P(T(\mathbf{X}) \leq x_{1-\alpha}) = 1 - \alpha, \\
				P(T(\mathbf{X}) \geq x_\alpha) = 1 - \alpha.	
			\end{split}
		\end{equation*}
\end{enumerate}

\subsection{Testování hypotéz}
\begin{itemize}
	\item \textbf{Statistická hypotéza} -- předpoklad (tvrzení) o rozdělení náhodné veličiny
	\item Zdrojem statistických hypotéz jsou například předchozí zkušenosti, teorie, kterou je třeba doložit, požadavky na kvalitu produktu, dohady založené na náhodném pozorování.
	\item Příklady statistických hypotéz:
	\begin{itemize}
		\item Střední životnost žárovek Ed je nižší než výrobcem udávaných 5 let.
		\item Mortalita je u laparoskopických operací nižší než u operací konvenčních.
		\item Průměrné výsledky srovnávacích testů závisí na typu absolvované střední školy.
		\item Pořízený datový soubor je výběrem z populace mající normální rozdělení.
	\end{itemize}
	\item \textbf{Parametrická statistická hypotéza} -- tvrzení ohledně efektu
	\begin{itemize}
		\item Hypotézy o parametru jedné populace (o střední hodnotě, rozptylu, mediánu, parametru binomického rozdělení,...)
		\item Hypotézy o parametrech dvou populací (srovnávací testy)
		\item Hypotézy o parametrech více než dvou populací (ANOVA, Kruskalův--Wallisův test,...)
	\end{itemize}
	\item \textbf{Neparametrická statistická hypotéza} -- tvrzení o jiné vlastnosti rozdělení náhodné veličiny než o jejím parametru (např. hypotézy o typu rozdělení NV, hypotézy o závislosti NV,...)
\end{itemize}
\textbf{Příklad, ověření, zda statistická hypotéza je pravdivá}: Domníváme se, že střední hodnota obsahu cholesterolu v krvu je u české populace 4,7 mmol/l.
	\begin{equation*}
		\begin{split}
			H_0 : \mu = 4.7	 \\
			H_A : \mu \not = 4.7 
		\end{split}
	\end{equation*}
\subsubsection*{Jak tento předpoklad ověřit?}
\begin{itemize}
	\item Zjistíme údaje o obsahu cholesterolu v krvi u 100 náhodně vybraných Čechů.
	\item Průměrný obsah cholesterolu v krvi probandů (tj. jedinců, kteří jsou předmětem zkoumání) byl 5,4 mmol/l.
\end{itemize}
\subsubsection*{Jsou tyto výsledky v souladu s naší hypotézou?}
\begin{itemize}
	\item[$\circ$] I kdyby byla testovaná hypotéza pravdivá, nelze očekávat, že průměrná hodnota pozorovaná ve výběru bude přesně 4,7 mmol/l.
	\item[$\circ$] \textbf{Nulovou hypotézu zamítneme, pokud získané uspořádání výberu bude za předpokladu platnosti nulové hypotézy velmi nepravděpodobné}.
\end{itemize}
\begin{itemize}
	\item Rozhodovací proces, v němž proti sobě stojí nulová a alternativní hypotéza.
	\item \textbf{Nulová hypotéza $\mathbf{H_0}$} -- tvrzení, že efekt je nulový, resp. že neexistuje závislost, že data mají určitý typ rozdělení,...
	\item \textbf{Alternativní hypotéza $\mathbf{H_A}$ ($\mathbf{H_1}$)} -- tvrzení, popírající hypotézu nulovou (obvykle to, co chceme dokázat)
\end{itemize}

\subsubsection{Klasický přístup při testování hypotéz}
\begin{enumerate}
	\item Formulujeme \textbf{nulovou a alternativní hypotézu}.
	\item Zvolíme tzv. \textbf{testovací statistiku}, tj. výběrovou charakteristiku, jejíž rozdělení závisí na testovaném parametru $\theta$. (Rozdělení testované statistiky za předpokladu platnosti nulové analýzy nazýváme \textbf{nulové rozdělení}.)
	\item Ověříme předpoklady testu ! %Hanzi, to musíš zařvat na Edu!
	\item Určíme \textbf{kritický obor} $W^*$, tj. množinu, v níž se, za předpokladu platnosti $H_0$, hodnoty testované statistiky vyskytují s velmi malou pravděpodností.
	\begin{itemize}
		\item Doplňkem k $W^*$ je tzv. \textbf{obor přijetí} $V^*$.
		\item Hranici mezi kritickým oborem a oborem přijetí označujeme jako \textbf{kritická hodnota testu} $t_{krit}$.
	\end{itemize}
	\item Na základě konkrétní realizace výběru určímě \textbf{pozorovanou hodnotu} $X_{OBS}$ testované statistiky. 
	\item Na základě vztahu mezi $X_{OBS}$ a $t_{krit}$ rozhodneme o výsledku testu (\uv{Zamítáme $H_0$.} nebo \uv{Nezamítáme $H_0$.})
\end{enumerate}

\subsubsection*{Parametrická statistická hypotéza}
Jednovýběrové testy
\begin{itemize}
	\item Test o střední hodnotě (z--test, t--test).
	\item Test o rozptylu.
	\item Test o parametru binomického rozdělení.
	\item Test o mediánu (Wilcoxonův test, Mediánový test).
\end{itemize}
Dvouvýběrové testy
\begin{itemize}
	\item Test o shodě dvou středních hodnot (t--test, Aspinové--Welchův test).
	\item Test o shodě rozptylů (F--test).
	\item Test o shodě parametrů dvou binomických rozdělení (test homogenity dvou binomických rozdělení).
	\item Test o shodě mediánů (Mannův--Whitneyův test).
	\item Párové testy (párový t--test, párový znaménkový test).
\end{itemize}
Vícevýběrové testy
\begin{itemize}
	\item Testy shody rozptylů (Bartletův test, Hartleyův test, Cochranův test, Leveneův test).
	\item Analýza rozptylu (tzv. ANOVA, tj. shody středních hodnot) -- post hoc analýza pro analýzu rozptylu
	\item Kruskalův--Wallisův test (test shody mediánů) -- post hoc analýza Kruskal--Wallisův test
\end{itemize}
ANOVA
\begin{itemize}
	\item Test umožnující srovnání průměrů více než dvou výběrových souborů
	\item Můžeme například zkoumat, zda
	\begin{itemize}
		\item typ absolvované střední školy ovlivňuje počet bodů dosažených studenty u přijímací zkoušky z matematiky,
		\item použitá medikace ovlivňuje krevní tlak pacientů,
		\item typ použitého hnojiva ovlivňuje výnosy určité plodiny,
		\item pracovní výkon dělníka závisí na umístění stroje, apod.
	\end{itemize}
\end{itemize}