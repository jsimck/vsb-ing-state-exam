\documentclass[11pt]{article}

% Packages
\usepackage[czech]{babel}
\usepackage[utf8]{inputenc}
\usepackage[useregional]{datetime2}
\usepackage[T1]{fontenc}
\usepackage[a4paper, total={15.24cm, 23.32cm}]{geometry}
\usepackage[thinlines]{easytable}
\usepackage{graphicx}
\usepackage[ampersand]{easylist}
\usepackage{changepage}
\usepackage{float}
\usepackage{color}
\usepackage{enumitem}

% Config
\setlength\parindent{0pt}
\renewcommand{\baselinestretch}{1.2} 
\setitemize{itemsep=0pt}

\title{\textbf{IV. Počítače a sítě}}
\date{\small\vspace{-9ex}Update: \today}

\begin{document}
\maketitle

\section{Architektura univerzálních procesorů. Principy urychlování činnosti procesorů.}

\pagebreak
\section{Základní vlastnosti monolitických počítačů a jejich typické integrované periférie. Možnosti použití.}

\pagebreak
\section{Struktura OS a jeho návaznost na technické vybavení počítače.}

\pagebreak
\section{Protokolová rodina TCP/IP.}

\pagebreak
\section{Metody sdíleného přístupu ke společnému kanálu.}

\pagebreak
\section{Problémy směrování v počítačových sítích. Adresování v IP, překlad adres (NAT).}

\pagebreak
\section{Bezpečnost počítačových sítí s TCP/IP: útoky, paketové filtry, stavový firewall. Šifrování a autentizace, virtuální privátní sítě.}

\end{document}
