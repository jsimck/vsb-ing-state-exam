
\subsection{Osvětlovací modely}

\subsection*{Phongův osvětlovací model} % https://www.youtube.com/watch?v=L4oAQL_Mv5w
\begin{itemize}
	\item $I = I_a + I_d + I_s$ 
	\item Phongův osvětlovací model je empirický (na fyzice nezaložený) osvětlovací model pro výpočet osvětlení povrchu nějakého objektu.
	\item Je vytvořen tak, aby dával dostatečně korektní výsledky v rozumném čase.
	\item Díky tomu se zpravidla používá pro real-time zobrazování.
	\item Vytváří na povrchu odlesky Výpočet je:
	\begin{equation*} 
			\begin{array}{c}
				I_p = k_ai_a + \sum\limits_{m \in lights} (k_d(\vec{L_m} \cdot \vec{N})i_{m,d} + k_s(\vec{R_m} \cdot \vec{V})^\alpha i_{m,s}). \\
			\end{array}
			\end{equation*}
		\begin{itemize}
			\item kde: \textbf{a} -- ambientní, \textbf{d} -- difůzní, \textbf{s} -- spekulátní,
			\item 	$lights$ -- je soubor všech světelných zdrojů,
			\item 	$k$ -- parametry materiálu,
			\item 	$i$ -- složka světla,
			\item 	$\alpha$ -- shininess -- udává ostrost spekulárního odrazu (velikost specular highlightu),
			\item 	$L$ -- směr ke světlu,
			\item 	$N$ -- normála v daném bodě,
			\item 	$R$ -- směr dokonale odrazené svěla v daném bodě,
			\item 	$V$ -- směr z bodu k pozorovateli (kameře).
			\item Směr odraženého paprsku určuje vztah $\vec{R_m} = 2(\vec{L_m} \cdot \vec{N})\vec{N} - \vec{L_m}$.
		\end{itemize}
\end{itemize}

\subsection*{Blinn-Phong osvětlovací model}
\begin{itemize}
	\item Výchozí osvětlovací model pro standardní zobrazovací řetězec v OpenGL/DirectX. Zavádí tzv. half vector.
	\item $ H = \frac{L + V}{| L + V |}$
	\item Místo $R \cdot V$  pak použijeme při výpočtu $N \cdot H$.
	\item Blinn-Phong \textbf{je pomalejší} než původní Phong, protože při výpočtu half vectoru je nutné provést odmocninu (při normalizaci).
	\item Jelikož je odmocnina \textbf{hardwarově implementovaná}, rozdíl je zanedbatelný.
	\item Je však \textbf{rychlejší}, když se pozorovatel a světlo předpokládá v nekonečnu -- směrová světla.
	\item V tom případě je half vector \textbf{nezávislý na poloze a zakřivení povrchu}, stačí jej vypočítat pouze jednou pro každé světlo.
\end{itemize}

\subsection{Systémy barev v PG}
\textbf{Barevný prostor}:
\begin{itemize}
	\item Základem barevného prostoru je \textbf{barevný model}, který nám dává abstraktní matematický popis, jak lze barvy vyjádřit pomocí n-tic čísel, nejčastěji trojic. 
	\item Mezi nejznámější barevné modely v dnešní době patří \textbf{RGB model}. 
	\item Model RGB pracuje se třemi základními barvami: \textbf{červenou, zelenou} a \textbf{modrou}, z nichž se odvíjí i jeho název. 
	\item Tyto barvy byly zvoleny na základně toho, jak \textbf{čípky v lidském oku} vnímají jednotlivé záření. 
	\item Zároveň je RGB \textbf{aditivní barevný model}, což znamená, že se jednotlivé barevné složky \textbf{míchají} a výsledkem jsou další barevné odstíny, případně vyšší intenzita barvy
	\item Když k tomuto modelu definujeme, jak mají být tyto n-tice interpretovány, dostáváme barevný prostor. 
	\item Barevný prostor je tedy \textbf{definován rozsahem barev}, které dokáže zobrazit. 
	\item Tomuto rozsahu se také říká \textbf{gamut}. Ten se zpravidla zobrazuje jako oblast v CIE 1931 chromatickém diagramu 
\end{itemize}
\begin{figure}[H]
\centering
\includegraphics[width=0.3\textwidth]{assets/1_rgb_gamut}
\end{figure}
\subsubsection{RGB}
\begin{itemize}
	\item Nejrozšířenější barevný prostor postavený na RGB barevném modelu je \textbf{sRGB} - standardní RGB. 
	\item Jeho určení je pro zobrazování \textbf{na monitorech} nebo \textbf{kódování barev} na internetu. 
	\item Pro všechny tři barevné složky má definovány barvy v \textbf{chromatickém diagramu}, které vymezují jeho gamut. 
	\item Každá barva, kterou tento prostor zobrazuje, je dána zastoupením jednotlivých barevných složek, buďto relativně (hodnoty jsou v rozmezí 0 - 1) nebo absolutně (konkrétní \uv{bitové} hodnoty, zpravidla 0 - 255).
	\item RGB je možné zobrazit jako krychli.
	\item Často se přidává \textbf{Alpha kanál} pro průhlednost - \textbf{RGBA}.
	\begin{figure}[H]
	\centering
	\includegraphics[width=0.6\textwidth]{assets/1_rgb_gamut_krychle}
	\end{figure}
\end{itemize}

\subsubsection{HSV a HSL}
\begin{itemize}
	\item \textbf{Hue, Saturation, Value/Lightness} - barevný model, který nejvíce odpovídá lidskému vnímání barev.
	\item Barvy popisuje pomocí 3 hodnot, které však samy barvy nereprezentují:
	\begin{itemize}
		\item \textbf{Hue} - \textbf{barevný tón}, převládající. Neboli odstín - barva \textbf{odražená} nebo \textbf{procházející} objektem. Měří se jako poloha na standardním barevném kole (\ang{0} až \ang{360}). Obecně se odstín označuje názvem barvy. \ang{0} - červená, \ang{120} - zelená, \ang{240} - modrá.
		\item \textbf{Saturation} - \textbf{sytost} barvy, příměs jiné barvy. Někdy též chroma, síla nebo čistota barvy, představuje množství šedi v poměru k odstínu, měří se v procentech od 0\% (šedá) do 100\% (plně sytá barva). Na barevném kole vzrůstá sytost od středu k okrajům.
		\item \textbf{Value} - \textbf{hodnota jasu}, množství bílého světla. Relativní světlost nebo tmavost barvy. Jas vyjadřuje \textbf{kolik světla barva odráží}, dalo by se také říct přidávání černé do základní barvy.
	\end{itemize}
	\item Nejčastěji se tato reprezentace (popř. \textbf{HSL}) používají v grafických nástrojích jako komponenty pro výběr barvy, protože je mnohem intuitivnější než RGB. 
	\item Vyberu si odstín, jak má být sytý a jasný a hotovo. Není třeba řešit jak smíchat 3 barevné složky, abych dostal to co chci.
	\begin{figure}[H]
	\centering
	\includegraphics[width=0.6\textwidth]{assets/1_hsv_hsl}
	\end{figure}
\end{itemize}

\subsubsection{CMY a CMYK}
\begin{itemize}
	\item Substraktivní barevné systémy, \textbf{C}yan, \textbf{M}agenta, \textbf{Y}ellow a \textbf{K}ey (Blac\textbf{K}). 
	\item Barvy se \textbf{odečítají od bílé}.
	\item Používají se \textbf{pro tisk}.
	\item Černá se přidala, protože smíchání CMY nedává plně černou barvu, navíc je černý inkoust levnější než barevné.
	\item Nevýhodou je, že \textbf{nedokáže správně zobrazit} sytě červenou, zelenou a modrou.
	\item Při tisku to však není poznat.
	\item Před tiskem se RGB obraz převádí do CMYK.
	\item To provádí buďto ovladač tiskárny nebo RIP (Raster Image Processor - u profi tiskáren).
	\item RGB se používá pro aktivní zdroje světla, CMYK jsou pasivní (světlo pouze odrážejí), proto nedokáží udělat tak jasné odstíny.
\end{itemize}
	\begin{figure}[H]
	\centering
	\includegraphics[width=0.6\textwidth]{assets/1_cmyk}
	\end{figure}
%Osvětlovací modely a systémy barev v počítačové grafice.