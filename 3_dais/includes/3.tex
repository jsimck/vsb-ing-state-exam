\subsection{Transakce}
\textbf{Logická (nedělitelná, atomická) jednotka práce s databází}, která musí proběhnout buď celá, nebo (v případě že je přerušena) obnovit původní stav databáze a spustit se znovu. Začíná operací \textbf{BEGIN TRANSACTION} a končí provedením operací \textbf{COMMIT} nebo \textbf{ROLLBACK}.
\begin{itemize}
\item Obecně zahrnuje posloupnost operací.
\item Jejím úkolem je převést \textbf{korektní stav databáze} na jiný korektní stav.
\item O řízení se stará \textbf{manager transakcí} nebo \textbf{monitor transakčního zpracování}.
\item Operace transakce jsou nejprve zaznamenávány do\textbf{ logu}.
\item Transakce nemohou být vnořovány.
\item Všechny SQL příkazy v transakci jsou atomické
\item Npoužití transakcí může dojít k nekonzistenci databáze
\end{itemize}

\subsubsection{COMMIT}
\begin{itemize}
	\item Transakce doběhla úspěšně a změny mohou být \textbf{trvale uloženy}, zámky a adresace uvolněny (kromě WITH HOLD).
	\item Zavádí \textbf{potvrzovací bod}.
	\item Odpovídá úspěšnému ukončení logické jednotky práce a \textbf{označuje korektní stav DB}.
\end{itemize}
\subsubsection{ROLLBACK}
\begin{itemize}
	\item Označuje, že databáze může být v \textbf{nekorektním stavu} a všechny změny transakce musí být \textbf{zrušeny}.
\end{itemize}

\subsubsection{SAVEPOINT}
\begin{itemize}
	\item Rozdělení transakcí na menší části
	\item \texttt{ROLLBACK} lze provést pouze částečně, pouze do předem vytvořeného \texttt{SAVEPOINT}u, co bylo před zůstane zachováno. Tím není zrušená celá transakce, ale může klidně pokračovat i nadále dalšími SQL příkazy až do závěrečného \texttt{COMMIT}u.
	\item Po ukončení transakce je savepoint zahozen
\end{itemize}

\subsubsection{ACID}
Každá transakce by měla splňovat následující vlastnosti:
\begin{itemize}
\item\textbf{Atomičnost (Atomicity)} -- transakce musí být atomická: jsou provedeny všechny operace transakce nebo žádná.
\item\textbf{Korektnost (Correctness)} – transakce převádí korektní stav databáze do jiného korektního stavu databáze, mezi začátkem a koncem transakce nemusí být databáze v korektním stavu.
\item\textbf{Izolovanost (Isolation)} – transakce jsou navzájem izolovány: změny provedené jednou transakcí jsou pro ostatní transakce viditelné až po provedení COMMIT. 
\item\textbf{Trvalost (Durability)} – jakmile je transakce potvrzena, změny v databázi se stávají trvalými i po případném pádu systému.
\end{itemize}

\subsection{Zotavení}
\begin{itemize}
\item Nastává po \textbf{chybě SŘBD} => Zotavení databáze z nějaké chyby.
\item Výsledkem musí být \textbf{korektní stav DB}.
\item Využívají se\textbf{ skryté redundantní} informace.
\item Jednotkou zotavení je \textbf{transakce}.
\item Všechny změny jsou zapisovány do logu před zápisem změn do DB => \textbf{pravidlo dopředného zápisu do logu}.
\item Do logu se zapisuje \textbf{sekvenčně}, proto poskytuje \textbf{vyšší výkon} než přímý zápis dat.
\end{itemize}

\subsubsection{Chyby zotavení}
\begin{itemize}
\item\textbf{Lokální} - pouze v rámci jedné transakce (chyba v dotazu, přetečení hodnoty atributu). Vyskytuje se \textbf{10-100x/min} a čas pro zotavení je shodný, jako čas provedení transakce
\item\textbf{Globální} - ovlivňují více transakcí najednou:
\begin{itemize}
\item\textbf{Systémové (soft crash)} (výpadek proudu, pád systému). Může se vyskytovat i několikrát do roka. Čas potřebný k obnově je několik minut.
\item\textbf{Chyby média (hard crash)} - chyba disku (zotavení probíhá ze záložní kopie a z logu jsou obnoveny potvrzené transakce po vytvoření zálohy). Vyskytuje se zřídka a obnova může trvat i hodiny.
\end{itemize}
\end{itemize}

\subsubsection{Průběh zotavení}
Základním problémem vzniklým při systémové chybě je ztráta obsahu hlavní paměti, tedy ztráta obsahu vyrovnávací paměti SŘBD. Přesný stav transakce přerušené chybou není znám a transakce musí být \textbf{zrušena} (\textbf{UNDO}). Někdy je transakce úspěšně dokončena, ovšem změny, nejsou přeneseny z vyrovnávací paměti na disk. V tomto případě musí být transakce po restartu systému přepracována (\textbf{REDO}). \textbf{Typy zotavení}: 

\subsubsection*{Odloženou aktualizací (deferred update, NO-UNDO/REDO)}
\begin{itemize}
\item Neprovádí aktualizace databáze na disk dokud transakce nedosáhne\textbf{ potvrzovacího bodu}. Všechny změny jsou v paměťovém bufferu.
\item Jakmile transakce dosáhne potvrzovacího bodu, tak se \textbf{nejprve vše zapíše do REDO logu} a pak do DB (\textbf{pravidlo dopředného zápisu do logu}).
\item Při selhání transakce není nutné provádět \textbf{undo}, změny jsou ztraceny spolu s vyrovnávací pamětí.
\item \textbf{Redo} se provádí při chybě během zápisu do DB.
\item Do logu jsou v případě odložené aktualizace zapsány nové hodnoty (kvůli REDO).
\item Minimální I/O operace, používá se pouze pro \textbf{krátké} a \textbf{nenáročné transakce} - \textbf{hrozí přetečení bufferu}.
\end{itemize}
\subsubsection*{Okamžitou aktualizací (immediate update, UNDO/NO-REDO)}
\begin{itemize}
\item \textbf{Provádí aktualizaci DB} než transakce dosáhne potvrzovacího bodu.
\item Operace jsou zapsány do UNDO logu a zároveň je aktualizována DB (pravidlo dopředného zápisu do logu).
\item Při chybě je nutné provést \textbf{undo}, protože \textbf{došlo k aktualizaci DB}.
\item Do logu se zapisují \textbf{původní hodnoty}, což umožní systému provést UNDO.
\item \textbf{Velká zátěž disku / nízký výkon}.
\end{itemize}
\subsubsection*{Kombinovanou aktualizací (UNDO/REDO)}
\begin{itemize}
\item Používaná v praxi. Využívá obou operací v kombinaci s technikou kontrolních bodů.
\item Nezapisuje všechny potvrzené operace na disk, místo toho vytváří \textbf{kontrolní body}.
\begin{itemize}
\item Zápis operací hromadně po určitém počtu záznamů.
\item Zapisuje se obsah vyrovnávací paměti na disk a záznam o kontrolním bodu do logu.
\end{itemize}
\item Po restartu systému se provádí: \textbf{undo} na všechny transakce, které se \textbf{nestihly potvrdit} a \textbf{redo} na všechny transakce, které \textbf{se potvrdily} po vytvoření kontrolního bodu.
\end{itemize}

\subsection{Kontrolní body}
Kontrolní body \textbf{jsou vytvářeny např. po určitém počtu záznamů}, které byly zapsány do logu a zahrnují:
\begin{itemize}
\item zápis obsahu vyrovnávací paměti na disk,
\item zápis záznamu o kontrolním bodu do logu.
\end{itemize}

\noindent V případě následující situace musí být:
\begin{figure}[H]
\centering
\includegraphics[width=0.6\textwidth]{assets/kontrolni_body.png}
\end{figure}

\begin{itemize}
\item Po restartu systému musí být transakce typu $ T_3 $ a $ T_5 $ zrušeny (undo).
\item Transakce typu $T_2$ a $T_4$ musí být přepracovány (redo).
\item Jelikož změny provedené transakcí $ T_1 $ byly provedeny p kontrolním bodem tc, tuto transakci při zotavení vůbec neuvažujeme.
\end{itemize}

Postup zotavení systému:
\begin{enumerate}
	\item Vytvoří se seznamy UNDO a REDO
	\item UNDO se naplní všemi neuloženými transakcemi (vše kromě T1 na obrázku výše)
	\item Procházíme všechny transakce - je v logu COMMIT pro danou transakci -> přesuň transakci do REDO
	\item Všechny transakce z UNDO jsou postupně zrušeny
	\item Všechny transakce z REDO jsou přepracovány a uloženy
	\item Nyní je systém použitelný
\end{enumerate}

\subsection{Problémy souběhu}
\begin{itemize}
\item Pro víceuživatelský DB systém (kolik současně). Pro jednouživatelský přístup (SQLite) se toto vůbec neřeší.
\item Souběh umožňuje SŘBD \textbf{zpřístupnit databázi mnoha transakcím ve stejném čase}.
\item Souběh také přináší mnoho \textbf{problémů}, které je nutné řešit i na \textbf{aplikační} úrovni.
\end{itemize}

\subsubsection{Plán provádění transakce a anomálie}
Plán provádění transakce = posloupnost operací transakce, při souběžném provedení - \textbf{plán souběžný/paralelní}. Vznikají \textbf{3 problémy:}
\begin{itemize}
\item \textbf{Problém ztráty aktualizace} -- jedna transakce \textbf{přepíše právě prováděnou hodnotu}. Časová posloupnost: \texttt{read\_A}, \texttt{read\_B}, \texttt{write\_A}, \texttt{write\_B}.
\\\\
\noindent\makebox[\textwidth]{\includegraphics[width=10cm]{assets/soubeha3}}
\item \textbf{Problém nepotvrzené závislosti} -- 
	\begin{itemize}
		\item \textbf{Scénař 1} - tr. A pracuje se špatnými daty - Transakce B zapíše \uv{X}, transakce A přečte \uv{X}, transakce B provede ROLLBACK}
		\item \textbf{Scénař 2} - změna tr. A je ztracena - Transakce B zapíše \uv{X}, transakce A zapíše \uv{X}, transakce B provede ROLLBACK}
	\end{itemize}
\\
\noindent\makebox[\textwidth]{\includegraphics[width=10cm]{assets/soubeha2}}
\item \textbf{Problém nekonzistentní analýzy} -- A provádí součet na účtech, před dokončením B provede přesun z účtu na účet, přičemž 1 už byl započítán a druhý ne. \textbf{Špatný součet zůstatků!} A čte commited data (B provede commit, než si A vyžádá další účet), ale i tak to není správné.
\\\\
\noindent\makebox[\textwidth]{\includegraphics[width=10cm]{assets/soubeha1}}
\end{itemize}

\subsection{Konflikty čtení/zápis}
A a B chtějí \textbf{číst/zapisovat stejnou entici} (záznam). Nastávají 4 možnosti konfliktu:
\begin{itemize}
\item \textbf{RR (READ-READ)} -- negativně se neovlivní, není problém.
\item \textbf{RW (READ-WRITE)} -- \texttt{read\_A}, \texttt{write\_B} = A dále počítá s daty $\rightarrow$ RW zapříčiňuje \textbf{problém nekonzistentní analýzy}. \texttt{read\_A}, \texttt{write\_B}, \texttt{read\_A} $\rightarrow$ A načte odlišené hodnoty = \textbf{neopakovatelné čtení (non repeatable read)}
\item \textbf{WR (WRITE-READ)} -- \texttt{write\_A}, \texttt{read\_B}, \texttt{rollback\_A}? $\rightarrow$ \textbf{Problém nepotvrzené závislosti}. Pokud B přečte data $\rightarrow$ \textbf{Špinavé čtení (dirty read)} = čtení non-commited dat.
\item \textbf{WW (WRITE-WRITE)} -- \texttt{write\_A}, \texttt{write\_B}, \texttt{rollback\_A}? $\rightarrow$ \textbf{Ztráta aktualizace} (pro A) a \textbf{nepotvrzená závislost} pro B. \textbf{Špinavý zápis}\textbf{ (dirty write)} - přepisování non-commited dat.
\end{itemize}

\subsection{Techniky řízení souběhu}
\subsubsection{Správa verzí - optimistický přístup}
Předpoklad, že se paralelní \textbf{transakce ovlivňovat nebudou}. Systém \textbf{vytváří} při aktualizaci\textbf{ kopie dat }a sleduje, která z verzí má být viditelná pro ostatní transakce (podle úrovně izolace).

\subsubsection{Zamykání - pesimistický přístup}
Předpokládáme, že se paralelní \textbf{transakce budou ovlivňovat}. Systém spravuje jednu kopii dat a jednotlivým transakcím přiděluje \textbf{zámky}. Používá se nejčastěji

\begin{itemize}
\item Chce-li transakce A provést čtení/zápis nějakého objektu v DB (nejčastěji n-tice), \textbf{požádá o zámek} na tento objekt. Žádná jiná paralelní transakce zámek získat nemůže, dokud jej A \textbf{neuvolní}.
\item \textbf{2 typy zámků (existuje jich i více):} 
\begin{itemize}
\item \textbf{výlučný} zámek (exclusive lock / write lock) \textbf{X}.
\item \textbf{sdílený} zámek (shared lock / read lock) \textbf{S}.
\end{itemize}
\item A má zámek X a B \textbf{nedostane žádný zámek} hned. A má zámek S, B \textbf{může hned dostat S}, X nikoliv.
\item \textbf{Matice kompability} - vzájemné vztahy typů zámků, sloupce a řádky: X, S, -; A (okamžitě), N (ne)
\begin{table}[H]
	\centering
	\begin{tabular}{|l|l|l|l|}
		\hline
		& X & S & - \\ \hline
		X & N & N & A \\ \hline
		S & N & A & A \\ \hline
		- & A & A & A \\ \hline
	\end{tabular}
\end{table}
\item \textbf{Operace aktualizace} - mění obsah DB - \texttt{UPDATE}, \texttt{INSERT} i \texttt{DELETE}.
\item \textbf{Uzamykací protokol} - většinou žádání zámků implicitně $\rightarrow$ při \textbf{získání} n-tice z DB žádán \textbf{zámek S}. Při aktualizaci \textbf{zámek X}; žádá-li zámek X a má už S, je mu \textbf{S změněn na X}; když nemůže být zámek přidělen okamžitě, transakce přechází do stavu \textbf{čekání} (wait state).
\item Systém musí zajistit aby v tomto stavu nesetrvala navždy - situace "\textbf{livelock}" nebo "\textbf{starvation}" $\rightarrow$ řadit požadavky do \textbf{fronty} (FIFO). Zámky uvolněny až po operaci \texttt{COMMIT} nebo \texttt{ROLLBACK}.
\item \textbf{Explicitní uzamykání} - \texttt{LOCK TABLE <names> IN [ROW SHARE|ROW EXCLUSIVE|SHARE UPDATE|SHARE|SHARE ROW EXCLUSIVE|EXCLUSIVE] MODE [NOWAIT]}
\end{itemize}

\subsection{Uváznutí}
\begin{itemize}
\item \textbf{Deadlock} - dvě nebo více transakcí jsou ve stavu \textbf{čekání} na uvolnění zámků držených jinou transakcí.
\item \textbf{Detekce uváznutí} - \textbf{časové limity} (nastavení max. času pro vykonání transakce), \textbf{detekce cyklu v grafu Wait-For} (zaznamenává, které transakce na sebe čekají $\rightarrow$ u jedné provede \texttt{ROLLBACK}).
\item \textbf{Prevence uváznutí pomocí časových razítek} - 2 verze uzamykacího protokolu. Každá transakce na začátku dostane časové razítko (unikátní). Pokud A požaduje zámek na entici, která je zamčená B pak:
\begin{itemize}
\item při \textbf{Wait-Die} - pokud je A starší než B, A přejde na čekání; je-li mladší, A je zrušena \texttt{ROLLBACK} a spuštěna znovu.
\item při \textbf{Wound-Die} - pokud A je mladší než B, A přejde na čekání; starší $\rightarrow$ B zrušena \texttt{ROLLBACK} a spuštěna znovu.
\end{itemize}
\item Při opětovném spuštění si transakce nechá své časové razítko. \textbf{Nevýhodou} je velký počet operací \texttt{ROLLBACK}. První část jména - situace kdy A je starší než B. \textbf{Nemůže nikdy dojít k uváznutí.}
\end{itemize}

\subsection{Sériový a serializovatelný plán}
\begin{itemize}
\item \textbf{Ekvivalentní plán} - 2 plány jsou ekvivalentní, pokud dávají shodné výsledky
\item \textbf{Sériový plán} - n-tice uspořádaná dle \textbf{pořadí vykonávání} jednotlivých transakcí. (transakce jsou provedeny zasebou).
\item \textbf{Serializovatelný plán} - \textbf{plán vykonávání dvou transakcí} je korektní jen tehdy, pokud je serializovatelný $\rightarrow$ plán ekvivalentní s výsledkem libovolného sériového plánu.
\end{itemize}

\subsubsection{Dvoufázové uzamykání}
Transakce které dodržují protokol dvoufázového uzamykání jsou vždy serializovatelné
\begin{enumerate}
	\item Transakce musí požádat o zámek, než začne pracovat s nějakou enticí
	\item Po uvolnění jakéhokoli zámku nesmí žádat jiný zámek. Všechny držené zámky musí uvolnit
\end{enumerate}


\subsection{Úroveň izolace transakce}
Serializovatelnost garantuje izolaci transakcí ve smyslu podmínky \textbf{ACID}. Je-li plán transakcí serializovalný, neprojeví se negativní vlivy souběhu. Za izolovanost transakcí se platí \textbf{menším výkonem} souběhu $\rightarrow$ \textbf{nižší propustností}. SŘBD umožňuje nastavit úroveň izolace - ta \textbf{sníží míru izolace} transakce a \textbf{zvýší propustnost}.

Pod pojem špinavé čtení spadá i špinavý zápis. Výskyt fantomů nastane v případě:
\begin{itemize}
	\item Zámek probíhá jen na existujícímí enticemi
	\item Pokud provedeme SELECT .. WHERE xxx BETWEEN 1 AND 10 - dostaneme zámek na všechny entice, které jsou v rozsahu
	\item Při vložení nového záznamu s xxx mezi 1 a 10 nebo úpravě jiného kde za xxx bude dosazeno číslo mezi 1 a 10, se při opakovaném čtení objeví fantom. Protože i přes zámek stále dostáváme jiné výsledky, protože dané nové/upravené entice zámek nemají, ale již spadají do podmínky výše.
\end{itemize}
\\\\
\noindent\makebox[\textwidth]{\includegraphics[width=11cm]{assets/izolace}}