\documentclass[11pt]{article}

% Packages
\usepackage[czech]{babel}
\usepackage[utf8]{inputenc}
\usepackage[useregional]{datetime2}
\usepackage[T1]{fontenc}
\usepackage[a4paper, total={15.24cm, 23.32cm}]{geometry}
\usepackage[thinlines]{easytable}
\usepackage{graphicx}
\usepackage[ampersand]{easylist}
\usepackage{changepage}
\usepackage{float}
\usepackage{color}
\usepackage{enumitem}
\usepackage{hyperref}
\usepackage{minted}

% Config
\renewcommand{\baselinestretch}{1.2} 
\setitemize{itemsep=0pt}
\hypersetup{
	colorlinks,
	citecolor=black,
	filecolor=black,
	linkcolor=black,
	urlcolor=black
}
\title{\textbf{II. Softwarové inženýrství}}
\date{\small\vspace{-9ex}Update: \today}
\setminted{fontsize=\small,baselinestretch=1}

\begin{document}
\maketitle
\setcounter{tocdepth}{1}
\tableofcontents
\pagebreak

\section{Modelování databázových systémů, konceptuální modelování, datová analýza, funkční analýza; nástroje a modely.}
\textbf{Softwarové inženýrství} je inženýrská disciplína zabývající se praktickými problémy vývoje
rozsáhlých softwarových systémů.

\section*{Softwarový proces}
\textbf{Softwarový proces} je po částech uspořádaná množina kroků směřujících k vytvoření nebo úpravě softwarového díla.
\begin{itemize}
\item Krokem může být \textbf{aktivita} nebo opět \textbf{podproces} (hierarchická dekompozice procesu). 
\item Aktivity a podprocesy mohou \textbf{probíhat v čase souběžně}, tudíž je vyžadována jejich koordinace. 
\item Je \textbf{nutné zajistit opakovatelnost použití procesu ve vztahu k jednotlivým softwarovým projektům}, tedy zajistit jeho \textbf{znovupoužitelnost}.  Cílem je dosáhnout stabilních výsledků vysoké úrovně kvality.
\item Řada činností je zajišťována lidmi vybavenými určitými schopnostmi a znalostmi a majícím k dispozici technické prostředky nutné pro realizaci těchto činností.
\item \textbf{Softwarový produkt} je realizován v kontextu organizace s danými ekonomickými možnostmi a organizační strukturou.
\end{itemize}

\subsection{Vyspělost úrovně}
Úroveň definice a využití softwarového procesu je hodnocena dle stupnice \textbf{SEI (Software Engineering Institute)} \textbf{1 - 5} vyjadřující vyspělost firmy či organizace z daného hlediska. Tento model hodnocení vyspělosti a schopností dodavatele softwarového produktu se nazývá \textbf{CMM (Capability Maturity Model)} a jeho jednotlivé úrovně lze stručně charakterizovat asi takto:

\begin{enumerate}
\item \textbf{Počáteční (Initial)} - firma \textbf{nemá definován softwarový proces} a každý projekt je řešen \textbf{případ od případu} (ad hoc).
\item \textbf{Opakovatelná (Repeatable)} - firma identifikovala v jednotlivých projektech \textbf{opakovatelné postupy} a tyto je schopna reprodukovat v každém novém projektu.
\item \textbf{Definovaná (Defined)} - softwarový proces je \textbf{definován (a dokumentován)} na základě integrace dříve identifikovaných opakovatelných kroků.
\item \textbf{Řízená (Managed)} - na základě definovaného softwarového procesu je firma schopna jeho \textbf{řízení} a \textbf{monitorování}.
\item \textbf{Optimalizovaná (Optimized)} - zpětnovazební informace získaná \textbf{dlouhodobým procesem monitorování} softwarového procesu je využita ve prospěch jeho optimalizace.
\end{enumerate}

\subsection*{Modely softwarového procesu}
\subsection{Vodopádový model}
Základem téměř všech modelů softwarového procesu se stal vodopádový model. Tento vodopádový model vychází z \textbf{rozdělení životního cyklu softwarového díla} na čtyři základní fáze:
\begin{enumerate}
\item Analýza požadavků a jejich specifikace.
\item Návrh softwarového systému.
\item Implementace.
\item Testování a udržování vytvořeného produktu.
\end{enumerate}
Princip vodopádu spočívá v tom, že \textbf{následující množina činností spjatá s danou fází} nemůže započít dříve, než skončí předchozí. Jinými slovy řečeno, výsledky předchozí fáze „vtékají“ jako vstupy do fáze následující.
\\\\
\noindent\makebox[\textwidth]{\includegraphics[width=12cm]{assets/swi1}}
Model je možno v \textbf{různých modifikacích} a \textbf{rozšířeních} nalézt ve většině současných přístupů. Tyto modifikace vznikly především kvůli odstranění některých jeho \textbf{nedostatků}, mezi které patří:
\begin{itemize}
\item \textbf{Prodleva} mezi zadáním projektu a vytvořením spustitelného systému je příliš dlouhá.
\item \textbf{Výsledek závisí} na \textbf{úplném a korektním zadaní požadavků} kladených na výsledný produkt.
\item \textbf{Nelze odhalit výslednou kvalitu produktu} danou splněním všech požadavků, dokud není výsledný softwarový systém hotov.
\end{itemize}

\subsubsection{Modifikace Vodopádového modelu}
\begin{itemize}
\item \textbf{Inkrementální}: \textbf{postupné vytváření verzí} softwaru zahrnujících postupně širší spektrum funkcí definovaných postupně v průběhu jeho vytváření. V podstatě se jedná o více menších vodopádů provedených za sebou tak, aby každý z nich odpovídal nové sadě doplněných požadavků.
\item \textbf{Spirálový}: zahrnuje do svého \textbf{životního cyklu další fáze} jako je vytvoření a hodnocení \textbf{prototypu} ověřující funkcionalitu cílového systému, přičemž \textbf{každý cyklus nabaluje další požadavky} specifikované zadavatelem.
\end{itemize}


\subsection{RUP (Rational Unified Process)}
Rational Unified Process (RUP) je \textbf{metodika vývoje softwaru} vytvořená společností Rational Software Corporation. Je použitelná pro \textbf{jakýkoliv rozsah projektu}, ale díky vysoké rozsáhlosti RUPu je vhodné přizpůsobit metodiku specifickým potřebám. RUP je vhodnější spíš pro \textbf{rozsáhlejší projekty} a \textbf{větší vývojové týmy}, neboť klade důraz na \textbf{analýzu} a \textbf{design}, \textbf{plánování}, \textbf{řízení zdrojů} a \textbf{dokumentaci}.

\subsubsection{Hlavní znaky}
\begin{itemize}
\item Softwarový produkt je vyvíjen \textbf{iteračním způsobem}.
\item Jsou \textbf{spravovány požadavky} na něj kladené.
\item Využívá se \textbf{již existujících softwarových komponent}.
\item Model softwarového systému je \textbf{vizualizován} (\textbf{UML}).
\item Průběžně je \textbf{ověřována} \textbf{kvalita} produktu.
\item \textbf{Změny} systému\textbf{ jsou řízeny} (každá změna je přijatelná a změny jsou sledovatelné).
\end{itemize}
V současné době, kdy se předmětem vývoje staly softwarové systémy vysoké úrovně sofistikace, je \textbf{nemožné nejprve specifikovat celé zadání}, následně navrhnout jeho řešení, vytvořit softwarový produkt implementující toto zadání, vše otestovat a předat zadavateli k užívání. Jediné možné řešení takovému problému je přístup postavený na \textbf{iteracích}, umožňující \textbf{postupně upřesňovat cílový produkt} cestou jeho \textbf{inkrementálního rozšiřovaní} z původní hrubé formy do výsledné podoby.  Softwarový systém je tak \textbf{vyvíjen ve verzích}, které lze průběžně ověřovat se zadavatelem a případně jej pozměnit pro následující iteraci.


\subsubsection{Cykly, fáze, iterace (Stále se váže k RUP)}
Každý \textbf{cyklus vede k vytvoření takové verze systému}, kterou lze \textbf{předat uživatelům a implementuje jimi specifikované požadavky}. Jak již bylo uvedeno v předchozí kapitole, každý takový vývojový cyklus lze rozdělit do \textbf{čtyř} po sobě jdoucích \textbf{fází}: 
\begin{enumerate}
\item \textbf{Zahájení}, kde je původní myšlenka rozpracována do \textbf{vize koncového produktu} a je definován rámec toho, jak celý systém bude vyvíjen a implementován. 
\item \textbf{Rozpracování} je fáze věnovaná \textbf{podrobné specifikaci požadavků} a \textbf{rozpracování architektury} výsledného produktu. 
\item \textbf{Tvorba} je zaměřena na \textbf{kompletní vyhotovení požadovaného díla}.  Výsledné  programové vybavení je vytvořeno kolem navržené kostry (architektury) softwarového systému. 
\item \textbf{Předání} je závěrečnou fází, kdy \textbf{vytvořený produkt je předán do užívání}. Tato fáze zahrnuje i další aktivity jako je beta \textbf{testování}, \textbf{zaškolení} apod. 
\end{enumerate}
Každá fáze může být dále \textbf{rozložena do několika iterací}.

\subsubsection{Iterace}
Iterace je \textbf{úplná vývojová smyčka vedoucí k vytvoření spustitelné verze systému} reprezentující \textbf{podmnožinu} vyvíjeného cílového produktu, a která je \textbf{postupně rozšiřována každou iterací} až do výsledné podoby. 
\\\\
\noindent\makebox[\textwidth]{\includegraphics[width=9cm]{assets/swi2}}


\pagebreak
\section{Relační datový model, SQL; funkční závislosti, dekompozice a normální formy.}
\subsection{Bezkontextové gramatiky (BG)}
Bezkontextová gramatika definuje \textbf{bezkontextový jazyk}. Je tvořena \textbf{neterminály} (proměnné), \textbf{terminály} (konstanty) a \textbf{pravidly}, které každému neterminálu definují přepisovací pravidla. Jeden neterminál označíme jako \textbf{startovní}, kde začínáme a podle pravidel je dál přepisujeme na výrazy složené z terminálu a neterminálu. Jakmile už není co přepisovat, výraz obsahuje už jen neterminály, získali jsme \textbf{slovo}.

\begin{itemize}
\item Je \textbf{uzavřená} vůči operacím \textbf{sjednocení}, \textbf{zřetězení}, \textbf{iteraci} a \textbf{zrcadlový obraz}.
\item Ke každé bezkontextové gramatice existuje \textbf{ekvivalentní zásobníkový automat}.
\end{itemize}

\subsubsection{Formální definice BG}
Bezkontextová gramatika je definována jako uspořádaná čtveřice $G = (\Pi, \Sigma, S, P)$, kde:
\begin{itemize}
	\item $\Pi$ (\textit{velké pí}) je konečná množina \textbf{neterminálních} symbolů (neterminálů).
	\item $\Sigma$ je konečná množina \textbf{terminálních} symbolů (terminálů), $\Pi \cap \Sigma = \emptyset$.
	\item $S$ je \textbf{počáteční neterminál}, $S \in \Sigma$.
	\item $P$ je konečná množina \textbf{přepisovacích pravidel}, $P \subseteq \Pi \times (\Pi \cup \Sigma)^*$.
\end{itemize}

\subsubsection{Základní pojmy}
\begin{itemize}
\item \textbf{Bezkontextový jazyk} -- formální jazyk, který je akceptovaný nějakým zásobníkovým automatem.
\item \textbf{Derivace slova} -- jedno konkrétní odvození slova pomocí gramatiky, tedy záznam postupných přepisů od startovního neterminálu po konečné slovo. Derivace se podle postupu při přepisování dělí na:
\begin{itemize}
\item \textbf{levou} -- přepisujeme nejprve levé neterminály,
\item \textbf{pravou} -- přepisujeme nejprve pravé neterminály.
\end{itemize}
\item \textbf{Derivačni strom} -- grafické znázornění derivace slova stromem. Pro všechny možné derivace (levou, pravou, moji) by měl derivační strom být \textbf{stejný}. Není-li tomu tak jedná se o \textbf{nejednoznačnou gramatiku}, což je nežádoucí jev. 
\begin{itemize}
	\item \textbf{Špatně} = A $\rightarrow$ A | $\epsilon$ (lze generovat až N způsoby), \textbf{Správně} = A $\rightarrow \epsilon$ 
\end{itemize}
\item \textbf{Chomského normální forma} -- gramatika může obsahovat pouze pravidla typu: \textbf{A $\rightarrow$ BC} nebo \textbf{A $\rightarrow$ a} nebo \textbf{S $\rightarrow \epsilon$} (pokud gramatika generuje pouze prázdný řetězec).
\item \textbf{Nevypouštějící gramatika} -- neobsahuje $\epsilon$ (\textit{epsilon}) přechody.
\end{itemize}

\begin{figure}[H]
	\centering
	\includegraphics[width=0.6\textwidth]{assets/bg}
\end{figure}

\subsection{Zásobníkové automaty (ZA)}
Slouží k \textbf{rozpoznání bezkontextových jazyků}. S využitím zásobníků si může pamatovat kolik a jaké znaky přečetl, což je potřeba právě k rozpoznání bezkontextového jazyka. Zásobníkový automat je v podstatě konečný automat rozšířený o zásobník. 

\begin{figure}[H]
	\centering
	\includegraphics[width=0.30\textwidth]{assets/za}
	\includegraphics[width=0.30\textwidth]{assets/cfl}
\end{figure}

\begin{itemize}
\item ZA na základě \textbf{aktuálního znaku} na pásce, \textbf{prvního znaku v zásobníku} a \textbf{aktuálního stavu} změní svůj stav a \textbf{přepíše} znak v zásobníku podle daných pravidel.
\item ZA \textbf{přijímá} dané slovo, jestliže skončí v konfiguraci $(q, \epsilon, \epsilon)$, tedy když se přečte celé vstupní slovo a zásobník je \textbf{prázdný}.
\item \textbf{Konfigurace} je dána: aktuálním stavem, obsahem pásky a obsahem zásobníku.
\item \textbf{Deterministický} -- nesmí se objevit dvě pravidla se stejnou levou stranou, a pokud existuje
pravidlo $(q, \epsilon, X)$, tak nesmí zároveň existovat pravidlo $(q, a, Y)$. Pokud se deterministický ZNKA ocitne
ve stejné konfiguraci více než jednou, tak obsahuje nekonečný cyklus.
\end{itemize}

\subsubsection{Formální definice zásobníkového automatu}
Zásobníkový automat $M$ je definován jako šestice $M = (Q, \Sigma, \Gamma, \delta, q_0, Z_0)$, kde:
\begin{itemize}
\item $Q$ je konečná neprázdná množina \textbf{stavů}.
\item $\Sigma$ je konečná neprázdná množina \textbf{vstupních symbolů} (vstupní abeceda).
\item $\Gamma$ (\textit{velká gamma}) je konečná neprázdná množina \textbf{zásobníkových symbolů}.
\item $\delta$ je \textbf{přechodová funkce} (konečná množina instrukcí), $\delta: Q \times (\Sigma \cup \{\epsilon\}) \times \Gamma \rightarrow P_{\rm fin}(Q \times \Gamma^*)$.
\item $q_0$ je \textbf{počáteční stav}, $q_0 \in Q$.
\item $Z_0$ je \textbf{počáteční zásobníkový symbol}, $Z_0 \in \Gamma$.
\end{itemize}

\subsubsection{Definice instrukcí (pravidel) v ZA}
Instrukce (sady instrukcí reprezentují přechodovou funkci $\delta$) definují \textbf{chování automatu}:
\begin{equation}
(q, a, X) \rightarrow (q', \alpha)\textrm{, kde } a\in \Sigma.
\end{equation}
Tato instrukce je aplikovatelná jen v situaci (neboli konfiguraci), kdy \textbf{řídicí jednotka} je ve stavu $q$, \textbf{čtecí hlava} na vstupní pásce čte symbol $a$ a na vrcholu zásobníku je symbol $X$. Pokud je \textbf{instrukce aplikována}, vykoná se následující:
\begin{enumerate}
\item řídicí jednotka \textbf{přejde do stavu} $q'$,
\item čtecí hlava na vstupní pásce se \textbf{posune o jedno políčko doprava},
\item vrchní symbol v zásobníku se \textbf{odebere} (vymaže),
\item \textbf{na vrchol zásobníku se přidá} řetězec $\alpha$ tak, že jeho nejlevější symbol je aktuálním vrcholem zásobníku.
\end{enumerate}

\begin{table}[H]
	\vspace{-2mm}
	\centering
	\begin{tabular}{l|l|p{6.5cm}}
		\textbf{Pravidlo} & \textbf{Akce (Z = zásobník)} & \textbf{Význam} \\\hhline
		$ \delta(q_1, a, X) \rightarrow (q_1, YX) $ 	&                 \textbf{přidání} prvku do Z & na začátek zásobníku se vloží $Y$ \\ 
		$ \delta(q_1, a, X) \rightarrow (q_1, Y) $	&                  \textbf{přepsání} prvku v Z & první prvek zásobníku se přepíše na $Y$ \\ 
		$ \delta(q_1, a, X) \rightarrow (q_1, \epsilon) $	&                  \textbf{smazání} prvku ze Z & první prvek zásobníku se smaže neboli nahradí prázdným slovem $\epsilon$ \\ 
		$ \delta(q_1, a, X)\rightarrow(q_2, X) $	&                  \textbf{změna stavu}& stav $ q_1 $ se změní na stav $ q_2 $ \\ 
		$\delta(q_1, a, X)\rightarrow\emptyset$	&                  \textbf{pád} automatu & ukončení výpočtu, slovo nebylo přijato \\ 
	\end{tabular}
\end{table}

\subsection{Převod BG na zásobníkový automat}
Využívá se tzv. metody shora-dolů, která obsahuje pouze \textbf{1 stav}:
\begin{enumerate}
\item pro všechny \textbf{neterminály} vypíšu pravidla typu: $(q, \epsilon, A) \rightarrow \{(q, B), (q, C)\}$,
\item všechny \textbf{terminály} přepíšu na pravidla typu: $(q, a, a) \rightarrow (q, \epsilon)$.
\end{enumerate}

\noindent\\\begin{minipage}[t]{0.35\textwidth}
\textbf{Vstupní gramatika:}\\
$S \rightarrow A | B$\\
$A \rightarrow a$\\
$B \rightarrow (c)$\\\smallskip\\
$\Sigma = \{A, B, S\}$\\
$\Gamma = \{a, c, (, )\}$
\end{minipage}
\begin{minipage}[t]{0.65\textwidth}
\textbf{Instrukce, převedené dle výše uvedených pravidel:}\\
$(Q, \epsilon, S) \rightarrow \{(q, A), (q, B)\}$\\
$(Q, \epsilon, A) \rightarrow (q, a)$\\
$(Q, \epsilon, B) \rightarrow (q, (c))$\\\smallskip\\
$(Q, a, a) \rightarrow (q, \epsilon)$\\
$(Q, (, () \rightarrow (q, \epsilon)$\\
$(Q, c, c) \rightarrow (q, \epsilon)$\\
$(Q, ), )) \rightarrow (q, \epsilon)$\\
\end{minipage}


\pagebreak
\section{Transakce, zotavení, log, ACID, operace COMMIT a ROLLBACK; problémy souběhu, řízení souběhu: zamykání, úroveň izolace v SQL.}

\pagebreak
\section{Procedurální rozšíření SQL, PL/SQL, T-SQL, triggery, funkce, procedury, kurzory, hromadné operace.}

\pagebreak
\section{Základní fyzická implementace databázových systémů: tabulky a indexy; plán vykonávání dotazů.}

\pagebreak
\section{Objektově‐relační datový model a XML datový model: principy, dotazovací jazyky.}

\pagebreak
\section{Datová vrstva informačního systému; existující API, rámce a implementace, bezpečnost; objektově-relační mapování.}

\pagebreak
\section{Distribuované SŘBD, fragmentace a replikace.}

\end{document}
